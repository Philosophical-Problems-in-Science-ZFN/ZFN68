\begin{artplenv}{Łukasz Mścisławski}
	{Eter, stara teoria kwantów, filozofia przyrody i~problematyka ciągłości w~ujęciu Bohdana Szyszkowskiego. Przyczynek do badań nad recepcją starej teorii kwantów i~mechaniki kwantowej na ziemiach polskich przed rokiem 1953}
	{Eter, stara teoria kwantów, filozofia przyrody i~problematyka\ldots}
	{Eter, stara teoria kwantów, filozofia przyrody i~problematyka ciągłości w~ujęciu Bohdana Szyszkowskiego}
	{Politechnika Wrocławska}
	{Ether, the old quantum theory, philosophy of nature and the problem of continuity as defined by Bogdan Szyszkowski. Contribution to research on the reception of the old quantum theory and quantum mechanics in Poland before 1953.}
	{The aim of this paper is to present the views of one of prominent Polish chemists, namely Bohdan Szyszkowski. Presented views are concerning the relation of the concepts of aether, continuity and causality in the context of revolution in physics that took place at the turn of 19th and 20th centuries. The issues particularly concern his thoughts related to the old quantum theory and special relativity. In particular, this article presents the role of these concepts in the foundations of physics and -- in a~more general aspect -- in the recognized fields of knowledge. The nature of Szyszkowski's analysis allows him to be considered a~very interesting thinker in the area that today is called the philosophy of physics or, perhaps more accurately, the philosophy \textit{in} science. An important observation made by Szyszkowski is emphasized that mathematical structures cannot have physical properties. Attention was also paid to the Polish intellectual community, which was formed in Kiev before 1919.}
	{Szyszkowski, Kiev, natural philosophy, philosophy of physics, aether, continuity, old quatnum theory, quantum mechanics, special theory of relativity, causality.}


\begin{flushright}
\textit{Człowiek [...] może uczynić przyszłość bujną i~gigantyczną,\\
o~ile będzie myślał o~przeszłości.\\}
G. K. Chesterton
\end{flushright}


Historia recepcji mechaniki kwantowej przed wybuchem II wojny światowej na ziemiach polskich jest mało znana. Jest to jednak tematyka intrygująca, zwłaszcza, jeśli rozciągnąć całość zagadnienia nie tylko na terytorium Polski -- po uzyskaniu przez nią niepodległości w~roku 1918 -- lecz biorąc pod uwagę także poszczególnych uczonych, filozofów czy myślicieli pochodzenia polskiego, niezależnie od miejsca ich pracy czy języka, w~którym te prace były publikowane. Niniejszy artykuł stanowi skromny początek próby zaradzenia wspomnianemu niedostatkowi. Pewien ramowy szkic badań w~tym zakresie, które podjął autor niniejszej pracy, zostanie przedstawiony w~zakończeniu. W~tym miejscu wypada zwrócić uwagę na to, że zwykle w~dyskusjach nad recepcją mechaniki kwantowej, a~zwłaszcza jej filozoficznymi implikacjami, największym zainteresowaniem cieszy się okres już po jej ukształtowaniu się jako dojrzałej teorii, czyli po roku 1924\footnote{Rok ten podany jest symbolicznie, w~nawiązaniu do stwierdzenia Czesława Białobrzeskiego, który stwierdza, że rok 1925 można uznać za przełomowy ze względu na sformułowanie \textit{nowej mechaniki}. Białobrzeski zaznacza także, że lata 1924-1926 należałoby uznać za okres czas budowania nowych podstaw fizyki przez de Broglie, Heisenberga, Schrödingera i~Diraca
%\label{ref:RNDhPX4VytAVH}(por. Białobrzeski, 1937, s.~244–245).
\parencite[por.][s.~244–245]{bialobrzeski_ogolnonaukowe_1937}. %
 W~artykule tym warszawski fizyk zwraca uwagę na to, że ,,najbardziej zdumiewającym wynikiem ich dzieła jest usunięcie determinizmu, a~więc i~przyczynowości z~przebiegu elementarnych procesów atomowych. Rzeczą godną uwagi jest to, że nikomu z~nich nie przychodziło na myśl wprowadzać do fizyki indeterminizmu'' (tamże, s.~245). Ta charakterystyka fizyki kwantowej jest bardzo znamienna, a~sam temat zostanie jeszcze poruszony, choć w~nieco odmiennym kontekście, w~ramach niniejszego opracowania.}.

Nie oznacza to jednak, że okres poprzedzający tę cezurę nie jest wart zainteresowania. Przeciwnie, zdaje się on być wyjątkowo niedoceniany, zwłaszcza w~kontekście recepcji starej teorii kwantów przez naukowców i~filozofów polskich. Interesującym może wydawać się także fakt, że dotychczasowe próby uwzględnienia tego procesu (głównie w~odniesieniu do mechaniki kwantowej po roku 1925), ograniczały się w~zasadzie do uwzględnienia wzmianek na ten temat w~pracach fizyków i~filozofów. Nie jest jednak prawdą, że zainteresowanie tą tematyką ogranicza się tylko do tego grona (zarówno przed, jak i~po roku 1924), co zostanie pokazane w~ciągu dalszym tego artykułu. Całość tematu warta jest znacznie szerszego opracowania (które autor zamierza w~przyszłości podjąć), tutaj wskazany przykład, że -- mimo dość niewielkiego dotąd zainteresowania -- problematyka ta wydaje się być bardzo interesująca i~sprawiająca wiele niespodzianek. W~zakresie recepcji, przynajmniej w~środowiskach naukowych zagranicznych (krajów Europy Zachodniej), rozwój pojęciowy zarówno starej teorii kwantów, jak i~mechaniki kwantowej, został skrupulatnie przestudiowany\footnote{Wystarczy wspomnieć klasyczne już pozycje:
%\label{ref:RND8LDREylqyk}(Jammer, 1966)
\parencite[][]{jammer_conceptual_1966} %
 czy 
%\label{ref:RNDWnjrB46MxG}(Kragh, 1999).
\parencite[][]{kragh_quantum_1999}.%
}. Jeśli jednak chodzi o~historyczne ujęcie recepcji w~zakresie implikacji filozoficznych sytuacja nie wydaje się już jednak satysfakcjonująca, jeśli chodzi o~źródła polskojęzyczne czy też autorów polskich. Sytuacja jest jednak analogiczna, jeśli chodzi także o~inne kraje
europejskie\footnote{Por. np. 
%\label{ref:RNDQM1WEuRcMF}(Bensaude-Vincent, 1988).
\parencite[][]{bensaude-vincent_when_1988}. %
 Bensaude-Vincent odnosi się co prawda do realiów francuskich, niemniej to samo można także powiedzieć w~odniesieniu do historii recepcji starej teorii kwantów i~mechaniki kwantowej wśród uczonych i~myślicieli polskich (niezależnie od tego, w~jakim języku powstawały ich prace). Brakuje jednak jednolitych opracowań tej tematyki, najczęściej przytaczane są przykłady analiz, które są dojrzałe i~ukształtowane, ale opublikowane w~głównej mierze po drugiej wojnie światowej. Inne ograniczają się w~zasadzie do przytaczania poglądów Czesława Białobrzeskiego, lub traktują tematykę implikacji filozoficznych starej teorii kwantów lub mechaniki kwantowej w~pewnym sensie marginalnie. Por. np. 
%\label{ref:RNDAjIAeifpS4}(Dąbrowski, 1980),
\parencites[][]{dabrowski_o_1980}[][]{kostro_philosophie_1969}[][]{heller_krakowska_2007}[][]{heller_krakowska_2007-1}[czy][]{mscislawski_miedzy_2017}.}
%%\label{ref:RNDrTE6WnrAJD}(Kostro, 1969),
%\parencite[][]{kostro_philosophie_1969},%
% 
%%\label{ref:RNDSCSHPcAsIN}(Heller, Mączka, 2007a),
%\parencite[][]{heller_krakowska_2007},%
% 
%%\label{ref:RND6950iYXEUF}(Heller, Mączka, 2007b)
%\parencite[][]{heller_krakowska_2007-1}%
% czy 
%%\label{ref:RNDHRV30Nn5gF}(Mścisławski, 2017).
%\parencite[][]{mscislawski_miedzy_2017}.%
W~tym miejscu należy podkreślić, że w~odniesieniu do teorii względności (zwłaszcza szczególnej teorii względności), sprawa przedstawia się znacznie lepiej\footnote{Głównie dzięki bardzo dobrej monografii 
%\label{ref:RNDSUdpxuxCBC}(Polak, 2012).
\parencite[][]{polak_bylem_2012}.%
}.

Trzeba także zwrócić uwagę, że w~odniesieniu do zagadnień recepcji teorii naukowych (a także ich możliwych implikacji) przed 1939 rokiem, najczęściej -- oczywiście w~przypadku polskich uczonych i~filozofów -- przedstawiane są środowiska związane z~Krakowem, Warszawą i~Lwowem. Tymczasem bardzo interesujące może okazać się opracowanie zagadnień związanych ze środowiskiem polskim, które wytworzyło się w~Kijowie\footnote{Dogłębne studium dotyczące środowiska polskiego w~Kijowie w~pierwszym dwudziestoleciu XX wieku można znaleźć w~monografii
%\label{ref:RNDqyUo0i5xF7}(Korzeniowski, 2009).
\parencite[][]{korzeniowski_za_2009}. %
 Działalność oświatowa i~intelektualna (choć nie tylko) tego środowiska jest doskonale opisana tamże na s.~143-432.}. W~niniejszym artykule -- właśnie w~kontekście tego środowiska, którego przedstawiciele okazali się nie tylko wybitnymi uczonymi, lecz także ludźmi o~niezwykłej intuicji i~zainteresowaniach filozoficznych, zaprezentowana zostanie problematyka związana z~zagadnieniem ciągłości, powstała w~kontekście pojawienia się nowych teorii fizycznych: szczególnej teorii względności i~starej teorii kwantów. Niniejsza praca stanowi prezentację poglądów w~tym zakresie jednej z~dwóch najważniejszych -- jak się wydaje -- postaci jeśli chodzi o~tę tematykę. Chodzi mianowicie o~refleksje nad tym zagadnieniem, przeprowadzone przez chemika Bohdana Szyszkowskiego\footnote{Bohdan Szyszkowski (20~VI~1873–13~VIII~1931), chemik, wykładowca Uniwersytetu Św. Włodzimierza w~Kijowie oraz Zakładu Fizycznego Kijowskiego Instytutu Politechnicznego, prorektor Polskiego Kolegium Uniwersyteckiego w~Kijowie, Uniwersytetu Jagiellońskiego i~Akademii Górniczej w~Krakowie (od 1947 roku Akademia Górniczo-Hutnicza), członek Polskiego Towarzystwa Chemicznego oraz członek korespondencyjny Wydziału Matematyczno-Przyrodniczego Polskiej Akademii Umiejętności. Drugą najważniejszą postacią polskiego środowiska intelektualnego w~Kijowie związaną z~tym obszarem refleksji jest Czesław Białobrzeski.}.

Nie sposób nie zgodzić się ze słowami Stefana Zameckiego
%\label{ref:RND9azdIy9Lsq}(1998, s.~149),
\parencite*[][s.~149]{zamecki_bohdan_1998}, %
 że o~Bohdanie Szyszkowskim w~Polsce napisano z~jednej strony na tyle dużo, że można stwierdzić wyjątkowość tej postaci, a~jednocześnie zbyt mało aby mógł on zaistnieć w~obecnej świadomości społeczeństwa. Świętosławski 
%\label{ref:RNDI0YSJSRnXC}(1931)
\parencite*[][]{swietoslawski_sp_1931} %
 charakteryzuje go jako człowieka o~,,gorącym sercu i~subtelnych uczuciach'',bogato uposażonego intelektualnie, który potrafił połączyć chłodną pracę naukową z~uczuciowością rozwiniętą do wysokiego stopnia. Widzi go jako postać niezwykle barwną i~głęboką. Skąpski pisze o~nim: ,,Ścisły uczony empiryk, a~zarazem artysta i~filozof, pracujący równocześnie nad parallelą Słowackiego i~Bergsona i~nad najściślejszym prawem empirycznem rozcieńczeń, a~przedewszystkiem europejczyk i~gentelman w~każdym calu -- takim był Bohdan Szyszkowski''\footnote{Por. 
%\label{ref:RNDrALVuQvd26}(Skąpski, 1931).
\parencite[][]{skapski_charakterystyka_1931}. %
 We wszystkich cytatach pozostawiono pisownię oryginalną. Zainteresowanie Szyszkowskiego próbą zestawienia twórczości Słowackiego i~Bergsona znalazła swój wyraz w~pracach: 
%\label{ref:RNDj2AjPEotX5}(Szyszkowski, 1920a; Szyszkowski, 1920c; Szyszkowski, 1920b; Szyszkowski, 1921).
\parencites[][]{szyszkowski_slowacki_1920-1}[][]{szyszkowski_slowacki_1920}[][]{szyszkowski_slowacki_1920-2}[][]{szyszkowski_slowacki_1921}.%
}. Skąpski zwraca uwagę, że uwzględnienie tylko prac doświadczalnych Szyszkowskiego w~zakresie chemii dawałoby niepełny obraz tej nietuzinkowej osobowości. W~obrazie tym podkreśla on, między innymi także wagę w~tym obrazie prac o~charakterze, jak to określa, \textit{filozoficzno-naukowym}. Wspomina też, że Szyszkowski dał się poznać jako miłośnik filozofii przyrody -- zwłaszcza w~okresie krakowskim, który obejmuje lata 1920-1931, czyli od momentu przybycia do Krakowa aż do śmierci uczonego. Wypada jednak zaznaczyć, że predylekcje filozoficzne krakowskiego chemika ujawniły się już w~czasie jego pobytu w~Kijowie, przypadającego na lata 1892-1920\footnote{W~roku 1892 Szyszkowski przeniósł się w~Wydziału Fizyko-Matematycznego w~Odessie na analogiczny wydział w~Kijowie 
%\label{ref:RNDdaromYHA8T}(Sroka, 2002).
\parencite[][]{sroka_szyszkowski_2002}. %
 Jest bardzo prawdopodobne, że filozoficzne zainteresowania pojawiły się u~Szyszkowskiego jeszcze wcześniej, niemniej trudno tę tezę oprzeć o~jakiekolwiek źródła.}. Wiadomo, że w~ramach Wyższych Polskich Kursów Naukowych w~roku 1917 prowadził on wykłady i~ćwiczenia ,,Z zagadnień filozofji przyrody''\footnote{Por. 
%\label{ref:RNDFDtrkxySLL}(Kamiński, 1917),
\parencite[][]{kaminski_wyzsze_1917}, %
 a~także 
%\label{ref:RNDoa0PPFCWsp}(Korzeniowski, 2009),
\parencite[][s.~229-236]{korzeniowski_za_2009}. %
Korzeniowski podkreśla dużą popularność tych kursów wśród Polaków zamieszkujących wówczas Kijów.}.Niestety, według wiedzy autora, jeśli chodzi o~twórczość Szyszkowskiego z~okresu kijowskiego, prac o~charakterze filozoficznym, zwłaszcza dotyczących filozofii przyrody, nie zachowało się zbyt wiele. Za szczególnie cenny w~tym względzie należy więc uznać artykuł \textit{O~ciągłości}, który ukazał się w~\textit{Przeglądzie Naukowym i~Pedagogicznym} 
%\label{ref:RNDSyOr887a1I}(Szyszkowski, 1916).
\parencite[][]{szyszkowski_o_1916}. %
 Do pewnego stopnia, uzupełnieniem obrazu jego zainteresowań w~zakresie filozofii przyrody z~omawianego okresu mogą być zachowane notatki ze wspomnianych już wykładów z~,,Z zagadnień filozofji przyrody''\footnote{Rękopis znajduje się w~Bibliotece Narodowej w~Warszawie: 
%\label{ref:RNDDC4fjQyrlo}([Noty z~wykładów na Wyższych Polskich Kursach Naukowych w~Kijowie], 1917).
\parencite*[][]{noauthor_noty_1917}. %
 Notatki z~wykładów Szyszkowskiego znajdują się na kartach: 211-244 i~251-267.}. Ten ostatni materiał jest niejednorodny, wydaje się, że notatki sporządzone zostały przez różnych słuchaczy. Pełniejsze, krytyczne opracowanie tych notatek może okazać się także źródłem ważnych przyczynków do badań nad recepcją myśli Szyszkowskiego, wykracza jednak poza zakres niniejszego opracowania.

\section{Ciągłość i~przyczynowość}
Artykuł \textit{O~ciągłości}, zamieszczony przez Szyszkowskiego w~\textit{Przeglądzie Naukowym i~Pedagogicznym}\footnote{
%\label{ref:RND80VZtUkqxh}(Szyszkowski, 1916).
\parencite[][]{szyszkowski_o_1916}.%
} chronologicznie wyprzedza wykłady ,,Z zagadnień filozofji przyrody''. Impulsem do jego napisania była sytuacja zagrożenia zamętem w~fizyce związanego z~zachwianiem wiary w~jej podstawy, co spowodowane było szeregiem zupełnie nowych odkryć i~dynamicznym rozwojem teorii na początku XX wieku. To zaś, zdaniem autora, doprowadziło do sytuacji, w~której ,,trzeba było szukać nowych dróg i~wśród rozmaitych kierunków wybrać ten, który najprościej prowadził ku rozwiązaniu zagadki bytu'' 
%\label{ref:RND4zAzRP3mS6}(Szyszkowski, 1916, s.~44).
\parencite[][s.~44]{szyszkowski_o_1916}. %
 Tezę tę oparł on na wypowiedziach m.in. Thomsona, Lorentza i~Rayleigha, które miały miejsce w~ramach Zjazdu Brytyjskiego Towarzystwa dla Postępu Nauki\footnote{British Association for the Advancement of Science, obecnie British Science Association.} w~Birmingham w~dniach od 10 do 17 września 1913 roku 
%\label{ref:RNDvVNTGDLv5D}(por. British Association for the Advancement of Science, 1914).
\parencite[por.][]{british_association_for_the_advancement_of_science_report_1914}. %
 Według tych uczonych, całokształt współczesnej im wiedzy w~zakresie fizyki nie może być ujęty w~logiczny system. Szyszkowski stwierdza, że prowadzi to do wniosku, iż istnieje cały szereg zjawisk, których człowiek wytłumaczyć nie potrafi. A~to jest, jego zdaniem, cios dla całości przedsięwzięcia naukowego, który wymusza rewizję jego fundamentów, począwszy od sprawy najbardziej zasadniczej -- relacji między \textit{poznającym umysłem a~poznawaną przyrodą} 
%\label{ref:RNDF5fQ8OzGVh}(Szyszkowski, 1916, s.~44)
\parencite[][s.~44]{szyszkowski_o_1916}%
\footnote{Trzeba zauważyć, że Szyszkowski nie precyzuje tutaj, czy chodzi o~sam sposób poznawania czy też o~treść, którą umysł poznaje, czy też i~jedno i~drugie.}. Temat ten jest godny zauważenia i~podkreślenia, gdyż relacje te wcale nie są oczywiste. Więcej nawet, w~analogicznym kontekście (w przypadku Szyszkowskiego -- jest to rodząca się teoria kwantów), w~przyszłości powstanie problem relacji obserwatora do obserwowanego układu, stanowiący jeden z~trudniejszych problemów interpretacyjnych mechaniki kwantowej (problem pomiaru). Trzeba przy tym zaznaczyć, że sam Szyszkowski był stosunkowo dobrze zorientowany we współczesnych mu kierunkach badań w~fizyce\footnote{Por. np. 
%\label{ref:RNDfhJtvwXRCc}(Szyszkowski, 1911b),
\parencite[][]{szyszkowski_1911-1}, %
 czy nieco późniejsza praca 
%\label{ref:RNDhYvbY9oGZw}(Szyszkowski, 1918),
\parencite[][]{szyszkowski_1918}, %
 która stanowi próbę wykorzystania w~chemii najnowszych znanych Szyszkowskiemu osiągnięć z~zakresu fizyki atomowej, co sugeruje bardzo dobrą orientację w~tej tematyce. Podkreślenia wymaga fakt, że Szyszkowski słynął z~dobrej orientacji w~bieżących osiągnięciach fizyki atomowej 
%\label{ref:RND25wXjdqvt8}(por. np. Świętosławski, 1931, s.~784).
\parencite[por. np.][s.~784]{swietoslawski_sp_1931}. %
 Tekst omawianego artykułu zastanawia jednak czasami brakiem odniesień do różnych ważnych wyników badań z~zakresu fizyki, o~czym później.}.
 Zasadniczo widzi on dwie możliwości wyjścia z~impasu. Pierwsza z~nich polegałaby na analizie fundamentalnych pojęć stosowanych w~fizyce i~ocenie ich \textit{względnej} wartości\footnote{Szyszkowski niestety nie rozwija tej myśli. Jakkolwiek nie pisze tego wprost, to zważywszy na jego zainteresowanie filozofią Henri Bergsona, można przypuszczać, iż ma tu na myśli konwencjonalistyczne ujęcie pojęć używanych w~nauce. Wypada jednak zaznaczyć, że bardzo trudno jest dokładnie wskazać moment, w~którym Szyszkowski zainteresował się myślą francuskiego intuicjonisty.}. Drugą możliwość widzi on w~próbie znalezienia ogólnej zasady, która ,,pozwoliłaby wytłumaczyć niedokładność i~niedoskonałość spółczesnego układu fizyki'' 
%\label{ref:RND0sXWbhNEzA}(Szyszkowski, 1916, s.~44).
\parencite[][s.~44]{szyszkowski_o_1916}. %
 Szyszkowski wybiera ten drugi sposób refleksji nad kryzysem w~fizyce, zaznaczając, że skłoniło go do tego wystąpienie Olivera Lodge',a, które miało miejsce na wspomnianym Zjeździe Brytyjskiego Towarzystwa dla Postępu Nauki. Lodge, jak stwierdza Szyszkowski, wskazał, jako remedium na ówczesny kryzys w~zakresie podstaw fizyki i~jej ochrony przed chaosem, zwrócenie uwagi na pojęcie ciągłości\footnote{Por. 
%\label{ref:RNDD9U61CNrSa}(Szyszkowski, 1916, s.~44–45).
\parencite[][s.~44–45]{szyszkowski_o_1916}. %
 Treść wystąpienia Olivera Lodge',a jest zamieszczona w~
%\label{ref:RNDVyIBD548hK}(British Association for the Advancement of Science, 1914, s.~3–42, o~ciągłości zwłaszcza s.~5-10).
\parencite[][o~ciągłości zwłaszcza s.~5-10]{british_association_for_the_advancement_of_science_report_1914}.%
}.

Trzeba w~tym miejscu podkreślić, co zauważa również Szyszkowski, swoistą ,,interdyscyplinarność'' wystąpienia Lodge',a, dotykającego bardzo rozległego spektrum zagadnień, dotyczących fizjologii, poprzez chemię, biologię, ekonomię i~nauki społeczne, na fizyce kończąc
%\label{ref:RNDlzqmxSa1bW}(Lodge, 1914, s.~5),
\parencite[][s.~5]{lodge_continuity_1914}, %
 i~poszukującego możliwie szerokiego spojrzenia na rzeczywistość, obejmującego także obszary także nie podlegające opisowi nauk pozytywnych 
%\label{ref:RNDw1kKcTHoOE}(por. Lodge, 1914, s.~6).
\parencite[por.][s.~6]{lodge_continuity_1914}. %
 Jakkolwiek wydawało się, że wystąpienie Lodge',a nie spowodowało większego poruszenia czy reakcji\footnote{Może z~wyjątkiem swego rodzaju dezorientacji, a~nawet niezrozumienia celu, który przyświecał Lodge',owi 
%\label{ref:RNDEGuPZRi465}(por. Szyszkowski, 1916, s.~45).
\parencite[por.][s.~45]{szyszkowski_o_1916}. %
 Szyszkowski zwraca uwagę na ten brak reakcji dodając, że trudno zorientować się od razu w~całości materiału przedstawionego przez brytyjskiego fizyka. }, dla Szyszkowskiego stanowiło swoisty impuls do własnych poszukiwań w~tym zakresie. Objęły one zarówno własną analizę przemian zachodzących w~ówczesnej fizyce, jak i~rozmowy z~Lodgem oraz intensywne studia nad jego pracami i~dziełami innych angielskich myślicieli (nie podaje jednak, o~jakich myślicieli chodzi). Sam artykuł \textit{O~ciągłości} ma stanowić -- wedle słów jego autora -- próbę przedstawienia zarysu ogólnego obrazu świata fizycznego i~duchowego (nie precyzując rozumienia ostatniego), widzianego w~świetle pojęcia ciągłości. Powstawał on u~Szyszkowskiego w~miarę rozwoju jego własnych badań filozoficznych, niekiedy dotykających zagadnień fundamentalnych dla możliwości funkcjonowania jakichkolwiek dyscyplin naukowych. Najważniejszą rolę w~tym obrazie odgrywa właśnie pojęcie ciągłości oraz pojęcie eteru, specyficznie przez Szyszkowskiego rozumiane.

\subsection{Rola pojęcia ciągłości}

Swoje refleksje na temat roli pojęcia ciągłości Szyszkowski rozpoczyna od potocznego doświadczenia, które zasadniczo wskazuje na brak ciągłości w~budowie świata materialnego. Zwraca uwagę, że układy, które wydają się ciągłe, w~rzeczywistości takimi nie są, stanowiąc zbiór ogromnej liczby mniejszych składników. Zatem -- jego zdaniem -- bezpośrednia obserwacja świata materialnego prowadzi do wniosku, że świat ten ma naturę nieciągłą
%\label{ref:RNDnbNvhhyIqt}(por. Szyszkowski, 1916, s.~45–46)
\parencite[por.][s.~45–46]{szyszkowski_o_1916}%
\footnote{Podkreśla on też, że najwspanialszym przykładem nieciągłości budowy materii jest \textit{zjawisko Smoluchowskiego}. Trudno jednak wywnioskować, czy chodzi tu o~zjawisko znane także pod nazwą ruchów Browna, czy też o~jeszcze inne zjawisko (np. opalescencji).}. Tam, gdzie bezpośrednia obserwacja zawodzi ze względu na ograniczone możliwości poznawcze zmysłów człowieka (nawet przy rozszerzeniu ich możliwości przy pomocy różnego rodzaju narzędzi, np. mikroskopu), trzeba uciec się do umiejętności wnioskowania. Ze współpracy doświadczenia i~wnioskowania w~zakresie badania materii wyłania się -- twierdzi Szyszkowski -- przekonanie będące podstawą chemii i~fizyki, że materia składa się ostatecznie z~atomów, których widzieć nie można, ale można obserwować skutki, które one sprawiają\footnote{Jako przykłady podaje Szyszkowski zliczanie cząstek alfa.} . Budowa atomów jednakże też nie jest ciągła -- można wyróżnić ich drobniejsze elementy składowe. Innymi słowy, eksperymenty sugerują nieciągłość materii.

Po nakreśleniu wniosków, które można wysnuć z~doświadczenia zmysłowego (także rozszerzonego o~najnowsze, współczesne mu przyrządy pomiarowe) Szyszkowski wygłasza bardzo ważną tezę: przeciwieństwem nieciągłej budowy materii jest ,,układ naszego umysłu, któremu pojęcie ciągłości jest wrodzone''
%\label{ref:RNDvLuoZxMF07}(Szyszkowski, 1916, s.~47).
\parencite[][s.~47]{szyszkowski_o_1916}. %
 Jest to stwierdzenie, które przyjmuje on za oczywiste do tego stopnia, że (niestety) nie podaje ani uzasadnienia, ani poszlak czy argumentów, które mogłyby czytelnikowi ułatwić zrozumienie czy obronę tej tezy. Co więcej, nie wiadomo, co dokładnie oznacza ciągłość w~odniesieniu do umysłu. Zauważa także: ,,Na nim [tj. na pojęciu ciągłości], jak na kamieniu węgielnym, opiera się rachunek różniczkowy i~całkowy, za pomocą którego umysł nasz pojmuje i~poznaje zjawiska przyrody i~układa je w~logiczny system''. Równanie różniczkowe stanowi precyzyjny opis zjawiska w~dowolnej chwili jego przebiegu i~stanowi ,,niejako system przesłanego logicznych, opartych na obserwacji, z~których za pomocą specjalnej formy wnioskowania, zwanej całkowaniem, tworzy wniosek w~postaci prawa przyrody'' 
%\label{ref:RNDIsTGJqYIS7}(Szyszkowski, 1916, s.~47).
\parencite[][s.~47]{szyszkowski_o_1916}. %
 Szyszkowski stwierdza także, że brak ciągłości -- patrząc od strony opisu matematycznego -- odpowiadałby sytuacji, w~której relacja przyczynowości (relacja przyczyna-skutek) dla danego prawa przyrody przestaje istnieć\footnote{Szyszkowski ujmuje to następująco: ,,Gieometrycznie każde prawo przyrody może być wyrażone za pomocą pewnej krzywej. Nieciągła krzywa [...] oznaczała by matematycznie, iż w~jakiejkolwiek chwili stosunek pomiędzy przyczyną i~skutkiem, który był na ciągłej części krzywej określony, raptem przestaje istnieć, gdyż nieskończenie mała zmiana w~przyczynie wywołuje nieskończenie wielką zmianę w~skutku'' 
%\label{ref:RNDoZ8svMXdWI}(Szyszkowski, 1916, s.~47).
\parencite[][s.~47]{szyszkowski_o_1916}.%
}. Jest to niewątpliwie bardzo interesujący punkt w~poglądach Szyszkowskiego. Po pierwsze, pojawia się tu sugestia, iż \textit{matematyczne} pojęcie ciągłości, lub pojęcie którego stanowi ono konstytutywną część, jest pojęciem wrodzonym umysłowi ludzkiemu\footnote{Można próbować przyjąć tezę, że wrodzone pojęcie ciągłości jest u~Szyszkowskiego zasadniczo różne od matematycznego pojęcia ciągłości, które także nie stanowi jego zasadniczej części. Wówczas jednak trudno byłoby zrozumieć dalszy ciąg rozumowania uczonego.} . Po drugie, według Szyszkowskiego, o~relacjach przyczynowości można mówić tylko wówczas, kiedy uda się podać -- w~odniesieniu do przyrody -- opis matematyczny oparty o~pojęcie ciągłości. Po trzecie wreszcie, następuje tutaj powiązanie kategorii opisu matematycznego z~ontologicznym pojęciem przyczynowości. Co więcej, jak twierdzi Szyszkowski, ,,pojęcie ciągłości jest podstawą wszelkiego uogólnienia nie tylko w~matematyce i~fizyce, lecz i~we wszystkich innych dziedzinach życia umysłowego'' 
%\label{ref:RNDbD2r0KLmHZ}(Szyszkowski, 1916, s.~47).
\parencite[][s.~47]{szyszkowski_o_1916}. %
 Jakakolwiek dyscyplina wiedzy jest możliwa właśnie dlatego, że istnieją związki między przyczyną a~skutkiem, gwarantowane przez ciągłość. Między innymi dlatego właśnie istnieje potrzeba uzgodnienia najnowszych osiągnięć naukowych z~pojęciem ciągłości.


Kolejnym krokiem jest podjęcie przez Szyszkowskiego refleksji nad pojęciem energii. Zauważa on, że dla fizyka energia istnieje w~sposób tak samo realny, jak materia. Trudno jednak powiedzieć, że widzi on tutaj równoważność masy i~energii, wynikającą ze szczególnej teorii względności\footnote{Wypada zaznaczyć, że fakt równoważności masy i~energii jako konsekwencji szczególnej teorii względności pozostał prawie niedostrzeżony niemalże do lat trzydziestych XX wieku. Początkowo bardzo niewielu uczonych dostrzegło doniosłość tego faktu
%\label{ref:RNDBnorfhSdqE}(por. zwłaszcza Pais, 2005, s.~149–159).
\parencite[por. zwłaszcza][s.~149–159]{pais_subtle_2005}. %
 Co wydaje się jednak ważne, wśród chlubnych wyjątków można znaleźć Czesława Białobrzeskiego 
%\label{ref:RND2f7Rh2YnYC}(1911, s.~15–17).
\parencite*[][s.~15–17]{bialobrzeski_zasada_1911}.%
}. Niemniej, zarówno \textit{energię} jak i~\textit{masę} uważa Szyszkowski za fundamentalne dla fizyki. Podkreśla jednak, że przeciętny człowiek poznaje treść tych pojęć w~różny sposób, poznanie materii jawi się bowiem jako bardziej bezpośrednie i~zbliżone do potocznego, zmysłowego doświadczenia, które prowadzi do konstatacji, iż wszystko, co posiada masę, jest materialne. Zwraca też uwagę, że człowiek nie jest w~stanie poznawać energii przy pomocy samych zmysłów, w~każdym razie nie bezpośrednio. Wysnuwa stąd wniosek, że pojęcie energii ,,nie wypływa z~pierwotnego i~bezpośredniego postrzegania, lecz w~znacznej mierze musi być uznane za twór naszego umysłu'' 
%\label{ref:RNDpgF9HQ3Btv}(Szyszkowski, 1916, s.~48).
\parencite[][s.~48]{szyszkowski_o_1916}. %
 A~skoro tak, to o~ile nieciągłość materii przyjmie umysł za wynik potocznego doświadczenia, to odrzucenie ciągłości energii może być dla umysłu nie do przyjęcia. Stanowi to konsekwencję -- jak można się domyślać -- traktowania przez Szyszkowskiego pojęcia ciągłości jako pojęcia wrodzonego umysłowi ludzkiemu.


Po rozważaniach dotyczących ciągłości energii, autor wprowadza pojęcie eteru, będącego ośrodkiem, w~którym -- bez strat energii -- rozprzestrzenia się wszelkie promieniowanie elektromagnetyczne. Szyszkowski wymienia szereg cech eteru: jest on całkowicie nieściśliwy i~doskonale sztywny (co skutkuje brakiem przewodzenia fal podłużnych). W~tym miejscu warto wspomnieć o~bardzo interesującej procedurze myślowej, którą przedstawia Szyszkowski czytelnikowi, by ten mógł lepiej zrozumieć właściwości eteru. Proponuje on, aby wyobrazić sobie cząsteczki ściśnięte tak, by zlały się ze sobą i~nie było między nimi jakiejkolwiek pustej przestrzeni. Następnym krokiem jest wyobrażenie sobie (w umyśle), że materia ta traci masę, dematerializuje się i~zmienia w~substancję nieważką. Taki twór będzie posiadał wszystkie mechaniczne cechy eteru
%\label{ref:RNDzHekXMBFw0}(por. Szyszkowski, 1916, s.~50).
\parencite[por.][s.~50]{szyszkowski_o_1916}. %
 Zwraca tutaj uwagę abstrakcyjny zabieg \textit{odmaterializowania}, który ma zostać przeprowadzony w~umyśle. Ma to -- jak się wydaje -- daleko idące konsekwencje, gdyż od tego momentu eter przestaje być pojęciem tylko fizykalnym i~zaczyna funkcjonować w~nieco inny sposób, a~ostatecznie (za Lodgem) dzięki dość dużej ekwilibrystyce intelektualnej, zostanie uznany za ,,jakby pomost między duchem a~materią'' (jakkolwiek ten wątek zostanie jeszcze poruszony, jednak już w~tym miejscu warto uczynić o~tym małą wzmiankę 
%\label{ref:RNDq9Cc1exhDn}(por. Szyszkowski, 1916, s.~55)
\parencite[por.][s.~55]{szyszkowski_o_1916}%
). Tak rozumiany eter wypełnia wszechświat w~sposób całkowity (przenika także przestrzeń między atomami). Jest jednocześnie doskonale przenikliwy, tzn. nie stawia żadnego oporu ciałom w~nim się poruszającym, nie będąc także przez nie ciągniętym. Bardzo ważną cechą eteru jest to, że jest on \textit{substancją ciągłą}, bo to powoduje, zdaniem Szyszkowskiego, brak tarcia i~rozpraszania energii. Tarcie (i straty energii) wiąże on z~ziarnistą strukturą materii i~jej nieciągłością. Szyszkowski postuluje, iż oddziaływanie między materią a~eterem ma charakter \textit{elektromagnetyczny}\footnote{Bardzo interesujący jest opis propagacji promieniowania elektromagnetycznego: ,,Wahadłowy ruch naboju elektrycznego zostaje odczuty przez eter w~postaci fal elektromagnetycznych, które rozchodzą się w~jego łonie, nie słabnąc i~bez zamierania we wszystkich kierunkach. W~danym wypadku fale te są falami światła, a~kierunki w~których się one rozchodzą, są jego promieniami [...]'' 
%\label{ref:RNDqX2tF3Ve9V}(Szyszkowski, 1916, s.~51).
\parencite[][s.~51]{szyszkowski_o_1916}.%
}. Ważną cechą eteru jest także to, że oddaje on energię materii całkowicie, bez żadnych strat. Wynika to -- jak podkreśla Szyszkowski -- z~jego ciągłości. Innym argumentem na rzecz ciągłości eteru jest fakt możliwości propagacji w~nim fal o~dużej częstotliwości bez rozpraszania energii 
%\label{ref:RNDKwh8BNoKpg}(por. Szyszkowski, 1916, s.~52).
\parencite[por.][s.~52]{szyszkowski_o_1916}. %
 W~pewnym sensie trudno się dziwić, że Szyszkowski broni tak rozumianego eteru, gdyż przytoczone powyżej jego określenie doskonale współgra z~podstawowym postulatem traktowania pojęcia ciągłości jako pojęcia fundamentalnego w~opisie całości rzeczywistości. Za najdoskonalszy wyraz teorii eteru uważa Szyszkowski teorię elektromagnetyzmu, zaproponowaną przez Maxwella, gdyż, jak stwierdza: ,,Równania różniczkowe, za pomocą których teorja ta jest wyrażona, opierają się na założeniu, iż eter przewodzi energję w~sposób zupełnie ciągły; w~czystym eterze ciągłość ta jest absolutną'' 
%\label{ref:RNDCkZ8ww3hea}(Szyszkowski, 1916, s.~51).
\parencite[][s.~51]{szyszkowski_o_1916}. %
 Szyszkowski podkreśla, że teoria Maxwella, poprzez założenie istnienia eteru jako ośrodka ciągłego\footnote{Por. 
%\label{ref:RNDdL4YpQdvcU}(Maxwell, 1865)
\parencite[][]{maxwell_dynamical_1865} %
 oraz 
%\label{ref:RND0SuoeOLuQu}(Maxwell, 1878);
\parencite[][]{maxwell_ether_1878}; %
 szerzej na temat pojęcia eteru u~Maxwella: 
%\label{ref:RNDXVOioqn7PD}(Theocharis, 1983).
\parencite[][]{theocharis_maxwells_1983}.%
}, gwarantuje zachowanie pojęcia ciągłości jako pojęcia fundamentalnego, co skutkuje możliwością zastosowania w~niej równań różniczkowych (a to, jak można się domyślać, chroni od rozbicia kategorię przyczynowości).

Szyszkowski podaje także, jako jedno z~osiągnięć teorii Maxwella, tezę o~wywieraniu ciśnienia przez promieniowanie elektromagnetyczne. Zaznacza przy tym, że jest to fakt potwierdzony doświadczalnie, a~jego znaczenie w~zjawiskach kosmicznych wykazał Arrhenius
%\label{ref:RNDLCqcgIXt3V}(1903, s.~91–220).
\parencite*[][s.~91–220]{arrhenius_lehrbuch_1903}. %
 W~tym kontekście bardzo zastanawiające jest, dlaczego nie wspomina o~pracy Czesława Białobrzeskiego 
%\label{ref:RNDM6q9XQ53p5}(1913),
\parencite*[][]{bialobrzeski_sur_1913}, %
 dotyczącej roli ciśnienia promieniowania w~stanach równowagowych gwiazd. Jest to o~tyle interesujące, że obaj uczeni pracowali w~tym samym czasie, obracając się w~tym samym środowisku i~publikując w~tym samym czasopiśmie (\textit{Przegląd Naukowy i~Pedagogiczny})\footnote{Na temat ówczesnego środowiska kijowskiego por: 
%\label{ref:RND5E538LU6b2}(Róziewicz, Zasztowt, 1991)
\parencite[][]{roziewicz_polskie_1991} %
 oraz 
%\label{ref:RND4B9qPC5DT6}(Korzeniowski, 2009).
\parencite[][]{korzeniowski_za_2009}. %
 Brak wzmianki o~Białobrzeskim jest także intrygujący z~tego powodu, że w~pewnym sensie i~Szyszkowski i~Białobrzeski mieli spore zacięcie filozoficzne, co zdradzają obie ich publikacje w~\textit{Przeglądzie Naukowym i~Pedagogicznym}, por. 
%\label{ref:RNDkDCat8K9AQ}(Białobrzeski, 1916)
\parencite[][]{bialobrzeski_rzeczywistosc_1916} %
 oraz omawiany właśnie artykuł 
%\label{ref:RNDFaGEiFgony}(Szyszkowski, 1916).
\parencite[][]{szyszkowski_o_1916}. %
 Artykuł Białobrzeskiego także ma cechy swego rodzaju refleksji metanaukowej, wskazującej na silne związki nauki z~filozofią (Białobrzeski pisze wprost o~metafizyce). Nieobca jest także Białobrzeskiemu tęsknota za harmonią między umysłem a~poznawaną przyrodą, w~kształtowaniu której to harmonii widzi on doniosłą rolę dla filozofii. Porusza także problematykę związaną z~zastosowaniem matematyki do opisu przyrody. Artykuł ten jest mocnym świadectwem potwierdzającym fakt, iż filozoficzny namysł nad nauką oraz przyrodą pojawiły się u~Białobrzeskiego dość wcześnie. Temat ten stanowi jednak odrębny, niezwykle interesujący temat badawczy, i~jako taki zasługuje na odrębne opracowanie.}.

\subsection{Problemy związane z~eterem i~kwantowa kontrowersja}

Szyszkowski zauważa, że teoria Maxwella, przy wszystkich jej niezaprzeczalnych osiągnięciach, nie nadąża za szybkim rozwojem wiedzy i~nie jest w~stanie wyjaśnić wszystkich zjawisk obserwowanych w~coraz to nowszych doświadczeniach
%\label{ref:RNDhGNIJzLvAM}(por. Szyszkowski, 1916, s.~53).
\parencite[por.][s.~53]{szyszkowski_o_1916}. %
 Odwołując się do teorii kwantów zaproponowanej przez Plancka, Szyszkowski zauważa, że energia jest emitowana przez materię w~sposób nieciągły (przy pomocy -- jak to określa -- ,,atomów energii, czyli kwantów''), co stoi w~jawnej sprzeczności z~dotychczasowymi osiągnięciami fizyki. Podkreśla, że \textit{matematycznie} teoria kwantów doskonale opisuje i~wyjaśnia zaobserwowane zjawiska, niemniej -- jego zdaniem -- \textit{fizycznie} jest trudna do ubrania w~odpowiedni schemat pojęciowy, co współcześnie najbardziej odpowiadałoby doniosłemu problemowi fizycznej interpretacji struktur matematycznych. Stwierdza, że nie ma w~tym nic dziwnego, gdyż -- jak już wcześniej wspomniano -- zakłada on, że energia nie jest tworem czysto zmysłowym, lecz zawiera komponentę umysłową. Stąd też -- jak stwierdza -- posiada ona charakter ciągły, a~pojawiającą się nieciągłość emisji promieniowania tłumaczy on nieciągłością materii\footnote{Szyszkowski stwierdza także, że eter absorbuje energię promieniowania w~sposób ciągły. Niemniej, emisja energii z~materii musi następować w~sposób nieciągły, gdyż inaczej cała energia zostałaby skoncentrowana tylko i~wyłącznie w~eterze, czego jednak się nie obserwuje. Sam określa taki stan rzeczy -- tę swoistą dychotomię -- ,,wyrazem instynktu samozachowawczego''. Interesujące jest także, że Szyszkowski wspomina o~skwantowanej emisji promieniowania, stąd przywołanie nazwiska Plancka. Natomiast nic nie pisze o~tym, że nieciągły jest także proces absorpcji -- nazwisko Einsteina pada tylko w~kontekście szczególnej teorii względności, zaś problem i~wyjaśnienie efektu fotoelektrycznego (por. 
%\label{ref:RNDkIxetMKN1B}(Einstein, 1905)
\parencite[][]{einstein_uber_1905}%
) zostały całkowicie przez Szyszkowskiego pominięte.}.

Zdaniem Szyszkowskiego właśnie w~teorii promieniowania ,,po raz pierwszy w~fizyce odbija się kardynalne przeciwieństwo pomiędzy poznającym duchem i~poznawaną materią''
%\label{ref:RND3CmEmR2ev2}(Szyszkowski, 1916, s.~54).
\parencite[][s.~54]{szyszkowski_o_1916}. %
 W~świetle przeprowadzanych przez niego analiz (zaprezentowanych wyżej), wydaje się, że chodzi tutaj o~dychotomię między ciągłym światem ducha (lub umysłu), a~nieciągłym światem materii, który to świat jest przez ten umysł poznawany. Teoria Plancka jest -- w~tym kontekście -- widziana jako swego rodzaju formalny kompromis pomiędzy tymi dwiema rzeczywistościami, który jednak, jak się wydaje, nie do końca satysfakcjonuje Szyszkowskiego. Stawia on zatem pytanie, czy nie można wybrnąć z~tej sytuacji w~sposób radykalny? Jako próbę takiego rozwiązania problemu wskazuje na szczególną teorię względności Einsteina. Szyszkowski zwraca uwagę na silną podbudowę doświadczalną tej teorii, podkreślając także, że jednym z~podstawowych postulatów teorii Einsteina jest możliwość pomiaru (poznawania) ruchu względnego, to znaczy (jak to wyraża Szyszkowski) materii względem materii. Nie istnieje jednak możliwość poznania ruchu materii względem eteru. Szyszkowski zauważa też, że najbardziej znaczącą konsekwencją szczególnej teorii względności jest radykalna zmiana pojęcia czasu i~jednoczesności\footnote{Kolejnym interesującym faktem jest to, że nie wspomina on tutaj o~pojęciu czasoprzestrzeni, zaproponowanym przez Minkowskiego 
%\label{ref:RNDDKyJn8AIWe}(1909).
\parencite*[][]{minkowski_raum_1909}.%
}. W~kontekście pojęcia eteru wskazuje on, że konkluzją wypływającą z~teorii względności jest stwierdzenie, iż jeśli nie sposób doświadczalnie dowieść istnienia eteru, to można zrezygnować z~tego pojęcia lub założyć, że eter nie istnieje 
%\label{ref:RNDizw7lTnJM9}(por. Szyszkowski, 1916, s.~54–55).
\parencite[por.][s.~54–55]{szyszkowski_o_1916}. %
 Szyszkowski stwierdza jednakże, że z~faktu niemożliwości doświadczalnego wykrycia eteru (pośrednio lub bezpośrednio), nie wynika jeszcze, iż on nie istnieje\footnote{Brak możliwości eksperymentalnego wykrycia eteru Szyszkowski uzasadnia tym, że to, co jest poznawalne zmysłowo (doświadczalnie), dotyczy materii i~poznawać to można tylko jako naruszenie ciągłości. Eter zaś jest substancją ciągłą i~dlatego nie podlega poznaniu zmysłowemu (eksperymentalnemu) 
%\label{ref:RNDn8oolQ5DXx}(Szyszkowski, 1916, s.~55).
\parencite[][s.~55]{szyszkowski_o_1916}.%
 }. Zauważa także, że np. Planck odrzuca istnienie eteru, postulując rozchodzenie się fal elektromagnetycznych w~próżni. Szyszkowski czyni tutaj jednak bardzo ważne spostrzeżenie, iż jest to odrzucenie tylko pozorne, gdyż -- jego zdaniem -- Planck nie twierdzi, że fale elektromagnetyczne rozchodzą się w~\textit{przestrzeni geometrycznej}, gdyż nie posiada ona własności elektromagnetycznych (jako struktura matematyczna)\footnote{Pojawia się tutaj pewna trudność, gdyż z~tekstu Szyszkowskiego niestety nie wynika, do jakich dzieł Plancka się on odnosi. Wiadomo, że napisał recenzję 
%\label{ref:RNDUQ2joXcpWb}(Szyszkowski, 1911a)
\parencite[][]{szyszkowski_1911} %
 klasycznego dzieła Plancka \textit{Acht Vorlesungen über theoretische Physik} 
%\label{ref:RNDPRztxUK5G5}(Planck, 1910).
\parencite[][]{planck_acht_1910}. %
 Można zatem wysunąć hipotezę, że Szyszkowski niewątpliwie uwzględnił je w~swoich analizach, co nie wyklucza znajomości innych źródeł (ponadto jednak trudno jest stwierdzić cokolwiek więcej). Planck na podstawie analizy zasady względności wskazuje na konieczność odrzucenia teorii eteru stacjonarnego, podobnie czyni w~odniesieniu do eteru rozumianego jako nośnik promieniowania elektromagnetycznego i~na tym poprzestaje 
%\label{ref:RNDa3sNfLwSbt}(por. zwł. Planck, 1910, s.~116).
\parencite[por. zwł.][s.~116]{planck_acht_1910}. %
 Więcej na temat teorii eteru por. 
%\label{ref:RNDaBSyuu87B8}(Whittaker, 1953).
\parencite[][]{whittaker_history_1953}.%
}.
Szyszkowski zatem wprowadza tutaj bardzo istotne rozróżnienie pomiędzy obiektami fizycznymi i~strukturami matematycznymi, które nie posiadają cech fizycznych\footnote{Rozróżnienie to dzisiaj także nie dla wszystkich jest oczywiste. Jest ono także interesujące jeśli wziąć pod uwagę, w~jaki sposób Szyszkowski (o czym była mowa wyżej) traktuje pojęcie energii -- przeprowadzając w~pewnym sensie procedurę odwrotną, mianowicie prowadząc do ujmowania energii niemalże jako pewnego konstruktu umysłowego (czy też pojęcia teoretycznego).} . Dokonuje się tutaj, jego zdaniem, tylko pewna zmiana pojęć -- pojęciu próżni przypisuje się te same cechy fizyczne (omówione już wcześniej), które Szyszkowski przypisuje eterowi. Co więcej, tego rodzaju próżnia, jak twierdzi polski chemik, przewodzi energię w~sposób ciągły -- a~zatem zostaje zapewniona możliwość opisu tego przewodnictwa przy pomocy równań różniczkowych. A~to -- jak już także wcześniej zaznaczono -- prowadzi do zachowania kategorii przyczynowości i~tym samym zabezpiecza fizykę przed popadnięciem w~pojęciowy chaos. Jeśli przyjąć, że poszczególne kwanty energii są emitowane w~geometryczną przestrzeń, wówczas -- zdaniem Szyszkowskiego -- taka sytuacja nie daje się opisać przy pomocy równań różniczkowych. Przyjmuje on zatem istnienie eteru, widząc go jako pomost między duchem i~materią, choć zapewne lepiej byłoby nazwać go pomostem między poznającym umysłem a~poznawaną rzeczywistością przyrodniczą. Według Szyszkowskiego, w~ramach teorii eteru nieciągła materia jest zanurzona w~ośrodku ciągłym, co nadaje całości przyrody ciągłość, która jest charakterystyczną i~zasadniczą cechą poznającego umysłu. To zaś -- w~ogólnym rozrachunku -- umożliwia opis zjawisk przyrodniczych przy pomocy równań różniczkowych, które są -- jak twierdzi Szyszkowski -- ,,najwspanialszym wykwitem pojęcia ciągłości w~naszym umyśle''. Odrzucenie teorii eteru, jego zdaniem, pozbawiłoby badaczy takiej możliwości, co skutkowałoby koniecznością poszukiwań innych dróg opisu, znacznie trudniejszych lub zupełnie fałszywych 
%\label{ref:RNDmdoTcBEKLz}(por. Szyszkowski, 1916, s.~55)
\parencite[por.][s.~55]{szyszkowski_o_1916}%
\footnote{To ostatnie stwierdzenie może wydawać się szczególnie trafne, zwłaszcza w~obliczu prób przeformułowania opisu zjawisk fizycznych przy pomocy teorii kategorii -- a~zatem bez użycia równań różniczkowych.}.

Z~powyższej analizy artykułu Szyszkowskiego wyłania się zatem obraz filozofującego przyrodnika, który dąży do skonstruowania całościowego opisu rzeczywistości, w~ramach którego nastąpi -- w~jakiś sposób -- połączenie opisu świata materialnego i~świata duchowego oraz interakcji między nimi. Podstawową rolę w~tym obrazie odgrywa pojęcie ciągłości (jako cechy charakterystycznej dla umysłu ludzkiego), którego utrzymanie warunkuje możliwość opisu zjawisk przyrodniczych przy pomocy równań różniczkowych. To zaś gwarantuje możliwość utrzymania w~mocy kategorii przyczynowości, która, wedle Szyszkowskiego, jest kategorią fundamentalną dla jakiejkolwiek dyscypliny wiedzy. Aby zapobiec dychotomii między nieciągłą materią a~ciągłym światem umysłu konieczne jest utrzymanie pojęcia eteru, będącego pomostem między oboma światami. Eter też ma dość szczególne właściwości -- jest ośrodkiem ciągłym, odmaterializowanym, posiadającym jednak własności fizyczne (mechaniczne i~elektromagnetyczne). Tym właśnie różni się od przestrzeni geometrycznej, będącej tworem ściśle matematycznym. Uwydatnienie tej różnicy przez Szyszkowskiego należy uznać za przejaw szczególnego wyczucia i~dostrzegania przez niego różnicy między środkiem opisu (czyli konstrukcjami matematycznymi) a~opisywaną rzeczywistością (obiektami fizycznymi). Wypada jeszcze zwrócić uwagę na dwie sprawy. Po pierwsze, wydaje się, iż koncepcja pola, tak, jak jest ona rozumiana między innymi w~ramach ogólnej teorii względności, najwidoczniej nie jest Szyszkowskiemu znana\footnote{Przyczyny takiego stanu rzeczy można upatrywać, między innymi, w~trwających działaniach wojennych i~wynikającej stąd niemożliwości zapoznania się środowiska kijowskiego z~najnowszymi osiągnięciami z~zakresu fizyki w~Zachodniej Europie.}. Po drugie natomiast, wydaje się nieco karkołomnym zabiegiem dokonanie przez niego przeniesienia matematycznego pojęcia ciągłości na obszary nie-matematyczne.

\section{Spojrzenie ogarniające całość rzeczywistości}
Powyższy obraz będzie jednak niepełny, zwłaszcza w~odniesieniu do filozoficznych refleksji związanych z~fizyką, jeśli nie zostanie -- choćby w~kilku zdaniach -- uzupełniony o~spostrzeżenia, jakie można wysnuć na podstawie wspomnianych już notatek z~wykładów ,,Z zagadnień filozofii przyrody'' Szyszkowskiego
%\label{ref:RND1oyFI2gEOk}(por. [Noty z~wykładów na Wyższych Polskich Kursach Naukowych w~Kijowie], 1917, k. 211–244, 251–267).
\parencite[por. ][k. 211–244]{noauthor_noty_1917}. %
 Wykłady te nie były dotychczas przedmiotem analiz historyczno-filozoficznych i~z pewnością stanowiąc bardzo interesujący przedmiot badań, zasługują na osobną pracę. W~tym miejscu z~uwagi na cel niniejszego artykułu możemy poczynić przynajmniej kilka spostrzeżeń, opartych o~stwierdzenia ogólne (stanowiące założenia wykładów), zawarte we wstępnej części wspomnianych notatek\footnote{Część ta obejmuje karty 211r–217v i~ze względu na kontekst poruszany w~niniejszej pracy ta właśnie część notatek zwiera -- jak się wydaje -- najistotniejsze spostrzeżenia. Notatki obejmujące karty 218r-237v zawierają m.in. wprowadzenie do termodynamiki. Stanowią one niejednorodny tematycznie materiał, często odwołujący się do świata ducha. Karty 239v–243v zawierają kontynuację tej tematyki oraz wprowadzenie elementów kinematyki i~astronomii, natomiast karty 251r–267r zawierają wprowadzenie wiadomości z~zakresu chemii i~biologii.}. Istnieje oczywiście ryzyko, że wnioski wyciągnięte na podstawie tej lektury, będą raczej dotyczyły tego, co było w~wykładach istotne dla osoby sporządzającej notatki. Niemniej -- analizując je w~kontekście artykułu \textit{O~ciągłości} -- można przedstawić kilka uwag, które -- jak się zdaje -- nieco wzbogacają znajomość poglądów Szyszkowskiego jako filozofa przyrody, w~czasie jego pracy w~Kijowie.

Nietrudno zauważyć, że praca \textit{O~ciągłości} zasadniczo skupia się na fundamentalnych problemach pojęciowych, które -- według autora -- warunkują możliwość prowadzenia badań w~zakresie fizyki. Cel wykładów wprowadza już kontekst znacznie szerszy i~jest nim próba określenia relacji między materią, życiem i~duchem, przy czym podkreślony jest fakt, że prawa fizyki nie wystarczają do określenia praw sfery duchowej i~fenomenu życia
%\label{ref:RNDTo2nKVFPyn}([Noty z~wykładów na Wyższych Polskich Kursach Naukowych w~Kijowie], 1917, k. 211r).
\parencite*[][k. 211r]{noauthor_noty_1917}. %
 Stąd przedmiot wykładów leży na granicy przyrody i~filozofii (wydaje się jednak, że chodzi o~styk nauk przyrodniczych i~filozofii). Podkreślenia zatem wymaga fakt, że Szyszkowski nie widzi możliwości prostego zredukowania całości rzeczywistości tylko do poziomu opisywanego przez fizykę. Fundamentalnymi pojęciami, jeśli chodzi o~opis świata materialnego (na poziomie fizycznym), które zostają przedstawione słuchaczom, to materia, eter oraz energia. Dodatkowo, wprowadzone zostaje pojęcie środowiska jako pierwszej podstawy budowy świata. Trzeba zaznaczyć, że energię określa się tutaj jako przyczynę zjawisk. Z~tych dwóch pojęć: środowiska i~energii -- według notatek -- można wyprowadzić całość zjawisk, które zachodzą na świecie 
%\label{ref:RNDUZNFQWK7Av}([Noty z~wykładów na Wyższych Polskich Kursach Naukowych w~Kijowie], 1917, k. 211v).
\parencite*[][k. 211v]{noauthor_noty_1917}. %
 Za najbardziej fundamentalne prawo w~przyrodzie zostało uznane prawo zachowania energii, które jest przez Szyszkowskiego pojmowane jako prawo pochodzące z~wewnętrznej potrzeby ludzkiego umysłu. W~tym kontekście warto wspomnieć, że bardzo często kijowski wykładowca wprowadza daleko sięgające (może nawet zbyt daleko), lecz bardzo swobodne analogie między rzeczywistością przyrodniczą, a~pedagogiczną czy społeczną 
%\label{ref:RNDyUGLkTrmbI}(por. [Noty z~wykładów na Wyższych Polskich Kursach Naukowych w~Kijowie], 1917, k. 211v, 217 r).
\parencite[por. ][k. 211v]{noauthor_noty_1917}.%


Szyszkowski podkreśla konieczność stworzenia ogólnego, filozoficznego wyjaśnienia zjawisk przyrody, jako przykład podając propozycje romantycznego filozofa Józefa Hoene-Wrońskiego
%\label{ref:RNDqD0PP6LUvZ}([Noty z~wykładów na Wyższych Polskich Kursach Naukowych w~Kijowie], 1917, k. 212r)
\parencite*[][k. 212r]{noauthor_noty_1917}%
\footnote{Szyszkowski nie precyzował jednak, co dokładnie w~idealistycznych poglądach Hoene-Wrońskiego mogłoby stanowić filozoficznie satysfakcjonujący opis zjawisk przyrody. }. Godną podkreślenia jednak wydaje się tu uwaga, że zasada zachowania energii nie wystarcza do określenia kierunku ewolucji świata, a~więc -- zdaniem Szyszkowskiego -- ma charakter ograniczony. Kierunek taki istnieje, ponieważ gdyby było inaczej, panowałby chaos. Wydaje się zatem, że pada tu sugestia istnienia pewnej cechy świata, którą jest uporządkowanie. W~innym miejscu Szyszkowski sugeruje, że -- w~kontekście II zasady termodynamiki -- można zauważyć wśród bytów ożywionych w~pewnym sensie odwrotny kierunek ewolucji, niż w~przypadku bytów nieożywionych. Wśród tych pierwszych zaobserwować można procesy organizacyjne, w~przypadku bytów nieożywionych -- następuje dezorganizacja. To tłumaczyłoby, dlaczego do zadania wytłumaczenia relacji między przyrodą i~duchem, dodaje on konieczność wytłumaczenia także fenomenu życia oraz jego relacji ze wspomnianymi już: przyrodą (w aspekcie nieożywionym) i~duchem. W~przypadku ducha i~życia zauważa Szyszkowski istnienie pewnego pierwiastka organizacyjnego -- stąd stawiał wniosek, że życie jest czymś więcej, niż sumą zmian fizycznych i~chemicznych martwej materii 
%\label{ref:RNDeyqVCBmxc0}(Szyszkowski, 1917).
\parencite[][]{szyszkowski_entropja_1917}. %
 Duże znacznie, w~kontekście obserwowanego kierunku zjawisk, przypisuje Szyszkowski także zasadzie nieodwracalności, podkreślając jej statystyczny charakter 
%\label{ref:RNDcOGFH9XDun}(por. [Noty z~wykładów na Wyższych Polskich Kursach Naukowych w~Kijowie], 1917, k. 217r).
\parencite[por. ][k. 217r]{noauthor_noty_1917}. %
 Jest to godne uwagi i~świadczy o~staraniu wykładowcy o~prezentację najaktualniejszych osiągnięć nauki, które mogą pomóc w~ukształtowaniu ogólnego poglądu na rzeczywistość. Wyjątkową rolę w~filozoficznych przemyśleniach kijowskiego chemika pełni \textit{zasada entropii}\footnote{Oczywiście chodzi tutaj o~II zasadę termodynamiki. Szyszkowski poświęcił swoistej interpretacji tej zasady osobny artykuł: 
%\label{ref:RND6mghlZsFYp}(Szyszkowski, 1909).
\parencite[][]{szyszkowski_1909}. %
 W~artykule tym używa on zamiennie określenia: \textit{zasada entropii} oraz \textit{prawo entropii}.}. Warto poświęcić temu zagadnieniu kilka słów mimo, gdyż -- podobnie jak w~przypadku zagadnień związanych z~pojęciem ciągłości -- stanowi ono bardzo dobrą ilustrację sposobu filozoficznej interpretacji teorii fizycznych przez Szyszkowskiego. Otóż, w~charakterystycznym dla siebie stylu, widzi on jej zastosowanie -- jako pewnego rodzaju meta-zasady, odnoszącej się do różnych warstw rzeczywistości (nie tylko fizycznej czy biologicznej, lecz także, m.in. psychologicznej czy społecznej) -- do wszelkich zjawisk statystycznych, także (przykładowo) w~samoregulacji procesów społecznych, poprzez przeciwstawienie się koncentracji -- przykładowo -- władzy i~kapitału 
%\label{ref:RNDz6x6UwNBz8}(por. Szyszkowski, 1909, s.~32–33).
\parencite[por.][s.~32–33]{szyszkowski_1909}.%
 Podkreślając wagę tej zasady, Szyszkowski zauważa, że w~życiu (podobnie jak w~fizyce), \textit{prawo entropii}, stwarza słabszym możliwość walki o~przetrwanie, poprzez ograniczenie mocy silniejszych. Wyjątkowość tej zasady widzi on jednak w~tym, jeśli interpretuje się tę zasadę zbyt wąsko, traktując ją tylko jako swego rodzaju \textit{pasożyta energii}, \textit{złego geniusza natury}, wówczas może ona stanowić fundament najbardziej pesymistycznej filozofii. Jednakże -- jak zdaje się uważać Szyszkowski -- można zasadę tę traktować jako najbardziej humanitarne prawo nieożywionej natury, swego rodzaju troski aby najwyższe dobro było rozprowadzane tak szeroko i~równomiernie, jak to tylko możliwe\footnote{,,\textcyrillic{Законъ энтропіи, какъ въ физикѣ, такъ и въ жизни, ограничивая могущество сильнаго, даетъ возможность слабому бороться за существованіе. Самое замѣчательное въ немъ то, что въ слишкомъ узкомъ толкованіи онъ, какъ злой геній природы, какъ неотлучный спутникъ и паразитъ энергіи, могъ лечь въ основу самой пессимистической философіи, въ то время, какъ въ дѣйствительности это самый гуманный законъ неодушевленной природы, какъ бы данный ею въ доказательство ея нѣжнаго попеченія и заботливости о томъ, чтобы высшее благо было распредѣлено по возможности широко и равномѣрно.}'' 
%\label{ref:RNDjNkcqtXssR}(Szyszkowski, 1909, s.~33).
\parencite[][s.~33]{szyszkowski_1909}.}.

W~kontekście kijowskich wykładów pt. ,,Z zagadnień filozofji przyrody'' warto także zauważyć, że w~pewnym sensie Szyszkowski proponuje tutaj inną metodologię otrzymania rezultatów końcowych (czyli ogólnego, filozoficznego oglądu rzeczywistości), niż miało to miejsce w~przypadku artykułu \textit{O~ciągłości}. W~artykule tym, jako antidotum na kryzysową sytuację w~fizyce, Szyszkowski -- za Lodge',em -- proponował wyjście od znalezienia podstawowej zasady, która pozwoliłaby rozjaśnić problematyczną sytuację. W~ramach wykładów natomiast wyjściem z~problematycznej sytuacji -- czyli braku filozoficznego opisu rzeczywistości -- miałaby być metoda indukcyjna, a~więc wyjście od faktów i~wyprowadzanie wniosków na ich podstawie. Wydaje się, że ta rozbieżność jest tutaj bardzo interesująca. Niemniej, Szyszkowski podkreśla, że uogólnienie zjawisk można przeprowadzić na dwa sposoby: wspomnianą już metodą indukcyjną (przez doświadczalne poznawanie przyrody) oraz filozoficznie -- metodą dedukcyjną przez tworzenie zasad. Za idealną uważa on sytuację, w~której rezultaty obu rodzajów badań były takie same
%\label{ref:RNDjeZTow0cBL}(por. [Noty z~wykładów na Wyższych Polskich Kursach Naukowych w~Kijowie], 1917, k. 212r-212v)
\parencite[por. ][k. 212r-212v]{noauthor_noty_1917}%
 \footnote{Wśród uczonych i~myślicieli polskich, współczesnych Szyszkowskiemu, na których zresztą się on powołuje, wypada wymienić Mariana Smoluchowskiego, który także nie stroni od refleksji filozoficznej 
%\label{ref:RNDNW3qMDQz7Y}(por. np. Dziekan, 2017).
\parencite[por. np.][]{dziekan_zagadnienie_2017}. %
 Dodajmy, że w~ramach wykładów wspomniano o~Smoluchowskim w~kontekście ruchów Browna.}. Trzeba także podkreślić, że wybierając metodę indukcyjną, Szyszkowski nadal dba o~to, aby doniosłą rolę w~tłumaczeniu rzeczywistości odgrywały odpowiednio wytłumaczone i~uargumentowane (także na gruncie poszczególnych dyscyplin) zasady -- na przykład wymienione już zasada zachowania energii, czy nieodwracalność.

\section{Zakończenie}
Nie sposób oprzeć się wrażeniu, że Bohdan Szyszkowski jest myślicielem bardzo oryginalnym, ale też -- zwłaszcza jeśli chodzi o~kijowski okres jego refleksji filozoficznej -- poniekąd nieznanym. Zaskakująca może być próba połączenia w~jednym, filozoficznym opisie problematyki związanej z~materią, duchem i~życiem, jednakże bez próby zastosowania prymitywnego redukcjonizmu, jak sugeruje treść jego wykładów ,,Z zagadnień filozofji przyrody''. Artykuł \textit{O~ciągłości} ukazuje Szyszkowskiego jako myśliciela wyczulonego na subtelne niuanse, czego świadectwem jest wspomniane już odróżnienie struktur matematycznych od opisywanych przez nie obiektów fizycznych. Interesującą propozycją, wspartą interesującą argumentacją, jest także jego postulat istnienia eteru jako gwarancji utrzymania ciągłości jako pojęcia fundamentalnego -- co w~jego opinii -- pozwala zachować kategorię przyczynowości. Trzeba jednak w~tym miejscu podkreślić zasadniczą trudność w~odniesieniu do tezy polskiego chemika: czy intuicyjne pojęcie ciągłości, które stosuje on do umysłu, jest tym samym pojęciem, którym operuje się w~matematyce? Jak należałoby rozumieć ciągłość eteru, który nie jest ani bytem mentalnym, ani -- tym bardziej -- strukturą matematyczną? Są to pytania, na które Szyszkowski nie udziela odpowiedzi. Można także odnieść wrażenie, że ma on trafne intuicje, np. pojęcia eteru, stwierdzając, że potrzebne medium o~własnościach fizycznych, w~którym może mieć miejsce propagacja fenomenów fizycznych (pojęcie analogiczne do współczesnego pojęcia pola). Nie potrafi odpowiednio wyrazić tych intuicji wyrazić, być może właśnie poprzez brak odpowiedniej bazy pojęciowej. Niekiedy też intuicje te zdają się ginąć w~całokształcie myśli polskiego chemika, mającego ambicję stworzyć ogólny ogląd rzeczywistości. W~sposób szczególny dotyczy to treści wspomnianych wyżej wykładów. Charakterystyczną cechą spostrzeżeń Szyszkowskiego, utrudniającą ich analizę, jest także to, że często fundamentalne dla jego analiz stwierdzenia przytacza on bez jakichkolwiek argumentów czy choćby śladowych uzasadnień. Dotyczy to -- przykładowo -- tak istotnego stwierdzenia, iż ciągłość jest wrodzonym pojęciem umysłu. Szyszkowski nie podaje -- niestety -- jakiegokolwiek uzasadnienia takiej tezy, która wydaje się być istotna dla jego poglądów
%\label{ref:RND9xEAwphLhJ}(por. Szyszkowski, 1916, s.~47).
\parencite[por.][s.~47]{szyszkowski_o_1916}. %
Pozostaje pytaniem otwartym, na ile poglądy Szyszkowskiego są oryginalne, a~na ile inspirowane przez francuski konwencjonalizm, zwłaszcza zaś przez idee Henri
Poincarégo\footnote{Na temat powiązania równań różniczkowych (oraz ciągłością w~sensie matematycznym) z~determinizmem por. 
%\label{ref:RNDuHKnV5BcOv}(Poincaré, 1908a, s.~104–132, b, s.~98–135, 160–169).
\parencites[][s.~98–135]{poincare_nauka_1908}[][s.~98–135, 160–169]{poincare_wartosc_1908}. O~tej kwestii zob. także 
%\label{ref:RNDo02BSNCiJ4}(Leszczyński, Szlachcic, 2003, s.~80–81).
\parencite[][s.~80–81]{leszczynski_wprowadzenie_2003}.}.

Szyszkowski jawi się także jako myśliciel, który w~refleksji nad przyrodą stara się wziąć pod uwagę możliwie najnowsze osiągnięcia naukowe, co czyni jego twórczość filozoficzną bardzo atrakcyjną a~jednocześnie dobrze ugruntowaną. Pomimo, że jest on chemikiem, można w~jego rozważaniach dostrzec elementy charakterystyczne dla filozofii fizyki, co w~sposób szczególny uwidacznia się w~roli przypisywanej przez niego pojęciu eteru i~możliwości zastosowania analizy matematycznej w~fizyce. Uzasadnione będzie także -- przynajmniej w~odniesieniu do artykułu \textit{O~ciągłości} -- określenie części jego analiz jako pokrewnych filozofii w~nauce, gdy sytuacja problemowa w~dyscyplinach naukowych owocuje także refleksją filozoficzną. W~tym ostatnim przypadku Szyszkowski wpisywałby się w~bardzo interesujący nurt polskich uczonych i~myślicieli
%\label{ref:RNDPluK7vkOZF}(zob. np. Polak, 2019),
\parencite[zob. np.][]{polak_philosophy_2019}, %
 którego przedstawicielem można także nazwać innego polskiego uczonego również związanego z~Kijowem, wspomnianego już Czesława Białobrzeskiego 
%\label{ref:RNDvdstPVPIaW}(zob. np. Mścisławski, 2017).
\parencite[zob. np.][]{mscislawski_miedzy_2017}. %
 Ten ostatni, mimo iż jego twórczość filozoficzna (oraz naukowa) cieszy się sporym zainteresowaniem, pozostaje jednak postacią wciąż intrygującą. Autor niniejszej pracy uważa, iż byłoby wysoce wskazane, aby nadal prowadzić badania dotyczące Czesława Białobrzeskiego i~jego dorobku. Niniejsze opracowanie, stanowiące przyczynek do badań dotyczących recepcji starej teorii kwantów i~mechaniki kwantowej przed rokiem 1953\footnote{Data ta stanowi pewnego rodzaju symbol. 12 października 1953 roku zmarł Czesław Białobrzeski, którego filozoficzne refleksje dotyczące mechaniki kwantowej stanowią najbardziej wyczerpujące polskojęzyczne dzieło w~tym zakresie 
%\label{ref:RND7kEFxlfuUF}(Białobrzeski, 1984).
\parencite[][]{bialobrzeski_podstawy_1984}. %
 W~roku 1953 miała także miejsce publikacja pierwszego tomu serii Zagadnienia Filozoficzne Fizyki, poruszającego analogiczną tematykę, jednakże już w~zupełnie innym (ściśle materialistycznym) duchu 
%\label{ref:RND3pYs1Y84Cm}(Aleksandrow i~in., 1953).
\parencite[][]{aleksandrow_zagadnienia_1953}.%
}, wskazuje że w~tematyce wiele jest kwestii wartych zbadania w~kolejnych opracowaniach.

Niewątpliwie dalsze badania poświęcone twórczości Szyszkowskiego z~zakresu filozofii przyrody -- także z~uwzględnieniem prac powstałych już po roku 1920 w~Krakowie, pozwoliłyby bardziej przybliżyć tę nietuzinkową postać i~lepiej zrozumieć uprawianą w~tym mieście refleksję w~kontekście nauki. Wraz z~innymi uczonymi i~filozofami Szyszkowski stanowi przykład niezwykłego bogactwa polskiego życia intelektualnego i~myśli filozoficznej, zarówno jeszcze w~okresie zaborów, jak i~już po odzyskaniu przez Polskę niepodległości w~roku 1918.

\end{artplenv}