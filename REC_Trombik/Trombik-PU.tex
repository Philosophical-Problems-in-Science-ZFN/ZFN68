\begin{recplenv}{Kamil Trombik}
	{Wokół myśli Józefa Życińskiego}
	{Wokół myśli Józefa Życińskiego}
	{\textit{Media -- kultura -- dialog. W~piątą rocznicę śmierci arcybiskupa Józefa Życińskiego}, red. R. Nęcek, W. Misztal, Wydawnictwo Naukowe Uniwersytetu Papieskiego Jana Pawła II w~Krakowie, Kraków 2017, ss.~343.}







Józef Życiński należał do grona najbardziej rozpoznawalnych polskich filozofów chrześcijańskich przełomu tysiącleci. Dość powiedzieć, że lubelski arcybiskup pozostawił po sobie setki publikacji filozoficznych, w~tym dziesiątki książek, wydawanych także w~językach obcych, m.in. ,,The Structure of the Metascientific Revolution -- An Essay on the Growth of Modern Science'', z~1988 roku, która ukazała się w~polskim tłumaczeniu po 25 latach dzięki Wydawnictwu Copernicus Center Press
%\label{ref:RNDfGKHqx5bSK}(Życiński, 2013).
\parencite[][]{zycinski_struktura_2013}.%
 W~tym ogromnym dorobku naukowym i~popularyzatorskim znaleźć można prace poświęcone zarówno problematyce filozofii przyrody, metodologii nauk, filozofii języka, teologii naturalnej, jak i~filozofii społecznej czy etyce. Świadczy to niewątpliwie o~rozległości zainteresowań tego myśliciela, ukształtowanego i~związanego od lat 70. XX wieku z~krakowskim środowiskiem interdyscyplinarnym\footnote{Należy dodać, że J. Życiński był współtwórcą krakowskiego Ośrodka Badań Interdyscyplinarnych i~periodyku ,,Zagadnienia Filozoficzne w~Nauce'',,wniósł także istotny wkład w~rozwój koncepcji ,,filozofii w~nauce'', 
%\label{ref:RND9ek2REd7gQ}(zob. np. Polak, 2019).
\parencite[zob. np.][]{polak_philosophy_2019}.%
}.

Nie powinno więc dziwić, że spuścizna J. Życińskiego stała się w~ostatnim okresie przedmiotem licznych dyskusji w~środowisku akademickim, zwłaszcza w~Krakowie\footnote{Warto w~tym kontekście odnotować również najnowsze prace poświęcone filozofii J. Życińskiego, jakie powstają w~środowisku krakowskim -- m.in. artykuł Z. Liany ,,Nauka jako racjonalna doxa. Józefa Życińskiego koncepcja nauki i~filozofii nauki -- poza internalizmem i~eksternalizmem'',
%\label{ref:RNDEdtOWWuDHP}(Liana, 2019).
\parencite[][]{liana_nauka_2019}.%
}. 11 lutego 2016 roku odbyła się tutaj ogólnopolska konferencja naukowa ,,Media -- kultura -- dialog'',,poświęcona właśnie pamięci zmarłego pięć lat wcześniej metropolity lubelskiego. W~organizację tego wydarzenia zaangażowało się wiele instytucji naukowych oraz organów samorządowych: Uniwersytet Jagielloński w~Krakowie, Uniwersytet Papieski Jana Pawła II w~Krakowie, Akademia Ignatianum, Naczelna Rada Adwokacka w~Warszawie oraz Okręgowa Rada Adwokacka w~Krakowie. Owocem tej konferencji jest m.in. omawiana monografia zbiorowa, zawierająca blisko 20 artykułów poświęconych szeroko rozumianej działalności społeczno-kulturowej J. Życińskiego, a~także okolicznościowe teksty wystąpień organizatorów i~hierarchów kościelnych, biorących udział w~konferencji\footnote{Zasadniczą część książki poprzedzają teksty wystąpień Wojciecha Nowaka (Rektora Uniwersytetu Jagiellońskiego), Gianfranco Ravasi (Przewodniczącego Papieskiej Rady do spraw Kultury), Celestino Migliore (ówczesnego Nuncjusza Apostolskiego w~Polsce), Stanisława Dziwisza (ówczesnego Metropolitę Krakowskiego), Wojciecha Zyzaka (Rektora Uniwersytetu Papieskiego Jana Pawła II). Monografię zamyka natomiast przemówienie Wojciecha Życińskiego do organizatorów i~uczestników konferencji, a~także tekst homilii bp. Grzegorza Rysia.}.

\enlargethispage{-.5\baselineskip}
Rozległość zainteresowań badawczych lubelskiego metropolity odzwierciedla rozpiętość tematyczna tekstów zawartych w~monografii. Mamy tutaj artykuły reprezentantów filozofii, wywodzących się z~różnych nurtów myślenia (m.in. Michał Heller, Jan Woleński, Jan Andrzej Kłoczowski, Alfred Wierzbicki), a~także teologii (m.in. Józef Kloch, Andrzej Draguła), literaturoznawstwa (Franciszek Ziejka), językoznawstwa (Jerzy Bartmiński), nauk społecznych (m.in. Michał Drożdż, Robert Nęcek, Wiesław Godzic), prawa (m.in. Andrzej Zoll) i~medycyny (Ewa Kucharska, Tomasz Trojanowski).

Choć omawianą książkę trudno uznać za wyczerpujące studium myśli Życińskiego, znalazło się tutaj miejsce na omówienie przynajmniej kilku istotnych aspektów jego akademickiej i~duszpasterskiej spuścizny. Jak pisał M. Heller, który przez wiele lat współpracował naukowo z~lubelskim arcybiskupem, ,,Józef Życiński jako człowiek miał wiele twarzy: twarz intelektualisty, działacza, duszpasterza, hierarchy, filozofa i~uczonego, pisarza, felietonisty o~ciętym języku, twarz wrażliwego człowieka'', (s.~25). Autorzy ukazują na kartach tej książki niektóre z~twarzy Życińskiego, przybliżając czytelnikowi złożony świat jego oryginalnej myśli filozoficznej, teologicznej i~społecznej.

Część artykułów wprost koncentruje się na problematyce filozoficznej, a~ich autorzy omawiają m.in. koncepcję pola racjonalności czy polemiki Życińskiego z~postmodernizmem\footnote{Zob. np. rozdziały M. Hellera ,,Jozefa Życińskiego idea pola racjonalności'',
%\label{ref:RNDPocXZSDkXd}(Heller, 2017)
\parencite[][]{heller_jozefa_2017}%
 i~M. Drożdża ,,Poszukiwanie sensu wobec postmodernizmu, relatywizmu i~ironii'', 
%\label{ref:RND0E7cy4NqCz}(Drożdż, 2017).
\parencite[][]{drozdz_poszukiwanie_2017}.%
}. Choć zagadnienia z~zakresu filozofii przyrody i~metodologii nauk -- czyli wiodącego obszaru zainteresowań Życińskiego -- rzadko stanowią centrum rozważań autorów tej książki, to w~licznych tekstach powraca wątek koncepcji interdyscyplinarności jako fundamentu filozofii krakowskiego myśliciela (np. s.~125–127). Poszukiwanie pomostów między naukami przyrodniczymi, humanistyką a~myślą chrześcijańską było jednym z~kluczowych elementów działalności Życińskiego, co autorzy tej książki podkreślają w~swoich artykułach dotyczących jego refleksji nad człowiekiem (zwłaszcza w~kontekście społecznym), kulturą, aksjologią, religią i~rolą Kościoła we współczesnym świecie.

Należy przy tym podkreślić, że teksty opublikowane na kartach tej książki stanowią zaledwie przyczynek do dalszych, bardziej pogłębionych badań nad spuścizną Życińskiego. Jak na pracę zbiorową, książka spełnia swoją rolę i~dostarcza wielu cennych informacji na temat różnych aspektów naukowej i~duszpasterskiej działalności krakowskiego filozofa. Książka pokazuje także drogi możliwego rozwinięcia tych idei, które kształtowały przez lata myślenie Życińskiego i~stanowiły o~specyfice jego poglądów filozoficznych. Istnieje jednak pokusa, aby marginalizować tego typu publikacje z~powodu ich przeglądowego i~przyczynkowego charakteru. W~przypadku tej pozycji można jednak wskazać przynajmniej dwa powody, które czynią tę publikację wartościową, szczególnie dla filozofów.

Po pierwsze, książka stanowi zbiór unikalnych informacji na temat wybranych etapów z~życia krakowskiego filozofa. Niektórzy autorzy dzielą się z~czytelnikiem cennymi wspomnieniami biograficznymi o~Życińskim\footnote{Mam tutaj na myśli przede wszystkim teksty J. Woleńskiego ,,Józef Życiński jako filozof i~człowiek'', oraz S. Kłysa ,,Niezłomny w~stanie wojennym'',
%\label{ref:RNDGUVmCNxoZe}(Woleński, 2017; Kłys, 2017).
\parencites[][]{wolenski_jozef_2017}[][]{klys_niezlomny_2017}.%
}, które przede wszystkim rzucają światło na działalność tego myśliciela w~trudnych realiach politycznych lat 80. XX wieku. Osobiste relacje ludzi z~otoczenia Życińskiego można potraktować przede wszystkim jako wartościowe źródło dla historyków filozofii, zainteresowanych próbą odtworzenia losów krakowskiego filozofa, a~także podejmujących badania dotyczące powiązań między jego życiem a~akademicką działalnością. Potrzeba takich badań jest o~tyle istotna, że jak dotąd nie ukazała się jeszcze monografia poświęcona Życińskiemu, która stanowiłaby kompendium wiedzy na temat jego życia i~naukowej twórczości.

Po drugie, książka jest świadectwem promieniowania twórczości Życińskiego w~środowisku polskich intelektualistów. Autorzy artykułów składających się na książkę nie tylko dzielą się osobistymi wspomnieniami o~Życińskim, nie tylko omawiają jego poglądy i~sytuują je w~kontekście różnych tradycji intelektualnych, ale również próbują zastosować je w~nowych kontekstach, wskazując przy tym na perspektywy badawcze i~możliwe kierunki rozwoju tych idei, które kształtowały myśl krakowskiego filozofa (np. s.~197–198). Jeśli przyjąć, że poglądy Życińskiego mają nie tylko wartość historyczną, lecz mogą okazać się pomocne w~kontekście wyzwań płynących ze strony współczesności -- na przykład w~kontekście aktualnego wciąż pytania o~matematyczność przyrody -- to książkę należy potraktować jako cenne źródło inspiracji do dalszych badań w~zakresie tak różnych dyscyplin i~obszarów badawczych, jak filozofia przyrody, metodologia nauk, medioznawstwo, filozofia społeczna, kulturoznawstwo, etyka, teologia (zwłaszcza eklezjologia i~teologia pastoralna). Jeśli chodzi o~przedstawicieli filozofii, to w~tym miejscu należałoby przypomnieć słowa M. Hellera, który na kartach książki stwierdził wprost, że filozoficzna spuścizna Życińskiego ,,to dzieło niedokończone. Warto o~nim nie tylko pamiętać, ale je także twórczo rozwijać'', (s.~25).

Dorobek intelektualny Życińskiego stanowi ciekawy przykład próby modernizacji filozofii chrześcijańskiej w~Polsce. Jest to również próba przekroczenia skrajnie systemowej filozofii katolickiej, jaką rozwijano w~XX wieku głównie w~kręgach polskich neotomistów. Krytycyzm i~antydogmatyzm szedł u~Życińskiego w~parze z~przekonaniem, iż możliwe jest wypracowanie wizji świata opartej na dorobku nauki, a~równocześnie spójnej z~humanistycznym i~teologicznym spojrzeniem na rzeczywistość, choć niekoniecznie w~świetle zasad wypracowanych na gruncie filozofii arystotelesowskiej. Dorobek współczesnych filozofów związanych z~Centrum Kopernika Badań Interdyscyplinarnych, Uniwersytetem Papieskim Jana Pawła II w~Krakowie i~zaprzyjaźnionymi instytucjami pokazuje, że idea ta padła na podatny grunt i~jest twórczo rozwijana przez kolejne pokolenia filozofów w~całej Polsce.



\autorrec{Kamil Trombik}


\subsubsection{Bibliografia}\nopagebreak[4]
\end{recplenv}