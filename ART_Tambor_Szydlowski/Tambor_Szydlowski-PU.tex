\begin{artplenv2auth}{Marek Szydłowski, Paweł Tambor}
	{Czy fizyczne teorie efektywne są wiarygodną strategią osiągnięcia teorii ostatecznej?}
	{Czy fizyczne teorie efektywne są wiarygodną strategią\ldots}
	{Czy fizyczne teorie efektywne są wiarygodną strategią osiągnięcia teorii ostatecznej?}
	{Wydział Filozofii KUL, Uniwersytet Jagielloński Kraków}
	{Are physical effective theories the reliable strategy for achieving certain knowledge?}
	{The paper presents the methodology of effective theories as a~strategy used in the process of development of modern physics to reach a~final theory. We present the definition and characteristic features of an effective theory, as well as the answer to the question of whether and what kind of scenario of reaching a~final theory is realized by contemporary physics. We argue that the process of development of physics in the direction of a~final theory is potentially final, i.e. expressible in the conceptual schema of effective theories and as such it is convergent to a~final theory. In each effective theory there are physical constants, however, whose status differs from logical constants. They have a~dimension (length, energy, etc.) and are used to compare physical quantities. The structure of relevant effective theory can be interpreted in the epistemological framework of approximated truth theory. In the case study of cosmological models the sequence of models is convergent to potentially true model. The Standard Cosmological Model is the theory of the structure and dynamics of the Universe.}
	{philosophy and methodology of science, effective theory, final theory, cosmology, approximated truth theories.}
	{%
			{\flushright\subbold{Marek Szydłowski}\\\subsubsectit\small{Uniwersytet Jagielloński}\par}%
			{\flushright\subbold{Paweł Tambor}\\\subsubsectit\small{Katolicki Uniwersytet Lubelski}\par}%
		}


\section{Wstęp}
\lettrine[loversize=0.13,lines=2,lraise=-0.04,nindent=0em,findent=0.2pt]%
{N}{}iewątpliwe sukcesy nauk przyrodniczych mierzone w~kategoriach wyjaśniania posiadanych danych empirycznych i~przewidywania nowych, mimo braku jednej teorii fundamentalnej, zasługują na refleksję w~ramach metodologii i~filozofii nauki. Naszym zainteresowaniem obejmiemy szczególną kategorię metodologiczną, która wyraźnie wyodrębnia się w~praktyce badawczej nauk empirycznych, mianowicie tzw. teorię efektywną (TE). Jest ona o~tyle specyficzna, że świetnie nadaje się do rekonstrukcji praktyk badawczych, dziedzicząc pewne cechy ogólnych teorii oraz modeli.

W~niniejszej pracy twierdzimy, że ta tradycyjna nomenklatura metodologiczna w~kontekście problemu \textit{teoria–model} często jest niewystarczająca. Zwróćmy uwagę choćby na Standardowy Model Cząstek Elementarnych (SMCE) -- przykład teorii efektywnej jako jednej z~teorii uznawanych za najbardziej fundamentalne
%\label{ref:RNDK9teGqhNU1}(Kaplan, 2000).
\parencite[][]{kaplan_effective_1999}. %
 Rok 1970, kiedy Model Standardowy ukonstytuował się jako unifikacja teorii oddziaływań elektrosłabych i~chromodynamiki kwantowej, można było uznać za zasadniczy sukces w~poszukiwaniu teorii fundamentalnej, która będzie miała prawdopodobnie postać renormalizowalnej kwantowej teorii pola 
%\label{ref:RNDuHbmjnVaF6}(Weinberg, 1987).
\parencite[][]{weinberg_newtonianism_1987}. %
 Po pierwszym entuzjazmie, związanym z~sukcesami SMCE, nastąpiła w~ostatnich latach zmiana spojrzenia -- zaczyna on być traktowany jako tzw. efektywna teoria pola, a~zatem pewne niskoenergetyczne ograniczenie teorii bardziej podstawowej, która może nawet nie być teorią pola 
%\label{ref:RNDWpH83MgWQP}(Hartmann, 2001; por. Weinberg, 1997).
\parencites[][]{hartmann_effective_2001}[por.][]{weinberg_what_1997}.%


Zatem w~zupełnie nowym świetle powraca pytanie o~status SMCE w~kontekście problemu fundamentalności w~fizyce. Zarówno fizycy, jak i~filozofowie nauki twierdzą, że możliwe są dwie odpowiedzi: a) istnieje teoria fundamentalna i~sensownym jest jej poszukiwanie (Weinberg); b) teoria fundamentalna jako taka nie istnieje; mamy natomiast do czynienia z~pewną zhierarchizowaną strukturą teorii/modeli, które możemy nazwać efektywnymi, tworzących pewien ciąg lub hierarchiczną strukturę; stanowią one dla siebie nawzajem pewne teoretyczne uwarunkowanie i~zależą od siebie w~specyficzny sposób
%\label{ref:RNDV3M4DRy7Dj}(Cao, Schweber, 1993).
\parencite[][]{cao_conceptual_1993}.%


Lee Smolin w~swojej książce \textit{Kłopoty z~fizyką. Powstanie i~rozkwit teorii strun, upadek nauki i~co dalej}
\parencite*[][]{smolin_klopoty_2008}
wyróżnił pięć głównych problemów fizyki współczesnej: 1) problem konstrukcji teorii kwantowej grawitacji; 2) problem z~fundamentalnością mechaniki kwantowej; 3) problem teoretycznej unifikacji oddziaływań i~cząstek (pytanie o~możliwość zunifikowanej ontologii); 4) problem \textit{fine-tuning} (wyjaśnienie wartości wolnych stałych w~modelu standardowym cząstek); 5) wyjaśnienie natury ciemnej energii i~ciemnej materii.
%\label{ref:RNDjLWFJwwm6Y}(Smolin, 2008).
 John Barrow upatruje potencjalny cel nauki -- teorię wszystkiego -- w~kategoriach epistemologicznych: teoria ta będzie dostarczała kompletnego zrozumienia w~odniesieniu do kilku składników: praw natury, warunków początkowych, własności sił i~cząstek elementarnych, stałych fundamentalnych, symetrii, zasad podstawowych czy kategorii myślowych, którymi się posługujemy 
%\label{ref:RNDDcfzJinUPq}(Barrow, 2007, s.~4).
\parencite[][s.~4]{barrow_new_2007}. %
 Właściwie próby rozwiązania większości z~tych problemów można podjąć posługując się kategoriami: teorii efektywnych, ostatecznych i~fundamentalnych.

Smolin w~książce \textit{Czas Odrodzony, od kryzysu w~fizyce do przyszłości wszechświata}
%\label{ref:RNDB2DTCtccOw}(2015),
\parencite*[][]{smolin_czas_2015} %
 formułuje kilka ciekawych uwag odnośnie teorii efektywnych wyrażanych w~kontekście Modelu Standardowego Cząstek. Twierdzi, że pojęcie teorii efektywnej wskazuje na dojrzałość w~podejściu do teorii cząstek elementarnych. Wspomina, że kiedyś fizyce ,,marzyło się'' odkrycie wszystkich fundamentalnych praw. Dopiero później uświadomiono sobie, że ten cel jest zbyt ambitny, ponieważ stosowalność Modelu Standardowego ogranicza się tylko do pewnej domeny. Stąd pojawiła się pewna frustracja wynikająca z~niespełnienia ambitnych oczekiwań w~stosunku do fizyki fundamentalnej, której celem jest odkrywanie nowych praw przyrody. Teoria efektywna ze względu na swój charakter tego nie czyni. Wydaje się to niemożliwe, by teoria efektywna jednocześnie była zgodna ze wszystkimi dotąd przeprowadzonymi eksperymentami i~stanowiła co najwyżej przybliżenie prawdy 
%\label{ref:RNDGALBxLRxnB}(Smolin, 2015, s.~165)
\parencite[][s.~165]{smolin_czas_2015}%
\footnote{W~zasadzie teoria efektywna nie prowadzi bezpośrednio do okrywania praw przyrody, natomiast może pełnić rolę narzędzia heurystycznego w~sensie pośrednim w~procesie odkrycia prawa.}.

W~oparciu o~tę bazę problemową oraz o~wybrane wyniki badań z~zakresu kosmologii współczesnej stawiamy sobie w~pracy cele przede wszystkim metodologiczne. Po pierwsze, rozróżnienie i~zdefiniowanie dwóch pojęć: teorii efektywnej oraz teorii ostatecznej oraz zaproponowanie relacji wiążących te pojęcia ze szczególnym wyróżnieniem możliwych strategii dochodzenia do teorii ostatecznej. Chcemy przy tym odnosić się w~analizie metodologicznej do faktycznej praktyki i~nomenklatury badawczej używanej w~fizyce. Nie tworzymy więc nowej metodologii, ale porządkujemy zastane kategorie w~taki sposób, że relacje między nimi stają się coraz bardziej jasne. Drugim celem pracy jest metodologiczne uogólnienie pojęcia ,,teoria efektywna'', usytuowanie go w~kontekście debaty nad fundamentalnością w~fizyce oraz pokazanie zasadniczego wkładu o~znaczeniu epistemologicznym, który teorie efektywne wnoszą we współczesny dyskurs nad pewnością w~nauce
%\label{ref:RND4auZZSrkNN}(Szydłowski, Tambor, 2008).
\parencite[][]{szydlowski_model_2008}. %
 Zwróćmy uwagę, że zasygnalizowane problemy dotyczące fundamentalności w~fizyce możemy wyrazić bardziej precyzyjnie w~filozoficznym schemacie pojęciowym w~następujący sposób: 1) jeśli istnieje tzw. teoria wszystkiego, to jaką będzie miała postać?; w~jakim sensie inne teorie będą redukowalne (jeśli będą) do teorii najbardziej fundamentalnej; jaki kontekst filozoficzny będzie najbardziej adekwatny do ich opisu: zdaniowy czy niezdaniowy schemat pojęciowy?; 2) jeśli natomiast brak teorii ostatecznej (jest epistemologicznie i~metodologicznie definitywnie nieosiągalna), a~jesteśmy niejako skazani (przynajmniej operacyjnie) na posługiwanie się teoriami efektywnymi, to pojawiają się zasadniczo dwa problemy: fundamentalności względnej (Szynkiewicz pisze na przykład o~pewnej klasie teorii ostatecznych -- cząstkowych) lub lokalnej (hierarchiczna struktura teoretyczna postuluje wielopoziomową \textit{ontologię}) oraz kwestia natury relacji między teoriami efektywnymi.

Niniejsza praca ma następującą strukturę: w~pierwszej części dokonujemy prezentacji podstawowych pojęć i~kontekstu ich stosowania. W~drugiej, zasadniczej części, definiujemy i~charakteryzujemy pojęcie ,,teorii efektywnej'' i~jej statusu metodologicznego w~kontekście tradycyjnego podziału na teorie i~modele w~nauce. W~części trzeciej wskazujemy na metodologiczne i~filozoficzne znaczenie teorii efektywnych, zwłaszcza przez ukazanie współczesnego oblicza debaty nad redukcjonizmem w~filozofii nauki. Część ostatnia ma charakter metateoretyczny, stanowi próbę pokazania, jak efektywne modelowanie w~nauce wpływa na modelowanie w~ogólnie pojętej metanauce.

\section{Nomenklatura: teorie efektywne, fundamentalne, ostateczne i~relacje między nimi}
Teoria efektywna posiada przede wszystkim następujące cechy\footnote{Prototypem teorii efektywnej na gruncie fizyki jest efektywna teoria pola
%\label{ref:RNDdsViEs8BX9}(Kim, 1998; Wilson, 1971).
\parencites[][]{kim_wilson_1998}[][]{wilson_renormalization_1971}. %
 W~niniejszej pracy podejmujemy próbę adaptacji pojęcia ,,teoria efektywna'' w~kontekście metodologii i~filozofii nauki.}:
 a)~wyraźnie określony i~ograniczony (najczęściej w~skali energetycznej) zakres stosowania; b) posługuje się parametrami, których natury sama teoria efektywna nie wyjaśnia, co nie przeszkadza w~skutecznym posługiwaniu się teorią; c) posługuje się pojęciami, które mają w~jej ramach status użytecznych fikcji (ze względu na zastosowane idealizacje); d) nie da się do niej stosować predykatu prawdziwości, ze względu na zastosowane idealizacje\footnote{Cechy (c) i~(d) są specyficzne dla analizy Uzana i~nie są powszechnie przyjmowane w~literaturze.}. Jean-Philippe Uzan potwierdza taką intuicję dotyczącą teorii efektywnych; a~zwłaszcza to, że standardowy model kosmologiczny jest taką~właśnie teorią. Model ten z~punktu widzenia teorii fizyki opiera się~na ogólnej teorii względności, elektromagnetyzmie i~fizyce jądrowej. Dość~dobrze da się~wyodrębnić jego ograniczenia w~skali energii. A~zatem, stwierdza Uzan, żeby dobrze rozumieć w~przyjętej skali nasz obserwowalny Wszechświat, nie musimy konstruować kwantowej teorii grawitacji 
%\label{ref:RNDtodLlEAyWN}(Uzan, 2017, s.~109).
\parencite[][s.~109]{uzan_emergent_2017}.%


W~rozumieniu Poppera teoria ostateczna to taka, która nie potrzebuje już wyjaśnienia żadnego ze swoich składników: ,,eksplikans nie może być już dalej wyjaśniany ani nie potrzebuje dalszego wyjaśniania''
%\label{ref:RNDHsxbgZkiZj}(Popper, 2002, s.~159).
\parencite[][s.~159]{popper_wiedza_2002}. %
 Wiadomo, że zarówno Popper, a~nawet Weinberg po okresie pewnej fascynacji ideą teorii wszystkiego, jak i~wielu innych filozofów i~uczonych wypowiadali się sceptycznie w~stosunku do możliwości skonstruowania takiej teorii. Jeśli zatem próbujemy definiować teorię ostateczną w~kategoriach wyjaśniania, to dość łatwo uchwycić różnicę między teorią efektywną i~ostateczną. Ta pierwsza, choć skuteczna w~danym obszarze fizyki, posługuje się parametrami efektywnymi, których nie musi wyjaśniać, by spełniać swoje zadanie w~odniesieniu do klasy zjawisk. Można jednak próbować konstruować ciąg teorii efektywnych w~oparciu o~funkcję wyjaśniania: każda następna byłaby coraz bliższa teorii ostatecznej.

Szynkiewicz, odnosząc się~do Poppera, określa teorię~ostateczną~jako: ,,[...] propozycję~stanowiącą końcowy, pełny i~niemodyfikowalny fragment wiedzy teoretycznej. Koncepcje tego typu, z~uwagi na ich zakres, dzielone są zazwyczaj na dwie zasadnicze klasy: 1) teorie wszystkiego -- najbardziej ambitne i~odnoszące się do wszystkich klas zjawisk, 2) teorie ostateczne (cząstkowe) -- opisujące wybrane, ograniczone obszary przedmiotowe (np. zjawiska fizyczne, biologiczne). [...] Teoria ostateczna byłaby więc w~proponowanym ujęciu koncepcją~posiadającą wewnętrznie niesprzeczną~i niemodyfikowalną aksjomatykę''
%\label{ref:RND4RBAmt5LFF}(Szynkiewicz, 2011, s.~1274–1275).
\parencite[][s.~1274–1275]{szynkiewicz_teorie_2011}. %
 Autor dookreśla definicję poprzez podanie istotnych cech teorii ostatecznej: zupełność (teoria odnosi się~do wszystkich zjawisk w~opisywanej przez siebie domenie), prostota logiczna (wypadkowa informacyjnej zawartości teorii i~ilości postulatów wyjściowych), stabilność i~niezmienność~strukturalna teorii.

Szynkiewicz, powołując się na Smolina i~Kaku, podaje taką wstępną charakterystykę teorii ostatecznej: ,,[jest to] taka koncepcja, która unifikowałaby w~ramach jednego opisu teoretycznego całość wiedzy na temat oddziaływań podstawowych i~materialnej struktury rzeczywistości''
%\label{ref:RNDIm7d4itGJV}(Szynkiewicz, 2009, s.~16).
\parencite[][s.~16]{szynkiewicz_teorie_2009}. %
 W~ramach podgrupy teorii ostatecznych, które przy pewnych założeniach można nazwać także teoriami wszystkiego, wyróżnia teorie homogeniczne i~heterogeniczne. Pierwsze, nazywane teoriami wszystkiego poziomu pierwszego, są to teorie o~maksymalnym zasięgu ontologicznym, tzn. podają mechanizm -- zasadę, która leży u~podstaw wyjaśnienia całej rzeczywistości, a~zatem nie tylko fizycznej, ale na przykład umysłowej, psychicznej (,,świadome i~planowe próby opisu całej rzeczywistości'' 
%\label{ref:RNDRxjEGYcr8v}(Szynkiewicz, 2009, s.~30)
\parencite[][s.~30]{szynkiewicz_teorie_2009}%
). Teorie heterogeniczne (tzw. teorie wszystkiego drugiego poziomu) dopuszczają istnienie wielu różnych poziomów rzeczywistości lub ich opisu (nauki szczegółowe, społeczne, humanistyczne), które są niewspółmierne terminologicznie, natomiast teoria homogeniczna jest świadectwem (efektem) możliwości istnienia i~rekonstrukcji redukcji między tymi poziomami.

Teoria fundamentalna niekoniecznie jest teorią ostateczną, co więcej, może w~pewnym sensie być także teorią efektywną (na przykład mieć możliwy do wyznaczenia zakres stosowania). Z~jednej strony istnieje bowiem wśród uczonych przekonanie, że wszystkie teorie fizyczne są efektywne, tzn. posiadają wyróżnione przez nas powyżej własności. Z~drugiej strony używa się przecież sformułowania ,,teorie fundamentalne'' w~odniesieniu do chromodynamiki kwantowej czy ogólnej teorii względności. Sokołowski używa w~odniesieniu do tych teorii predykatu: ,,teoria niemal fundamentalna''
%\label{ref:RNDn8L5blSDrX}(Sokołowski, 2006, s.~122).
\parencite[][s.~122]{sokolowski_teorie_2006}. %
 W~przypadku teorii oddziaływań grawitacyjnych bardziej fundamentalna będzie każda teoria, która unifikuje je z~pozostałymi. W~tym przypadku szukana teoria jest kwantowa, ponieważ natura zjawisk jest kwantowa. Zatem teoria oddziaływań grawitacyjnych jako klasyczna nie jest fundamentalna w~tym sensie, że istnieje w~stosunku do niej jakaś teoria nadrzędna.

Dla nas przykładem teorii efektywnej jest standardowy model cząstek. Teorie efektywne posiadają kilka charakterystycznych własności: 1) działają w~pewnym obszarze rzeczywistości (np. OTW nie opisuje świata w~fazie Plancka, gdzie powinna być zastąpiona teorią kwantową); 2) dostarczają wyjaśniania zjawisk w~ograniczonym zasięgu. Trudno powiedzieć, żeby mechanika kwantowa była dla nas zrozumiała. Uczeni ciągle spierają się o~jej interpretację. Z~drugiej strony teoria ta działa (my nazwalibyśmy ją w~związku z~tym teorią efektywną, choć fundamentalną). Teorie efektywne są niewspółmierne pojęciowo, natomiast można podejmować próby konstruowania teorii, które wyjaśniają lub stanowią ogniwo między kolejnymi teoriami efektywnymi.

Obok zaprezentowania kategorii teorii efektywnych i~ostatecznych, celem naszej pracy jest pewna spekulacja nad możliwymi scenariuszami poszukiwania teorii ostatecznej i~propozycja oparcia tych strategii na pojęciu ciągu teorii efektywnych. W~przyjętym rozumieniu teorii ostatecznej, będzie to proces osiągnięcia teorii ostatecznej w~sensie cząstkowym. Cytowany Szynkiewicz w~kontekście własnych analiz wskazuje za Barrowem na cztery scenariusze rozwoju nauki, traktowanej w~aspekcie treściowym
%\label{ref:RNDwI0uwM5FZZ}(Szynkiewicz, 2009, s.~118).
\parencite[][s.~118]{szynkiewicz_teorie_2009}. %
 W~pierwszym (oznaczonym predykatami: natura nieograniczona/zdolności poznawcze nieograniczone) nauka rozwija się w~sposób nieograniczony. W~tym scenariuszu nie ma mowy o~osiągnięciu wiedzy, którą nazwalibyśmy teorią/teoriami ostatecznymi. W~drugim (natura nieograniczona/zdolności poznawcze ograniczone) mamy do czynienia z~dwoma kierunkami rozwoju nauki: a) z~ograniczonymi zdolnościami poznawczymi związany może być (choć nie implikowany przez nie) także nieskończony rozwój nauki, a~zatem nie ma teorii ostatecznej, b) teoria ostateczna jest nieosiągalna, ale z~tego powodu, że ograniczone zdolności poznawcze (także ze względów ekonomicznych lub technicznej niezdolności przetworzenia ogromnej liczby danych) spowodują, że ostatecznie \textit{skazani} będziemy na teorie efektywne. Trzeci scenariusz rozwoju nauki (natura ograniczona/ zdolności poznawcze nieograniczone), jest optymistyczny i~przewiduje możliwość dopracowania się teorii ostatecznej lub przynajmniej zbioru teorii fundamentalnych. Pogląd ten wspierają sukcesy na polu konstruowania teorii kwantowych. Czwarta wizja rozwoju nauki wychodzi z~założenia o~ograniczoności samej natury i~ograniczoności zdolności poznawczych. Możliwe scenariusze zależą od tego, jak mają się do siebie zakresowo obie granice. Jeśli zdolności poznawcze są bardziej ograniczone niż złożoność natury, to teoria ostateczna jest nieosiągalna. Natomiast jeśli są równe lub większe, to możemy mówić potencjalnie o~końcu nauki i~możliwości zaproponowania teorii, która będzie teorią wszystkiego nawet w~sensie homogenicznym. Warto w~tym kontekście zacytować dwa niezwykle optymistyczne zdania Weinberga ze słynnej książki \textit{Sen o~teorii ostatecznej}: ,,Osobiście uważam, że teoria ostateczna istnieje i~że jesteśmy w~stanie ją odkryć. [...] Jednej rzeczy możemy być jednak pewni: odkrycie teorii ostatecznej nie będzie oznaczało kresu nauki'' 
%\label{ref:RNDQ0HYwE4IyO}(Weinberg, 1992, s.~186.190).
\parencite[][s.~186.190]{weinberg_sen_1992}.%


\section{Teorie efektywne: charakterystyka i~funkcja}
Tezą podstawową naszej pracy jest pokazanie, że przy pomocy pojęcia ,,teoria efektywna'' możemy łatwiej dokonywać rekonstrukcji przedmiotowych procedur badawczych współczesnej kosmologii oraz formułować wnioski metanaukowe i~filozoficzne. Teoria efektywna to przede wszystkim teoria fizyczna zrelatywizowana do pewnej skali (np. przestrzennej, energetycznej) oraz pewnych oddziaływań. W~ramach takiej teorii istnieje możliwość obliczania wielkości, konstruowania obserwabli, wyznaczania parametrów etc. Zbudujmy najpierw pewną podstawową intuicję, twierdząc, że konstrukcja teorii efektywnej polega na umiejętnym zatarciu informacji o~pewnych wewnętrznych stopniach swobody w~świecie mikroskopowym 
%\label{ref:RNDkuqPzAgRdi}(Morrison, 1998, 2005).
\parencites{morrison_modelling_1998}{morrison_approximating_2005}.
Uczonego/fizyka poznajemy po tym, jak przybliża, powiedzielibyśmy: jak ,,zaciera'' informacje o~wewnętrznych stopniach swobody zbędne do opisu tego, co interesujące\footnote{Profesor Kacper Zalewski z~UJ wywarł na jednym z~autorów ogromne wrażenie, ucząc studentów, że prawdziwego fizyka możemy poznać po tym, jak przybliża. Choć tworzenie teorii efektywnych nie polega ściśle na zabiegach przybliżania, można powiedzieć analogicznie w~naszym kontekście, że prawdziwego fizyka poznajemy po tym, jak konstruuje teorie efektywne.}. Na przykład w~teorii superstrun uwzględniane są wewnętrzne stopnie swobody i~cząstki posiadają strukturę strun.

Podstawowym atutem takiego podejścia jest możliwość efektywnego opisu danego (ograniczonego) obszaru świata fizycznego, bez konieczności wyjaśnienia czy opisu całej reszty. Jeśli potraktujemy kategorię ,,teoria efektywna'' tylko intuicyjnie jako ogólną ideę, którą posługują się uczeni, można pokusić się o~stwierdzenie, że prawie cała fizyka posługuje się teoriami efektywnymi. Fizyk cząstek elementarnych Howard Georgi twierdzi, że przy pomocy teorii efektywnej dokonuje się ,,adekwatnego opisu fizyki istotnej w~danym obszarze świata przy użyciu danego zbioru parametrów''
%\label{ref:RNDpkohw4VjhD}(Georgi, 1993).
\parencite[][]{georgi_effective_1993}. %
 Teorie efektywne są jakby podporządkowane, nakierowane na specyficzny cel, któremu podporządkowany jest opis. Na przykład w~kosmologii: konstruując model kosmologiczny, abstrahuje się od poza-grawitacyjnych stopni swobody\footnote{Michał Heller nazywa to zasadą wyłączności oddziaływań grawitacyjnych 
%\label{ref:RNDForT5bsU6y}(Heller, 1969, s.~60).
\parencite[][s.~60]{heller_ewolucyjny_1969}.%
}. Nasz cel to opis własności Wszechświata w~największej możliwej skali. W~ten sposób ,,stan wiedzy o~Wszechświecie'' wyrażony w~liczbach staje się podzbiorem R\textsuperscript{\textit{n}}, gdzie \textit{n} jest liczbą niezależnych parametrów modelu. Nasuwa się tutaj analogia do konstrukcji pojęcia rozmaitości mapowanej przez podzbiory R\textsuperscript{\textit{n}}. Ujęcie wiedzy w~postaci parametrów jest lokalną jej parametryzacją; tak jak rozmaitość, jest ona parametryzowana przez podzbiory R\textsuperscript{\textit{n}}. Teoria Glashowa-Weinberga-Salama (oddziaływań elektrosłabych) jest dzisiaj teorią fundamentalną w~obszarze swego działania. Jeśli kwarki, leptony staną się w~teorii niepunktowe (będą posiadać swoją strukturę), to nastąpi przejście do nowej teorii, którą będziemy traktować jako fundamentalną w~obszarze jej działania (celu).

Smolin zauważa, że okrywaniu wszystkich fundamentalnych teorii w~fizyce towarzyszyło przekonanie, że są to teorie fundamentalne, które dokładnie opisują rzeczywistość
%\label{ref:RNDF0lbmLwjhq}(Smolin, 2015, s.~163–167).
\parencite[][s.~163–167]{smolin_czas_2015}. %
 Dopiero później zdano sobie sprawę z~tego, że teorie te posiadają status teorii efektywnych, w~których pominięto pewne stopnie swobody na rzecz efektywności opisu 
%\label{ref:RNDVEoI60rzkz}(Wilson, 2010).
\parencite[][]{wilson_non-reductive_2010}. %
 OTW uznawana jest za teorię fundamentalną. Obecnie zwraca się uwagę, że ponieważ teoria ta nie opisuje zjawisk kwantowych, jest także teorią efektywną. Eksperymenty fizyczne, np. w~dziedzinie fizyki cząstek elementarnych, sięgają zawsze tylko do pewnych skal. W~konsekwencji SMCE daje poprawne wyniki, ale należy je uznać jedynie za przybliżone. Zawsze pozostanie poza naszymi możliwościami pewien obszar energii, w~którym nasz model nie pozwala na poprawne predykcje, co, jak wiemy, jest głównym celem teorii efektywnej. Obawiamy się, że w~wyższych skalach energii albo mniejszych skalach długości nasz opis efektywny będzie wymagał korekty, tak aby tę nową rzeczywistość opisać. Dlatego Smolin konstatuje: ,,Z powodu tych obaw mówimy, że Model Standardowy jest teorią efektywną, taką, która dobrze opisuje eksperymenty, ale niezawodną tylko w~ograniczonym zakresie'' 
%\label{ref:RND8KBPdEwbXR}(Smolin, 2015, s.~164).
\parencite[][s.~164]{smolin_czas_2015}.%


Bardziej pogłębiona analiza przedmiotowa pozwala na wyodrębnienie głównych cech teorii efektywnych. Są to przede wszystkim teorie/modele, których działanie ogranicza się do pewnego obszaru fizyki. W~tym swoistym reżimie, którego granice dają się sensownie określić w~skali energii lub odległości, teoria efektywna działa, natomiast załamuje się poza nim. Warto podkreślić, że możliwość wyodrębnienia tych poziomów lub obszarów aplikacji teorii uważa się za warunek \textit{sine qua non} uznania teorii za efektywną w~podanym sensie (mówi się czasem o~\textit{punkcie} \textit{obcięcia} teorii). Szczególną cechą teorii efektywnych jest to, że specjalnego znaczenia nabiera w~nich interpretacja parametrów, którymi dana teoria się posługuje. Zwłaszcza chodzi o~interpretację tzw. \textit{input parameters}, tzw. parametrów, które traktowane są jako pewien wkład informacyjny, którym operujemy, konstruując na bazie teorii modele w~sposób zadany bez istotowych wyjaśnień. Wartości tych parametrów (np. wartość ładunku elementarnego, spin jądra, masa elektronu, własności magnetyczne) uzyskuje się drogą empiryczną (wyznaczenie eksperymentalne) albo zadaje je teoria bardziej fundamentalna. Szczególnie interesujące metodologicznie jest to, że brak zrozumienia, dlaczego są one takie, a~nie inne, nie wpływa na skuteczność posługiwania się teorią. Dana teoria efektywna jest równocześnie częścią większej całości, a~zatem ona także może dostarczać fizycznego \textit{zrozumienia} istoty parametrów, które niejako zadaje teorii bardziej efektywnej.

Właśnie ta zależność na poziomie dostarczania istotnej informacji w~postaci wartości i~wyjaśniania parametrów jest podstawą twierdzenia o~relacjach między teoriami efektywnymi. Teorie przekazują sobie niejako dane wejściowe. Tak zbudowana hierarchiczna struktura teorii efektywnych może okazać się zarówno dla badaczy, jak i~filozofów nauki interesującym materiałem do studium dochodzenia do teorii fundamentalnej -- to jest jeden z~podstawowych atutów analizy znaczenia teorii efektywnych w~metodologii. Tak, a~nie inaczej, interpretując obecność parametrów, którymi posługuje się teoria efektywna, można w~zupełnie nowy sposób definiować lokalność wyjaśniania zjawisk fizycznych oferowaną przez teorie efektywne.

Teorie efektywne łączą w~sobie wiele cech teorii i~modeli\footnote{Wyrażenie ,,teoria efektywna'' jest stosowane zarówno w~odniesieniu do modeli, jak i~teorii naukowych. Czyli model (na przykład SMK) jest teorią efektywną, ale także OTW jest teorią efektywną. Zdajemy sobie sprawę z~trudności semantycznej, natomiast w~literaturze przedmiotu nie mówi się osobno o~modelach efektywnych i~teoriach efektywnych.}. Dostarczają, podobnie jak modele, fenomenologicznego wglądu w~badane zjawisko, opisując je w~kategoriach parametrów w~przyjętej skali energii i~liczby stopni swobody
%\label{ref:RNDaMrb5idk8E}(Drăgănescu i~in., 2004).
\parencite[][]{draganescu_effective_2004}. %
 Obejmują większy niż modele obszar możliwych zastosowań. Są lepsze w~dokonywaniu predykcji nowych faktów. Za ich pomocą łatwiejsze jest formułowanie predykcji testowalnych obserwacyjnie. Szczególnie istotną własnością teorii efektywnych z~punktu widzenia metodologii i~filozofii nauki jest tworzenie struktury, która umożliwia wgląd w~to, w~jaki sposób dokonuje się proces poznawczy w~kierunku odkrycia teorii fundamentalnej 
%\label{ref:RNDBNCg50LRDU}(Castellani, 2002).
\parencite[][]{castellani_reductionism_2002}. %
 Są efektywne w~tym znaczeniu, że łatwo wyodrębnić granice, w~których operują, oraz nastawione są na możliwość efektywnego wykonania rachunku w~celu wyprowadzenia obserwabli. Na przykład Standardowy Model Kosmologiczny (SMK), który jest efektywną teorią Wszechświata, pozwolił odkryć obecną przyspieszoną ekspansję Wszechświata 
%\label{ref:RNDZ4V2SW3gRP}(Perlmutter i~in., 1998; Riess i~in., 1998; por. Tegmark i~in., 2004).
\parencites[][]{perlmutter_discovery_1998}[][]{riess_observational_1998}[por.][]{tegmark_cosmological_2004}. %
 SMK nie jest raz na zawsze ustalony. Możemy go w~przyszłości poszerzać i~prolongować na epokę Plancka. Wówczas uzyskamy nowy kwantowy model standardowy. W~standardowym modelu kosmologicznym ciemnej zimnej materii, cofając się w~czasie w~kosmicznej ewolucji, natrafiamy na moment, w~którym nieskończone stają się gęstość energii oraz krzywizna. Ten stan nazywa się osobliwym i~jest generyczną własnością modelu. Zauważmy jednak, że, gdy zbliżamy się do tego stanu odpowiadającemu zerowej objętości, wówczas, przy rozmiarach planckowskich, powinniśmy nasz klasyczny opis ,,obciąć'' i~uwzględnić kwantową naturę ewolucji Wszechświata. Istnieją pewne kwantowe modele Wszechświata, np. implikowane przez pętlową teorię grawitacji, które pokazują, że osobliwość początkowa jest zastępowana fazą \textit{bounce} (odbicia). Wszechświat w~tej fazie kurczy się do skończonych rozmiarów i, po osiągnięciu minimalnego rozmiaru, odbija się, dalej ewoluując. Innymi słowy, w~takim scenariuszu Wszechświat może być wieczny, wykazywać oscylacje, a~osobliwość początkowa jest zastępowalna fazą odbicia.

Użytecznym schematem pojęciowym w~analizie relacji między teoriami efektywnymi są pojęcia redukcji i~emergencji. Teoria emergencji, która wyrosła ona na gruncie rekonstrukcji nieredukowalnych relacji między poziomami rzeczywistości, naszym zdaniem jest możliwa jako narzędzie opisu metodologicznego
%\label{ref:RNDvCS0CfI77V}(Kim, 1999).
\parencite[][]{kim_making_1999}. %
 Stwierdzono na temat teorii efektywnych, że posługują się parametrami, których wartości dana teoria efektywna nie jest w~stanie wyjaśnić. Można zasadnie postawić pytanie o~jakość własności tych parametrów. Naszym zdaniem, biorąc pod uwagę poziom danej teorii efektywnej, są to parametry emergentne w~sensach: 1) epistemicznym (parametr emergentny pojawiający się w~strukturze teorii efektywnej nie może zostać wyjaśniony w~kontekście i~na podstawie teorii, która dotyczy składników tej struktury\footnote{Parametry kosmologiczne nie być wyjaśnione, ale mogą być przewidziane ich wartości (przypadek parametru gęstości dla stałej kosmologicznej).}) i~2) aktualizacyjnym (emergencja jest realizacją szczególnych własności istniejących w~częściach całości, ale własność emergentna jest nieprzewidywalna na podstawie wiedzy o~własnościach składników całości).

W~rekonstrukcji relacji między teoriami efektywnymi, jeśli teorie te mają strukturę hierarchiczną, to w~przyjętym przez nas kontekście zmierzania do teorii ostatecznej będą miały zastosowanie zarówno pojęcie emergencji, jak i~redukcji
%\label{ref:RNDKCvV8WTs6n}(Butterfield, 2014; Castellani, 2002; Bain, 2013).
\parencites[][]{butterfield_reduction_2014}[][]{castellani_reductionism_2002}[][]{bain_emergence_2013}. %
 Powiązane ze sobą relacją emergencji będą teorie efektywne, które ze względu na różne konotacje znaczeniowe jednakowo brzmiących terminów są niewspółmierne. Tak będzie w~przypadku relacji między teoriami grawitacji Newtona i~Einsteina. Na poziomie wyjaśniania własności parametrów efektywnych, które występują jako takie w~teorii efektywnej, przez teorię ostateczną (cząstkową) powinna wystąpić redukcja przynajmniej na poziomie epistemologicznym (wiedza o~własnościach obiektów na poziomie teorii efektywnej może być wywiedziona z~teorii bardziej fundamentalnej) lub metodologicznym (prawa fenomenologiczne na poziomie teorii efektywnej mogą być wyjaśnione z~poziomu teorii bardziej fundamentalnej).

Zobaczmy, jak wygląda taka rekonstrukcja w~klasycznym rozróżnieniu na podejście \textit{bottom-up i~top-down}. Metodologiczna rekonstrukcja pierwszej z~nich (\textit{bottom-up}) zakłada dwa scenariusze. Punktem wyjścia jest brak teorii efektywnej działającej w~danym reżimie energii. Przykładem teorii tak konstruowanej (\textit{from scratch}) jest Fermiego teoria oddziaływań słabych. Istnieje już jakaś efektywna teoria pola, która reprezentuje fizykę w~ramach danej skali, i~którą charakteryzuje pewien parametr obcięcia (to może być na przykład model standardowy lub kwantowa elektrodynamika). By otrzymać teorię dla wyższych energii trzeba uwzględnić dwa scenariusze: Nie ma cząstek między poziomami energii L1 i~L2 -- w~tym przypadku trzeba modyfikować parametry (masy, ładunki) zgodnie z~równaniami grupy renormalizacji. Może pojawić się także potrzeba uwzględnienia nowych oddziaływań. Przykładami są teorie efektywne i~modele bazujące na OTW. Może być również tak, że istnieją nowe cząstki między poziomami L1 i~L2. Ten przypadek jest bardziej skomplikowany. Najpierw masy i~ładunki muszą być zaadaptowane do nowej skali energii (rozwiązanie równań grupy renormalizacji). Później następuje identyfikacja jakościowa cząstek (fermiony czy bozony, jakie mają masy i~ładunki, jak oddziałują z~innymi cząstkami). Dokonuje się tego przez wprowadzanie nowych członów do langranżjanu, według narzędzi, którymi dysponuje kwantowa teoria pola. Procedura ma charakter teoretyczny i~eksperymentalny. Przykład stanowią supersymetryczne rozszerzenia modelu standardowego
%\label{ref:RNDjJwK0eLTIl}(Cao, Schweber, 1993; por. Cao, 1999; Arnold, 1981; Kane, 2006).
\parencites[][]{cao_conceptual_1993}[por.][]{cao_why_1999}[][]{arnold_metody_1981}[][]{kane_supersymetria_2006}. %
 W~podejściu \textit{top-down} stwierdza się istnienie teorii bardziej fundamentalnej, która jest poprawna w~danej skali L1. Celem jest zatem skonstruowanie teorii efektywnej dla niższych energii. Typowy przykład to teoria H.H. Eulera i~W. Heisenberga -- teoria czysto fotonowa uzyskana z~kwantowej elektrodynamiki przez eliminowanie elektronowych stopni swobody.

Rozwój fizyki w~drodze konstrukcji kolejnych teorii efektywnych oznacza z~jednej strony zachowanie osiągnięć teorii poprzednich, a~z drugiej strony poszerza naszą wiedzę, zmienia koncepcyjne podstawy naszej dotychczasowej wiedzy. Klasycznym przykładem tego jest fizyka Newtonowska, która dobrze opisuje świat w~swojej domenie. Możemy ją uważać za teorię efektywną opisującą świat, w~którym prędkości są dużo mniejsze od prędkości światła. OTW, która zastąpiła teorię Newtona, jest uważana za teorię fundamentalną, i~jako taka teoria powinna dać podstawowy opis praw przyrody. OTW ma cechy teorii efektywnej, tj. załamuje się dla ekstremalnie dużych energii i~powinna być zastąpiona kwantową teorią grawitacji. Reasumując, koncepcja teorii efektywnych jest zgodna z~procesem rozwoju nauki traktowanym jako pewien proces ciągły, który wymyka się tradycyjnemu w~metodologii podziałowi na kumulatywizm i~antykumulatywizm. Z~jednej strony teorie efektywne mogą współistnieć ze sobą, realizując w~zakresie swojej efektywności określone cele badawcze (wpisują się zatem w~pewien schemat kumulacyjny). Z~drugiej strony struktura powiązanych ze sobą teorii efektywnych ma cechy antykumulatywne, ponieważ kolejne teorie zastępują poprzednie. Tak jest w~przypadku modeli kosmologicznych CDM i~LCDM. Ten ostatni uznawany jest za model standardowy.

Niezwykle ciekawe i~filozoficznie inspirujące stanowisko w~tym kontekście dochodzenia dzięki teoriom efektywnym do teorii fundamentalnej zaproponował Weinberg. Otóż posłużył się on sformułowaniem: ,,obiektywny redukcjonizm''. Chodzi o~stwierdzenie pewnej konwergencji kierunków (\textit{arrows}) wyjaśniania naukowego. Fizyka cząstek jest, w~naszym pytaniu o~fundamentalność w~nauce, bliżej źródła rozchodzenia się tych kierunków na inne działy fizyki. Dopowiedzmy, że są w~tym kontekście co najmniej trzy rodzaje redukcjonizmu: redukcjonizm co do teorii -- teorię da się sprowadzić do teorii bardziej fundamentalnej; teoria efektywna byłaby szczególnym przypadkiem teorii fundamentalnej; redukcjonizm tzw. konstytutywny (ontologiczny; teorie efektywne są świadectwem na możliwość redukcji do siebie poziomów rzeczywistości); jeszcze czymś innym jest redukcjonizm eksplanacyjny (wiedza o~składowych wystarczy do wyjaśnienia zachowania się struktury złożonej).

W~dalszej części pracy zostanie zaprezentowana procedura uzyskiwania teorii efektywnej na bazie teorii układów dynamicznych. W~teorii układów dynamicznych bardzo ważnym pojęciem jest pojęcie strukturalnej stabilności. Najogólniej rzecz biorąc układ dynamiczny jest strukturalnie stabilny, jeśli jest ,,odporny'' ze względu na małe zaburzenia jego prawych stron. Pojęcia małych zaburzeń posiada dokładny matematyczny sens
%\label{ref:RNDOHyuVEANnK}(Tambor, Szydłowski, 2017).
\parencite{tambor_czy_2017}.
Układ zaburzony, który spełnia własność strukturalnej stabilności, jest w~sensie topologicznym równoważny wyjściowemu. W~przypadku układów dynamicznych na płaszczyźnie możemy scharakteryzować klasę układów strukturalnie stabilnych, a~mianowicie są to układy, które 1) nie zawierają niehiperbolicznych punktów krytycznych, 2) nie zawierają nieskończonej liczby orbit zamkniętych, nie zawierają separatrysy łączącej siodło z~siodłem. Na portrecie fazowym SMK w~skończonych obszarach przestrzeni fazowej mamy siodło, które jest strukturalnie stabilne, oraz na domknięciu w~nieskończoności stabilny i~niestabilny węzeł. Czyli układ jest strukturalnie stabilny. Treścią twierdzenia Peixoto jest, że strukturalnie stabilne układy na przestrzeni zwartej tworzą otwarte i~gęste podzbiory w~przestrzeni wszystkich układów dynamicznych na płaszczyźnie.

\section{Teorie efektywne i~ostateczne: wzajemne relacje}
Efektywność teorii oznacza w~sensie potocznym, że w~ramach danej teorii posiadamy możliwość uzyskania z~pewnego zbioru zasad wyników opartych na obliczeniach, które dają się przy pomocy teorii wykonać. Koncepcja teorii efektywnych wyrasta z~pragmatyzmu pozyskiwania wyników, co stanowi jej cel nadrzędny. Zgodzimy się z~faktem, że w~rzeczywistości dla opisu zjawiska większość towarzyszących mu wydarzeń będzie bez znaczenia, a~zatem można je pominąć. W~tym sensie koncepcja teorii efektywnych jest modyfikacją pesymistycznej maksymy ,,everything affects anything''
%\label{ref:RNDHrzSM5WMVt}(Wells, 2012, s.~1).
\parencite[][s.~1]{wells_effective_2012}. %
 Konstrukcja każdej dobrej teorii efektywnej rozpoczyna się od starannej analizy tego, co jest ważne, a~co można pominąć bez szkody dla dokładności naszego opisu. Funkcja użyteczności opisu zjawiska jest zatem nadrzędnym celem w~konstrukcji teorii. Od teorii wymagamy, by opis był ekonomiczny, to znaczy, by możliwe były wartościowe przewidywania przy minimalnej liczbie parametrów wejściowych. Tę własność Jan Such nazwałby prostotą logiczną, a~nawet pięknem 
%\label{ref:RNDUhsWTfNEKQ}(por. Such, 1975, zob. także 2014, s.~45).
\parencites[por.][s.~45]{such_czy_1975}[zob. także][s.~45]{such_na_2014}. %
 Zauważmy, że teoria Newtona doskonale opisuje zjawisko grawitacji i~stąd jest teorią poprawną z~punktu widzenia efektywności, chociaż nie opisuje zjawiska czarnej dziury, co jest możliwe za pomocą Ogólnej Teorii Względności\footnote{Teoria Newtona jest poprawna w~warunkach ziemskiej grawitacji. Gdy już przejdziemy do skal układu planetarnego, konieczne jest stosowanie poprawek OTW. Oczywiście w~skalach ziemskich możemy używać poprawek relatywistycznych (na przykład w~systemie ustalania położenia GPS). Nadmieńmy, że w~systemie GPS używane są poprawki nie tylko STW, ale OTW.}. Praktycznie oznacza to, że teoria jest jeszcze niedokończona, niekompletna, a~my musimy poczynić wiele starań, by się skonfrontować z~jej słabościami, na przykład poszerzać obszar jej stosowalności w~drodze do konstrukcji nowej teorii.

Jak określić, że dana teoria efektywna jest lepsza od innej? Otóż nadrzędne będzie kryterium użyteczności. W~konstrukcji teorii świadomie ignoruje się pewne czynniki, skupiając się na tym, by można było w~jej ramach ,,liczyć''. Ponieważ w~fizyce dysponujemy eksperymentem, w~kosmologii dodatkowo obserwacjami astronomicznymi oraz znamy dobrze obszar stosowalności teorii, potrafimy ocenić niepewność spowodowaną pominięciem wspomnianych czynników. W~przypadku teorii efektywnych (w fizyce efektywne teorie pola i~cząstek, teorie rozpraszania pionów) potrafimy odseparować to, co ważne, relewantne, od tego, co nieistotne. Ważne jest to, że potrafimy w~oparciu o~wypracowane techniki obserwacyjne odróżniać to, co jest ,,szumem'', od tego, co jest istotą zjawiska. Nieistotne zaburzenia potrafimy identyfikować i~przez to kontrolować ich wpływ na nasz opis, który w~tym sensie jest efektywny.

W~fizyce i~kosmologii istnieją dwie przeciwstawne tendencje. Teoretycy budują modele teoretyczne, natomiast tzw. obserwatorzy robią coś zgoła odmiennego, a~mianowicie zacieśniają klasę dopuszczalnych modeli znajdując ograniczenia na teorie i~ich parametry. Czyli jedni poszerzają spektrum możliwości, a~drudzy je zawężają. Wyobraźmy sobie ansambl teoretycznych możliwości, tj. przestrzeń, której każdy punkt reprezentuje pewien model teoretyczny, powiedzmy Wszechświata.

Załóżmy, że ta przestrzeń posiada porządną strukturę
%\label{ref:RNDzpAsj8x3yO}(Szydłowski, 1982).
\parencite[][]{szydlowski_metoda_1982}. %
 Jeżeli przyjąć założenie, że przestrzeń jest jednorodna (niekoniecznie izotropowa), czyli posiada pewne symetrie, to wówczas równania Einsteina opisujące ewolucję Wszechświata redukują się do postaci układu równań różniczkowych zwyczajnych -- układu dynamicznego autonomicznego. Rozważmy przestrzeń, której każdy punkt jest takim układem dynamicznym. Jest on dokładnie identyfikowany poprzez jego gładkie prawe strony $f^i\left(x\right)=\frac{\mathit{dx}^i}{\mathit{dt}}=f^i\left(x^i,{\dots},x^n\right);f^i{\in}C^r$, będące składowymi pola wektorowego określonego na przestrzeni stanów układu $\left(x^i,{\dots},x^n\right)$ -- przestrzeni fazowej. Klasa jednorodnych modeli kosmologicznych jest podzbiorem tej przestrzeni. W~przestrzeni tej można wprowadzić metrykę przy pomocy normy Sobolewa. $\left\|f\right\|_r=\underset{x{\in}C}{\mathit{max}}\{\left|f\left(x\right)\right|,\left|{\partial}_1f\right|,{\dots},\left|{\partial}_rf\right|\}$, gdzie \textit{C}~jest domkniętym zbiorem w~przestrzeni fazowej 
%\label{ref:RNDhrBkTpTVHk}(Perko, 1996).
\parencite[][]{perko_differential_1996}. %
 Przestrzeń z~tak zdefiniowaną metryką ma strukturę przestrzeni Banacha (unormowanej zupełnej), gdzie ta konstrukcja została wykonana dla podklasy modeli jednorodnych i~izotropowych 
%\label{ref:RNDicupgv4yea}(Szydłowski, Kurek, 2007; por. Szydłowski, 2007).
\parencites[][]{szydlowski_towards_2007}[por.][]{szydlowski_cosmological_2007}. %
 W~przypadku jednorodnych izotropowych modeli kosmologicznych rozważana przestrzeń układów dynamicznych jest podzbiorem przestrzeni układów dynamicznych na płaszczyźnie. W~tej przestrzeni możemy wyróżnić podzbiory typowe (generyczne) i~nietypowe. W~tym kontekście użyteczne jest pojęcie strukturalnej stabilności 
%\label{ref:RNDK8wh0ld9Uj}(Tambor, Szydłowski, 2017).
\parencite[][]{tambor_czy_2017}. %
 Treścią twierdzenia Peixoto jest to, że układy strukturalnie stabilne na zwartej przestrzeni tworzą w~przestrzeni układów dynamicznych na płaszczyźnie podzbiory typowe -- w~matematycznym sensie otwarte i~gęste podzbiory. Jest to matematyczne wysłowienie tego, że dany podzbiór tworzy ,,duży'' podzbiór (a nie residualny) miary zero.
 
Zacieśnianie zbioru modeli możemy zrekonstruować przy pomocy ciągu odwzorowań zwężających, których obrazy będą tworzyć zstępujący ciąg zawierających się w~sobie zbiorów. Co wynika z~tej konstrukcji dla odpowiedzi na nasze pytanie? Otóż wiele, ponieważ możemy się posiłkować tzw. twierdzeniem Banacha o~punkcie stałym, które powiada, że ten zstępujący ciąg jest zbieżny do punktu stałego odwzorowania zwężającego, $f(X)=X$. Czyli wiemy, że taka granica istnieje. To oznacza, że \textit{X}~jest tym, co w~pytaniu nazywamy punktem konwergencji. Model można by traktować jako iluzję, choć użyteczną, gdyby takiej granicy nie było. W~naszej konstrukcji założyliśmy, że ansambl jest ustalony. Jednak tak nie musi być i~wtedy \textit{X}~będzie użyteczną iluzją albo fikcją. Oczywiście w~ogólnym przypadku tego zrobić się nie da, ale w~pewnych przypadkach, jak zostało pokazane przez Marka Szydłowskiego, jest to możliwe dla modeli kosmologicznych, w~których występują tzw. parametry kosmologiczne. Wówczas modele kosmologiczne reprezentowane są przez układy dynamiczne i~można skonstruować przestrzeń tych modeli kosmologicznych z~metryką Sobolewa. Ta przestrzeń jest przestrzenią zupełną Banacha
%\label{ref:RNDqSmjYD9Wjf}(Szydłowski, 2007).
\parencite[][]{szydlowski_cosmological_2007}. %
 Zdajemy sobie sprawę, że Szydłowskiego multiwers modeli kosmologicznych z~ciemną energią jest w~pewnym sensie \textit{toy model}, ale istnieje możliwość takiej konstrukcji, w~której przestrzeń posiada strukturę zupełnej przestrzeni Banacha. Szydłowski pokazuje, że obszary wiarygodności dla parametrów modelu kosmologicznego mogą definiować ciąg obrazów odwzorowań zwężających. Dla jasności warto przytoczyć słowa autora: ,,W swojej pracy doktorskiej sformułowałem powyższą koncepcję, ale dopiero dzisiaj potrafię zrozumieć jej ograniczenia. Rzecz polega na tym, że kompletna obserwacyjna przestrzeń stanów nie jest nam dana od Pana Boga. Jesteśmy zdani na posługiwanie się prostymi, naiwnymi modelami z~założoną przestrzenią parametrów. Przestrzeń ta, podobnie jak Wszechświat, będzie podlegać zmianom, na przykład jej wymiar może ulec redukcji''.

Wyjściem z~tej sytuacji jest odniesienie do obecnej epoki, dla której parametry kosmologiczne oznaczamy indeksem ,,0''. Na przykład H\textsubscript{0} oznacza obecną wartość parametru Hubble'a, który oczywiście jest zmienny w~czasie ewolucji. Wydaje się, że wtedy można zdefiniować zbiór istotnych parametrów kosmologicznych w~następujący sposób\footnote{W~standardowym modelu kosmologicznym występują parametry, które są traktowane jako istotne. Pozostałe są z~nich wyprowadzalne. Na ich zestaw składają się parametry gęstości określające udział składników materii-energii we Wszechświecie, a~także parametry charakteryzujące widmo zaburzeń promieniowania reliktowego.}. Załóżmy, że posiadamy model \textit{k}-parametrowy oraz inny konkurencyjny \textit{k+1}-parametrowy. Załóżmy, że dysponujemy pewnym kryterium pozwalającym porównać modele z~punktu widzenia jakości ich dopasowania do danych obserwacyjnych. Dodanie nowego parametru poprawia lub przynajmniej nie zmienia dopasowania. Najlepiej dopasowany model miałby nieskończoną ilość parametrów, chociaż większość z~nich nie wnosi nic istotnego do opisu Wszechświata. Dlatego, kierując się kryterium prostoty logicznej, wybieramy model z~\textit{k} istotnymi parametrami. Istnieją pewne kryteria informacyjne pozwalające określić, na ile dodany parametr poprawił dopasowanie, tak by można było o~nim mówić, że jest ważnym parametrem modelu
%\label{ref:RNDzU59fsm6CS}(Szydłowski, Kurek, 2007).
\parencite[][]{szydlowski_towards_2007}.%


Wadą konstrukcji Szydłowskiego jest to, że rozważa się multiwers na przykład wszystkich układów dynamicznych na płaszczyźnie, w~R\textsuperscript{3} itd. Wymiary tego układu zależą od przestrzeni stanów tego układu użytych do parametryzacji jego ewolucji. Prezentowana przez niego konstrukcja ogranicza się do przypadków, gdy ta przestrzeń stanów jest dwuwymiarowa. Innymi słowy, nie można zdefiniować kategorii wieloświata jako przestrzeni wszystkich układów dynamicznych dowolnie wymiarowych. Musimy się z~każdym zawęzić do wymiaru. Najprostsze modele kosmologiczne typu Friedmana są reprezentowane jako dwuwymiarowe układy dynamiczne typu Newtonowskiego
%\label{ref:RNDTm9CchEzMy}(Szydłowski, Kurek, 2007; por. Szydłowski, 1983).
\parencites[][]{szydlowski_towards_2007}[por.][]{szydlowski_filozoficzne_1983}. %
 W~tym kontekście tzw. układy strukturalnie stabilne tworzą układy i~gęste podzbiory zgodne z~tw. Peixoto. Dynamika SMK jest opisana przez układ dynamiczny, w~którym parametrami są parametry gęstości -- wielkości wyznaczane z~obserwacji. Ewolucja Wszechświata opisanego przez model jest opisana przez trajektorie w~przestrzeni fazowej -- w~przestrzeni wszystkich możliwych stanów układu. Przestrzeń fazowa jest podzielona przez trajektorie modelu płaskiego na ewolucje modeli zamkniętych i~otwartych. Przestrzeń fazową czynimy zwartą poprzez konstrukcję sfery Poincarégo
% \label{ref:RNDz9Pt3syyD8}(Perko, 1996).
\parencite{perko_differential_1996}.
 Należy przy tym odróżnić przestrzeń modelu kosmologicznego od przestrzeni fazowej, która geometryzuje wszystkie możliwe ścieżki ewolucyjne rozwiązań dla wszystkich dopuszczalnych warunków początkowych.

Warto zwrócić uwagę także na to, że w~strukturze każdej teorii fizycznej istnieją stałe fizyczne. Mają one wymiary, takie jak prędkość światła c, stała grawitacji G~etc., bądź przy ich pomocy możemy skonstruować pewne stałe bezwymiarowe, jak np. stała struktury subtelnej. Zauważamy fakt ich obecności w~efektywnych teoriach fizycznych. Przykładowo w~Modelu Standardowym mamy ich aż 19
%\label{ref:RND8McR1g1iZ5}(Duff, 2014).
\parencite[][]{duff_how_2014}. %
 Stałe takie jak stała Plancka (h), prędkość światła (c), stała grawitacji (G), ładunek elementarny (e) są konstrukcjami, którymi się posługujemy, aby mierzyć pewne wielkości fizyczne. Rodzi się pytanie, czy mają one jakiś fundamentalny sens? Wielu fizyków uważa, że należą one do podstawowych własności każdej efektywnej teorii fizycznej, a~ich wartości mogą być wyznaczane przez teorię bardziej fundamentalną. Istnieje wśród uczonych przekonanie, że teoria ostateczna, do której zmierzamy, nie powinna zawierać stałych wymiarowych, czyli powinna być wolną od tego, że to my jako istoty ludzkie poznajemy rzeczywistość fizyczną. Tym niemniej powinna zawierać stałe, ale bezwymiarowe. Jest to punkt wyjścia dla rozważań nad problemem zmienności stałych fizycznych względem czasu kosmologicznego i~stąd zmienności praw fizyki 
%\label{ref:RNDXxmR2toDbl}(Volovik, 2002).
\parencite[][]{volovik_fundamental_2002}.%


\section{Stevena Weinberga \textit{Sen o~teorii ostatecznej} a~realia metodologiczne}
Szynkiewicz w~studium metodologicznym na temat teorii ostatecznych zwraca uwagę na różne sposoby rozumienia tego, czym jest teoria ostateczna. Ma ona posiadać zasadniczo trzy własności, będąc: 1) prosta logicznie (tj. o~dużej zawartości informacyjnej przy minimalnej liczbie założeń wstępnych), 2) nieuchronna (chodzi o~stopień subiektywnego przekonania uczonego o~jej adekwatności) oraz 3) logicznie jednoznaczna, to znaczy zasadniczo niemodyfikowalna i~o~stabilnej strukturze 
%\label{ref:RND0Jiz889cii}(Szynkiewicz, 2009, s.~17).
\parencite[][s.~17]{szynkiewicz_teorie_2009}. %
 Ta ostatnia cecha często nazywana jest zbiorczo własnością domknięcia lub zamknięcia teorii; tzn. wskazuje na ,,brak możliwości wprowadzenia ulepszeń do już sformułowanych koncepcji bez radykalnej zmiany ich struktury'' 
%\label{ref:RNDQdEjRytRGR}(Szynkiewicz, 2009, s.~19).
\parencite[][s.~19]{szynkiewicz_teorie_2009}. %
 Podsumowując Weinberga charakterystykę własności teorii ostatecznych, Szynkiewicz wymienia cztery cechy: zamkniętość (próby modyfikacji prowadzą do rozpadu struktury teorii), elementarność, piękno, kompletność (zupełność, teoria dostarcza wyjaśnienia pewnej klasy zjawisk, której dotyczy). Uważamy, że te cechy nie są wystarczające (choć są konieczne) do opisu teorii ostatecznej, bo równie dobrze stosują się do teorii efektywnych, które jednak nie są ostateczne. Naszym zdaniem własnością istotną teorii ostatecznej jako takiej jest przede wszystkim także to, że nie tylko wyjaśnia i~charakteryzuje wszystkie obiekty i~zjawiska należące do danej klasy, ale niejako wyjaśnia sama siebie. To znaczy, że posługuje się parametrami bez wewnętrznego wyjaśnienia ich natury i~wielkości.

Weinberg w~znanej książce \textit{Sen o~teorii ostatecznej}
%\label{ref:RNDB2qJs41GK2}(1992),
\parencite*[][]{weinberg_sen_1992}, %
 zawarł swój pogląd na temat teorii ostatecznej, czyli teorii wszystkich oddziaływań fundamentalnych, ich symetrii i~cząstek elementarnych. Sam dokonał odkrycia teorii unifikującej oddziaływania elektryczne i~słabe, co jest wielkim sukcesem fizyki współczesnej. Postanowiliśmy zawrzeć poglądy Weinberga w~osobnym paragrafie, stanowią one bowiem pewną spójną całość myślenia na temat teorii ostatecznej. Podzielamy ponadto jego pogląd, formułowany na podstawie doświadczeń historycznych i~własnych obserwacji poczynionych w~kontekście aktywnego uprawiania fizyki.

Weinberg uważa, że teoria ostateczna nie wymaga już wyjaśnienia i~rozważa, w~jaki sposób fizycy mogą do niej dojść. Naiwnością jest sądzić, że obecnie mamy już odpowiedni zestaw pojęć, które mogły by nam posłużyć do sformułowania teorii ostatecznej
%\label{ref:RNDzAHI6V2Bgb}(Weinberg, 1992, s.~139).
\parencite[][s.~139]{weinberg_sen_1992}. %
 Bardzo prawdopodobne jest natomiast nasze odejście od mechanistycznej wizji świata lub dualizmu cząstek i~pól. Z~jednej strony występują duże trudności w~kwantowym opisie pola grawitacyjnego, z~drugiej strony próby przezwyciężenia tych trudności doprowadziły do sformułowania nowej teorii, w~której pola kwantowe są niskoenergetycznymi przejawami cząstek elementarnych nazywanych strunami.

Weinberg nie podziela często formułowanej wrogości w~stosunku do teorii superstrun, którą uważa za obecnie jedyną kandydatkę do miana teorii ostatecznej\footnote{ Warto przytoczyć ten pogląd uczonego, mimo że dyskusja nad statusem tej teorii ma już swoją historię
%\label{ref:RNDN6aCkAVTz8}(por. Weinberg, 1992, s.~174).
\parencite[por.][s.~174]{weinberg_sen_1992}.%
 }. Według niego, fizyka współczesna ma problem nie tyle z~ontologią, co z~epistemologią, w~ramach której poszukujemy odpowiedzi na pytanie o~istotę i~pochodzenie naszej wiedzy. Jego zdaniem teorie fizyczne mogą zawierać pewne pojęcia czy aspekty, które nie są potwierdzone przez obserwacje czy eksperyment. Są pojęciami, którym -- jak twierdzi -- nie odpowiada żadna realność. W~tym kontekście pisze: ,,Jest to ważny problem, albowiem gdyby pozytywizm był słuszny, moglibyśmy zdobyć ważne wskazówki co do elementów przyszłej teorii ostatecznej, rozważając eksperymenty myślowe, w~których zbadalibyśmy, jakie elementy w~zasadzie obserwować'' 
%\label{ref:RNDxJMiZBRsik}(Weinberg, 1992, s.~141).
\parencite[][s.~141]{weinberg_sen_1992}.%


Weinberg uważa, że teoria ostateczna istnieje, a~fizycy są w~stanie ją odkryć. W~tym kontekście sądzi, że eksperymenty przy użyciu superakcelearatora mogą pomóc w~odkryciu teorii ostatecznej
%\label{ref:RNDvoHwh0qxGy}(Weinberg, 1992, s.~186).
\parencite[][s.~186]{weinberg_sen_1992}. %
 Jest to bardzo optymistyczny punkt widzenia, który nie do końca podzielamy, ponieważ nigdy nie będzie możliwe badanie zachowania cząstek o~energiach Plancka, co ma znaczenie w~kontekście kosmologii kwantowej opisującej Wszechświat w~tej najwyższej skali energii w~otoczeniu jego początku. \mbox{Weinberg} odpowiada na ten zarzut, stwierdzając, że taką teorię można znaleźć bez konieczności testowania fizyki w~obszarach planckowskich i~pisze: ,,[\ldots] być może znajdziemy taką teorię wśród rozważanej obecnie teorii strun [\ldots]. Według mnie, w~najlepszym wypadku możemy liczyć na to, że teoria ostateczna nie będzie logicznie konieczna, ale będzie logicznie wyizolowana'' 
%\label{ref:RNDjNF3uHqX27}(Weinberg, 1992, s.~187).
\parencite[][s.~187]{weinberg_sen_1992}. %
 Odkryta przez nas teoria ostateczna będzie, jego zdaniem, już tak sztywna i~jednoznaczna, że wszelkie próby dokonywania w~niej małych zmian będą prowadzić do logicznych sprzeczności. Weinberg zastanawia się również, jakie skutki spowoduje samo odkrycie teorii ostatecznej. Czy to oznacza kres nauki? Według niego odkrycie teorii ostatecznej nie będzie końcem fizyki. Dopiero po jej odkryciu staną się dla nas jasne skutki jej działania: ,,Być może prawa rządzące wszechświatem będą dla nas niespodzianką, jaką dla Talesa byłyby zasady dynamiki Newtona'' 
%\label{ref:RNDkyQ05oZJRM}(Weinberg, 1992, s.~190).
\parencite[][s.~190]{weinberg_sen_1992}.%


\section{Zakończenie}
W~pracy przedstawiliśmy charakterystykę teorii efektywnych w~nauce oraz przeprowadziliśmy analizę ich znaczenia dla metodologii i~filozofii nauki z~punktu widzenia dyskusji nad fundamentalnością. Przypomnijmy najważniejsze założenia ogólne i~własności teorii efektywnych, o~których była mowa w~niniejszym artykule. Na najbardziej fundamentalnym poziomie badania i~opisu istnieją głębokie prawa i~zasady nimi rządzące, do których zmierzamy. Do ogólnych praw podstawowych zmierzamy poprzez ciąg konstrukcji pośrednich. Na każdym poziomie wyjaśniania wygodnie jest operować prawami pomocniczymi albo wtórnymi, charakterystycznymi dla danego poziomu; nasza droga wiedzie w~górę, ale na każdym poziomie możliwe jest wprowadzanie nowych praw, nowych struktur, nowego -- bardziej adekwatnego -- języka. Ten nowo skonstruowany język stwarza możliwość lepszego opisu zjawisk emergentnych. Teorie efektywne zawierają pewną liczbę parametrów, które powinny być wyznaczone empirycznie albo wynikać z~bardziej fundamentalnej teorii; obecność tych parametrów gwarantuje czasową spójność teorii. Stosując teorie efektywne świadomie rezygnujemy z~atrybutów przypisywanych teorii fundamentalnej, takich jak m.in. renormalizowalność na rzecz wyjaśniania w~liczbach, łatwiejszego wyprowadzania praw, obserwabli dla planowania obserwacji i~eksperymentów
%\label{ref:RNDJQaenxlw8O}(Butterfield, Bouatta, 2015).
\parencite[][]{butterfield_renormalization_2015}.%


Gdy myślimy o~poznaniu fizykalnym w~kontekście dyskusji nad teoriami efektywnymi, może pojawić się przekonanie, że, jeśli chcemy poznawać świat w~takich kategoriach, rezygnujemy niejako od samego początku z~odkrywania podstawowych praw przyrody, będącego drogą do uzyskania pewności. Nie zgadzamy się z~taką tezą i~uważamy, że ten punkt widzenia wynika ze złej percepcji nauki. Nasze badania odnoszą się raczej do prób sformułowania na gruncie nauki aproksymacyjnej koncepcji prawdy.

Główne próby modyfikacji klasycznej teorii prawdy w~duchu aproksymacyjnym, to poza Popperem, propozycje N.~Cartwright, B.~van Fraassena, I.~Hackinga lub R.~Giera. Małgorzata Czarnocka, wskazując na problemy z~klasyczną referencyjną teorią prawdy, analizuje próby ,,ratowania klasycznej \textit{idei} prawdy'', co dokonuje się na drodze przekształcenia przedmiotu prawdy w~obiekt tworzony w~operacjach poznawczych
%\label{ref:RNDo3SQr9onaq}(Czarnocka, 1996, s.~104).
\parencite[][s.~104]{czarnocka_modyfikacje_1996}. %
 Spośród aproksymacyjnych teorii prawdy, które omawia Czarnocka, szczególnie bliska naszym analizom jest teoria prawdy cząstkowej i~prawdy stopniowalnej R.N. Gierego. ,,Abstrakcyjny model pełni funkcję obiektu, któremu przypisywana jest prawda. Zbiór abstrakcyjnych modeli tworzy wiedzę'' 
%\label{ref:RNDTKibYpXTo6}(Czarnocka, 1996, s.~110).
\parencite[][s.~110]{czarnocka_modyfikacje_1996}. %
 W~naszym podejściu zaprezentowany przykład Szydłowskiego multiwersu modeli kosmologicznych jest taką właśnie konstrukcją. Poza tym, naszym zdaniem, samo istnienie tendencji dążenia do prawdy jest ważniejsze niż udzielanie odpowiedzi na pytanie, czy jest to ideał, czy użyteczna fikcja. Taką odpowiedź sugeruje nam na przykład filozoficzne spojrzenie na fizykę przez pryzmat teorii efektywnych. Zwróćmy uwagę, że w~tym momencie rozróżniamy efektywne modelowanie w~nauce od efektywnego modelowania w~filozofii nauki.

W~pracy rozważaliśmy przykłady teorii efektywnych, które często są teoriami nierenormalizowalnymi i~stąd, w~pierwotnym sensie Weinberga, niefizycznymi. Od teorii ostatecznej oczekujemy, żeby była teorią renormalizowalną. W~wykładzie noblowskim Weinberg argumentuje, że kryterium renormalizowalności teorii wskazuje nam drogę do teorii ostatecznej. To kryterium pozwala na wybranie z~wielu teorii fizycznych jednej prawdziwej. Weinberg jest umocowany w~tradycji naukowego realizmu i~dla niego teorie opisują realny świat. Teorie efektywne mogą, ale nie muszą być renormalizowalne\footnote{Innymi słowy, teorie efektywne mogą być nierenormalizowalne. Jeśli teorie efektywne nie są teoriami renormalizowalnymi, należy wykazywać dużą ostrożność w~ich interpretacji. Często jest tak, że teorie same obnażają swoje granice
%\label{ref:RNDfao4cAdhm2}(Golbiak, Szydłowski, 2005; Butryn, 2011).
\parencites[][]{golbiak_kosmologia_2005}[][]{butryn_czy_2011}.%
}. Proponujemy w~związku z~tym scenariusz dojścia do teorii ostatecznej w~drodze konstrukcji teorii efektywnych, zbieżnych do punktu stałego (w duchu aproksymacyjnej teorii prawdy).

Weinberg argumentuje, że OTW oraz standardowy model cząstek są wiodącymi członami w~efektywnej teorii pola
%\label{ref:RNDp4y8y1xwKH}(Weinberg, 2016).
\parencite[][]{weinberg_effective_2016}. %
 Filozoficznie bliska propozycji Weinberga dotyczącej postulowanego przez niego realizmu poznawczego w~wersji konwergencyjnej (nauka zbliża się do prawdy asymptotycznie jako do granicy), jest idealizacyjna wersja korespondencyjnej teorii prawdy. Podstawowym odniesieniem poznawczym w~badaniu układów fizycznych jest ich korespondencja z~modelem/teorią efektywną. Jak pokazaliśmy w~pracy, istnieją pewne narzędzia teoretyczne pozwalające na całkiem ciekawą i~racjonalną metodologicznie analizę rodzaju aproksymacji, której dokonujemy dążąc do wiedzy pewnej. W~tym kontekście teoria ostateczna może być interpretowana jako punkt skupienia zbioru konwergentnych teorii efektywnych.

Jako przykład teorii, która wpisuje się w~ten schemat strategii, można podać kosmologię współczesną. Standardowy model kosmologiczny traktujemy jako teorię Wszechświata w~jego największej skali, badającą jego strukturę i~ewolucję. Nie jest to jednak model prawdziwy. Jest natomiast efektywnym opisem Wszechświata opisującym obserwacje. Model zawiera pewne pojęcia teoretyczne: np. ciemnej energii i~ciemnej materii, które są bytami postulowanymi. W~języku Cartwright powiedzielibyśmy -- użytecznymi fikcjami. Uważamy, że są one pojęciami emergentnymi -- modelujące je parametry stają się istotne z~punktu widzenia danych empirycznych\footnote{W~SMK możemy wyznaczyć z~obserwacji wartości parametru gęstości dla ciemnej materii i~ciemnej energii. Jeśli te pojęcia mają naturę substancjalną, to rodzi się pytanie, jakie cząstki składają się na ciemną materię czy ciemną energię. Do tej pory nie ma odpowiedzi na to pytanie. Istnieje alternatywa: natura ciemnej energii i~ciemnej energii jest niesubstancjalna, to znaczy, jej efekty można wyjaśnić na gruncie zmodyfikowanej teorii grawitacji.}. Dzisiaj możemy je mierzyć, estymować, ale w~przyszłości ich natura zostanie wyjaśniona przez inną teorię efektywną z~wyższego poziomu, dla której model LCDM będzie teorią fenomenologiczną. Obecnie poszukujemy teorii kosmologicznej mikroskopowej, która nada sens pojęciom ciemnej energii i~ciemnej materii.

J. Horgan w~książce \textit{Koniec nauki}
%\label{ref:RND2dtzmBbG7E}(1999)
\parencite*[][]{horgan_koniec_1999} %
 wiele miejsca poświęca rozważaniom na temat końca fizyki i~kosmologii. Autor analizuje poglądy m.in. Lindeya, który słusznie zauważa, że obecnie fizycy zajmujący się teorią superstrun, mającą ambicje pretendowania do miana teorii ostatecznej, bardziej niż eksperymentalnym potwierdzeniem teorii zajmują się jej estetycznymi walorami. To może jego zdaniem sprawić, że ta teoria stanie się gałęzią estetyki 
%\label{ref:RNDtGa0487vF1}(Horgan, 1999, s.~94).
\parencite[][s.~94]{horgan_koniec_1999}. %
 Weinberg twierdzi, że fizycy nie udowodnią istnienia teorii ostatecznej w~takim samym znaczeniu, w~jakim robią to matematycy dowodzący twierdzeń. Gdyby jednak fizycy potrafili wyjaśnić wszystkie dane obserwacyjne (masy cząstek), naturę wszystkich oddziaływań, to odpowiednia teoria posiadłaby cechy teorii niekwestionowalnej 
%\label{ref:RNDP1DUDrPsSU}(Horgan, 1999, s.~99).
\parencite[][s.~99]{horgan_koniec_1999}. %
 Zdaniem Weinberga teoria ostateczna będzie w~pierwszej kolejności dostępna poznawczo dla ludzi znających język matematyki, upłynie jednak wiele czasu, by jej sensowność została uchwycona przez innych. Horgan w~pewnym momencie swoich analiz stawia intrygujące pytanie: ,,Czy kosmologia może się skończyć, tak jak mechanika klasyczna?'' 
%\label{ref:RNDuxLREy0auY}(Horgan, 1999, s.~134).
\parencite[][s.~134]{horgan_koniec_1999}. %
 Naszym zdaniem może powstać teoria ostateczna w~kształcie, o~którym pisał Weinberg, ale cały czas będzie ona niedokończona. Możemy sobie na przykład wyobrazić, że równania fizyki mają charakter tensorowy i~podnosimy je do kwadratu, a~następnie sumujemy. Teoria ostateczna jest sumowaniem kwadratów tych równań, które przyrównujemy do zera. Suma kwadratów dwóch liczb jest równa zero tylko wtedy, gdy każde z~nich jest równe zero. Odrębnym problemem jest to, że nawet jeśli będziemy dysponować taką teorią finalną, to czy możliwe będzie obliczanie różnych wielkości w~sposób efektywny. Pojawia się też problem złożoności obliczeniowej.


\end{artplenv2auth}