\begin{artplenv}{Zbigniew Liana}
	{Józefa Życińskiego koncepcja racjonalizmu umiarkowanego: epistemologiczna i~doksalogiczna funkcja podmiotowego \textit{commitment}}
	{Józefa Życińskiego koncepcja racjonalizmu umiarkowanego\ldots}
	{Józefa Życińskiego koncepcja racjonalizmu umiarkowanego: epistemologiczna i~doksalogiczna funkcja podmiotowego \tocname{commitment}}
	{Uniwersytet Papieski Jana Pawła II w~Krakowie\\	
	Copernicus Center for Interdisciplinary Studies\label{liana-anfang}}
	{Joseph Życiński's understanding of temperate rationalism: epistemological and doxalogical role of subjective commitment}
	{One of the main problems of modern rationalistic theories of science is the non-eliminability of the subjective factor in the development of science. Temperate rationalism of Newton-Smith was an attempt to solve this problem. J. Życiński developed his own version of temperate rationalism in which the subjective factor played much more substantial role. In the article I~am presenting his specific idea of the personal \textit{commitment} as a~necessary condition for rationalism and science. In the first section I~proceed to reconstruct Życiński's argument leading him to the conclusion of this epistemological necessity. Next in the section 2 I~present his idea of the epistemological uncertainty principle as a~consequence of the subjective \textit{commitment}. In sections 3 and 4 I~explore axiological and pragmatic aspects of the Życiński's solution. Finally, in the section 5 I~do compare his temperate rationalism with Newton-Smith's proposal in the context of the Polanyi's idea of personal knowledge showing differences in their respective approach to the role of the subjective factor in science and rationalism.}
	{subjective commitment, underdetermination principle, presuppositions, justification circle, principle of uncertainty, epistemic axiology, personal knowledge, temperate rationalism, pragmatic rationality, epistemology, doxalogy, voluntarism, Joseph Życiński.}


\section{Wstęp}
\lettrine[loversize=0.13,lines=2,lraise=-0.01,nindent=0em,findent=0.2pt]%
{F}{}ilozofia nauki, która rozwinęła się w~reakcji na rewolucję naukową XX wieku, znalazła się w~stanie kryzysu, z~którego nie można wyjść drogą prostego rozwijania tradycyjnej koncepcji nauki i~racjonalności
%\label{ref:RND2H0c9rxPB7}(Życiński, 1985, s.~199, 1988b, s.~145, 1996, s.~189)
\parencites[][s.~199]{zycinski_teizm_1985}[][s.~145]{zycinski_structure_1988}[][s.~189]{zycinski_elementy_1996}%
\footnote{Diagnozę Życińskiego sytuacji filozofii nauki XX wieku przedstawiam szczegółowo w~
%\label{ref:RNDHbFvZYnbct}(Liana, 2019).
\parencite[][]{liana_nauka_2019_liana}. %
 Tam też przedstawiam ogólniejsze uwagi natury metodologicznej na temat mojego podejścia do tekstów Życińskiego 
%\label{ref:RNDhwx1uXSGFZ}(Liana, 2019, s.~148–152).
\parencite[][s.~148–152]{liana_nauka_2019_liana}. %
 Obecny artykuł jest kontynuacją tamtych rozważań.}. Centralnym problemem metanaukowym w~porewolucyjnej nauce jest fakt nieusuwalnej obecności w~niej i~w jej rozwoju czynnika pozaracjonalnego. Ani wybór teorii, ani wybór tradycji badawczych nie jest i~nie może być w~pełni racjonalny, lecz jest w~sposób znaczący uwarunkowany pozaracjonalnie. Dwie przeciwstawne tradycje rozwiązywania tego problemu, jakie pojawiły się w~metanauce, internalizm i~eksternalizm, same okazały się nieadekwatne w~stosunku do rzeczywistej nauki i~jej historii\footnote{Kategorie internalizmu i~eksternalizmu, którymi posługuje się Życiński, były przedmiotem moich szczegółowych analiz w
%\label{ref:RNDsl4vd1SBDV}(Liana, 2019).
\parencite[][]{liana_nauka_2019_liana}.%
}. Internalizm próbował zneutralizować pozaracjonalny czynnik przesuwając go do nieistotnego dla racjonalności nauki i~jej rozwoju psychospołecznego kontekstu odkrycia wbrew faktom z~historii nauki. Fakty obalają idealistyczne rozumienie nauki i~jej rozwoju, ukazując nieusuwalną obecność czynnika pozaracjonalnego także w~kontekście uzasadnienia. Eksternalizm z~kolei w~sposób radykalny starał się zredukować racjonalność do kontekstu pozaracjonalnych przyczyn, uznając ją za przygodny, kulturowy skutek tychże przyczyn. Także to skrajnie sceptyczne rozwiązanie okazuje się nazbyt wyidealizowane i~niezgodne z~danymi z~historii nauki. Bliższe analizy rzeczywistej nauki i~dyskusje metanaukowe wykazały, że nie jest ona ani tak racjonalna, jak tego chciał naiwny internalizm, ani tak przyczynowo uwarunkowana, jak postulował naiwny eksternalizm.

Jedynym wyjściem z~kryzysu metanaukowego wydaje się być nieunikniona rewolucja metanaukowa zmieniająca dogłębnie standardy naukowości i~racjonalności\footnote{Jak zauważa Życiński
%\label{ref:RND7zEwGtAzPY}(1988b, s.~202, 2013, s.~352),
\parencites*[][s.~202]{zycinski_structure_1988}[][s.~352]{zycinski_struktura_2013_liana}, %
 rewolucja metanaukowa nie była skutkiem arbitralnych decyzji naukowych, lecz została wymuszona wewnętrzną logiką rewolucji naukowej.}. Zarówno internalizm, jak i~eksternalizm odwołują się do uproszczonego, biało-czarnego przeciwstawienia racji i~podmiotowych przyczyn, uznającego te ostatnie za element irracjonalny w~nauce 
%\label{ref:RNDK5KmjHKi7z}(Życiński, 1985, s.~227).
\parencite[][s.~227]{zycinski_teizm_1985}. %
 Obie te tradycje odwołują się, każda na swój sposób, do tej samej Baconowsko-\\-Kartezjańskiej koncepcji rozumu i~nauki, która okazuje się niezdolna rozwiązać w~sposób realistyczny, to znaczy zgody z~faktyczną historią nauki, problemu czynników pozaracjonalnych. Życiński proponuje swoje własne rozwiązanie tego problemu jako element postulowanej rewolucji metanaukowej. Proponuje wypracowanie nowej, doksatycznej koncepcji racjonalności i~nauki, która unikając pochopnych idealizacji uwzględni istotną rolę czynników pozaracjonalnych w~rozwoju nauki. Ma to być rozwiązanie poważne, to znaczy takie, w~którym czynniki pozaracjonalne przestaną być wyłącznie dodatkiem do historii odkryć naukowych, mającym budzić ciekawość czytelnika, a~staną się istotnym elementem samej racjonalności naukowej.%notabene

W~niniejszym artykule pragnę zaprezentować rozwiązanie problemu czynników pozaracjonalnych zaproponowane przez Życińskiego. Rozwiązanie to w~swej całości zawarte jest już w~pierwszym tomie \textit{Teizmu i~filozofii analitycznej}
%\label{ref:RND4Ti3j6vIiq}(1985),
\parencite*[][]{zycinski_teizm_1985}, %
 pracy dziś mało znanej. W~późniejszych książkach Życińskiego z~filozofii nauki można odnaleźć jedynie pewne doprecyzowania i~rozwinięcia poszczególnych elementów tego rozwiązania, a~także pewne modyfikacje.

Rozwiązanie Życińskiego nie powstało w~metanaukowej próżni. W~typowym dla siebie erudycyjnym stylu Życiński podejmuje wiele współczesnych sobie koncepcji i~rozwiązań, by następnie poddać je krytycznemu rozwinięciu. W~szczególności rozwija on koncepcję Newtona-Smitha racjonalizmu umiarkowanego (\textit{temperate rationalism}) oraz koncepcję Polanyia wiedzy osobowej (\textit{personal knowledge}). Nie jest to jednak zwykłe zapożyczenie idei, lecz twórcza inspiracja. W~nowym kontekście idee te uzyskują nowe metanaukowe znaczenia. By lepiej wyartykułować tę dynamikę myśli Życińskiego oraz jej specyfikę, poddam jego rozwiązanie porównaniu z~tymi koncepcjami.

Specyfika rozwiązania Życińskiego najmocniej ujawni się jednak w~zestawieniu z~analogicznym i~współczesnym mu rozwiązaniem Larry'ego Laudana, z~tak zwanym siateczkowym modelem racjonalności przedstawionym w~\textit{Science and Values}
%\label{ref:RNDo8pnQkcZNv}(1984).
\parencite*[][]{laudan_science_1984}. %
 Wprawdzie Życiński nie odwołuje się do tego modelu, niemniej odwołuje się do poglądów i~koncepcji Laudana z~wcześniejszej pracy \textit{Progress and Its Problems} 
%\label{ref:RNDHcmJCQ033F}(1977),
\parencite*[][]{laudan_progress_1977}, %
 która była podstawą do stworzenia modelu siateczkowego. Podzielając w~dużej mierze zawartą w~tej pracy diagnozę sytuacji metanaukowej Życiński przedstawia własną i~niezależną propozycję wyjścia z~metanaukowego impasu, jaki powstał w~wyniku odkrycia w~nauce nieusuwalnej obecności czynnika pozaracjonalnego. Podobieństwo tych rozwiązań jest uderzające, a~jednocześnie są one istotnie różne.

Rozwiązanie Życińskiego zawiera wiele elementów, z~których dwa on sam uznaje za najważniejsze: koncepcję konieczności podmiotowego \textit{commitment} oraz epistemologiczną zasadę niepewności. W~obecnym artykule staram się pokazać centralną z~epistemologicznego punktu widzenia rolę podmiotowego \textit{commitment}. Główną metodą prezentacji poglądów Życińskiego jest metoda racjonalnej rekonstrukcji argumentacji, która prowadzi Życińskiego do odkrycia rozwiązania, jakim jest epistemologiczna konieczność podmiotowego \textit{commitment} w~nauce. Rekonstrukcję tę przedstawiam w~punkcie pierwszym i~jest to zasadnicza i~najobszerniejsza część obecnego artykułu. W~punkcie drugim omawiam epistemologiczną zasadę niepewności jako jedną z~głównych metanaukowych konsekwencji przyjętego rozwiązania. W~kolejnym punkcie porównuję rozwiązanie Życińskiego z~modelem Laudana, by omówić konsekwencje tego rozwiązania dla kwestii domknięcia aksjologicznej luki w~racjonalności. Następnie omawiam możliwość interpretacji zaproponowanej przez Życińskiego koncepcji racjonalności w~kategoriach racjonalności pragmatycznej. Ponieważ stanowisko Życińskiego w~tej kwestii jest niejednoznaczne, staram się pokazać możliwe opcje interpretacyjne. W~ostatnim punkcie powracam do centralnego pojęcia \textit{commitment}. Ukazuję specyfikę podejścia Życińskiego do tego pojęcia oraz przedstawiam próbę Życińskiego metanaukowej obiektywizacji tego pojęcia za pomocą kategorii wiedzy osobowej. W~zakończeniu wskazuję na fakty świadczące o~aktualności problemu podjętego przez Życińskiego i~na perspektywy krytycznego rozwijania jego \mbox{koncepcji}.

\section{Argument za koniecznością podmiotowego \textit{commitment}}
Życiński przyjmuje specyficzną strategię rozwiązania problemu czynników zewnętrznych. Nie zamierza on neutralizować obecności tego czynnika w~nauce, lecz zintegrować go z~elementem racjonalnym, w~taki jednak sposób, by nie podważyć zasadniczego prymatu racjonalności. Jedyna droga, jaka mu pozostaje, to wykazać głęboko ukryty charakter \textit{pro}-\textit{racjonalny} czynnika zewnętrznego\footnote{Określenie ‘pro-racjonalny' jest moje.}. Wszakże nie po to, by pokonać i~przekonać eksternalistów, to bowiem, jak zobaczymy, nie jest jego zdaniem możliwe, ale po to, by zneutralizować kontrargument eksternalistyczny. Jeśli okaże się, że element pozaracjonalny nie tylko jest możliwy do pogodzenia z~racjonalnością naukową, lecz także że jest jej warunkiem koniecznym, to kontrargument eksternalisty i~sceptyka straci swą destrukcyjną moc.

Przedstawienie rozwiązania Życińskiego wymaga racjonalnej rekonstrukcji jego głównego argumentu, który prowadzi go do uznania konieczności podmiotowego \textit{commitment}. Argument ten jest w~dużej mierze rozproszony w~całej pracy
%\label{ref:RNDKjvDFlgh3j}(Życiński, 1985).
\parencite[][]{zycinski_teizm_1985}. %
 Istotnymi elementami tego argumentu są kolejno: fakt koniecznej kolistości przedzałożeń, epistemologiczna i~metaepistemologiczna zasada niedookreśloności oraz fakt pluralizmu metateoretycznego.

\subsection{a) Konieczna kolistość przedzałożeń}

Kluczem do rozwiązania problemu roli czynników zewnętrznych w~nauce i~metanauce jest w~przekonaniu Życińskiego
%\label{ref:RND0vK0hAJxvO}(1985, s.~7)
\parencite*[][s.~7]{zycinski_teizm_1985} %
 właściwa koncepcja przedzałożeń. Na potrzeby rozwiązania problemu czynników zewnętrznych Życiński postanawia wypracować swoją własną koncepcję przedzałożeń względnie presupozycji, rozwijając w~tym celu idee Bridgmana 
%\label{ref:RNDzM1Nacb7ZC}(1950)
\parencite*[][]{bridgman_reflections_1950} %
 i~Polanyia 
%\label{ref:RNDXnqrhGAuab}(1962)
\parencite*[][]{polanyi_personal_1962}%
\footnote{Zob. zwł. tekst angielski 
%\label{ref:RNDbXrB6NIUat}(Życiński, 1988b, s.~202):
\parencite[][s.~202]{zycinski_structure_1988}: %
 ,,my theory''; 
%\label{ref:RNDwtWvXh5dsJ}(por. Życiński, 2013, s.~351, zob. także 1988b, s.~145, 2013, s.~256, 1996, s.~187 przypis 283, 2015, s.~254 przypis 18).
\parencites[por.][s.~351]{zycinski_struktura_2013_liana}[zob. także][s.~145]{zycinski_structure_1988}[][s.~256]{zycinski_struktura_2013_liana}[][s.~187 przypis 283]{zycinski_elementy_1996}[][s.~254 przypis 18]{zycinski_elementy_2015}. %
 Wypada zauważyć, że pierwsze wydanie pracy Polanyia (przez Chicago University Press) miało miejsce w~1958~r. Wydanie z~1962~r. jest wydaniem poprawionym.}.

Idea presupozycji nie jest oczywiście nowa. Życiński
%\label{ref:RNDTyiq5VH5nd}(1993, s.~311n)
\parencite*[][s.~311n]{zycinski_granice_1993} %
 mówi w~tym kontekście o~,,globalnym presupozycjonizmie'', jaki jego zdaniem nastał w~filozofii nauki w~wyniku prac Poppera i~Kuhna. Popper pokazał, że teoria naukowa nie jest niczym innym jak narzucaną ,,na rzeczywistość siecią założeń i~pojęć zgadywanych przy biurku''. Kuhn pokazał z~kolei, że każdą wspólnotę badaczy łączy ,,zbiór założeń wyjściowych (presupozycje), których się nie uzasadnia, lecz które się przyjmuje jako warunek uprawiania nauki w~danym paradygmacie''. We własnym ujęciu Życińskiego przedzałożenia są to najbardziej fundamentalne twierdzenia określające wizję świata, koncepcję filozofii, koncepcję racjonalności oraz warunki poprawności naukowej. Są to przedzałożenia metodologiczne, ontologiczne, epistemologiczne i~korespondujące z~nimi założenia metafizyczne leżące u~podstaw paradygmatów, programów badawczych i~tradycji badawczych 
%\label{ref:RNDxW65HYfUoW}(Życiński, 1985, s.~7.133, 1988b, s.~21n.143, 2013, n. 37nn.253)
\parencites[][s.~7.133]{zycinski_teizm_1985}[][s.~21n.143]{zycinski_structure_1988}[][s.~37nn.253]{zycinski_struktura_2013_liana}%
\footnote{O~roli presupozycji w~rewolucji metanaukowej w~ujęciu Życińskiego zob. 
%\label{ref:RNDv3Q8hpeOAL}(Liana, 2019, s.~173–177).
\parencite[][s.~173–177]{liana_nauka_2019_liana}. %
 Oryginalne kategorie ‘paradygmatu', ‘programu badawczego' i~‘tradycji badawczych' Życiński adaptuje do własnej koncepcji, zmieniając częściowo ich znaczenie.}. O~ile jednak w~tradycyjnej koncepcji racjonalności przedzałożenia traktowano jako dogmatyczne przesądy 
%\label{ref:RNDBaaJ8Zz7jz}(Życiński, 1985, s.~169n),
\parencite[][s.~169n]{zycinski_teizm_1985}, %
 o~tyle w~koncepcji porewolucyjnej muszą zostać uznane za konieczny warunek racjonalności 
%\label{ref:RNDFiLaAOmUiZ}(Życiński, 1985, s.~129.157.172n).
\parencite[][s.~129.157.172n]{zycinski_teizm_1985}. %
 Ich odkrycie stanowi istotny element metanaukowej rewolucji 
%\label{ref:RNDk0HuITFshq}(Życiński, 1988b, s.~143, 2013, s.~253).
\parencites[][s.~143]{zycinski_structure_1988}[][s.~253]{zycinski_struktura_2013_liana}.%


Centralnym elementem koncepcji Życińskiego jest idea kolistości w~uzasadnieniu przedzałożeń. Aby móc wykorzystać ideę przedzałożeń do rozwiązania problemu czynników zewnętrznych, Życiński wykazuje najpierw konieczność uznania przez racjonalistę kolistości w~uzasadnianiu przedzałożeń. Z~metodologicznego punktu widzenia przedzałożenia pełnią funkcję podstawy uzasadnienia wszystkich pozostałych twierdzeń w~programie lub tradycji badawczej. Jakkolwiek ostateczna, podstawa ta nie jest jednak epistemologicznie niewzruszona. Fundamentalne przedzałożenia z~konieczności pozbawione są tradycyjnego dowodu z~uprzednio udowodnionych przesłanek
%\label{ref:RNDRtuk7KZY9G}(Życiński, 1985, s.~160, przypis 1).
\parencite[][s.~160 przypis 1]{zycinski_teizm_1985}. %
 Ale jak zauważa Życiński, nie znaczy to, że są one pozbawione jakiegokolwiek uzasadnienia, jak chciałby tego łatwy sceptycyzm. Tym samym nie są one irracjonalne. Cechuje je ,,uzasadnienie częściowe'' 
%\label{ref:RNDhfKOiLBo8b}(Życiński, 1985, s.~7.156.161.169).
\parencite[][s.~7.156.161.169]{zycinski_teizm_1985}. %
 Jest ono rozumiane przez Życińskiego jako uzasadnienie koniecznie koliste. Nie może ono mieć charakteru rozstrzygającego i~definitywnego, gdyż sposób uzasadnienia przedzałożeń wyznaczany jest przez same te przedzałożenia, co z~konieczności prowadzi do \textit{circulum vitiosum} 
%\label{ref:RND4pN0mDmmyQ}(Życiński, 1985, s.~173 oraz 129.141.164)
\parencite[][s.~173 oraz 129.141.164]{zycinski_teizm_1985}%
\footnote{Pojęcie częściowego uzasadnienia jest znane od dawna, lecz jest ono używane w~odmiennych znaczeniach niż podane przez Życińskiego. Koncepcja \textit{partial justification} pojawia się na przykład u~P. Bernaysa 
%\label{ref:RNDVP5B0nmntc}(1964, s.~38)
\parencite*[][s.~38]{bernays_reflections_1964} %
 na określenie sposobu uzasadnienia decyzji o~wyborze zdań bazowych w~koncepcji Poppera. C.R. Kordig 
%\label{ref:RNDB8qjVWCVdI}(Kordig, 1978, s.~112)
\parencite*[][s.~112]{kordig_discovery_1978} %
 pisze z~kolei o~częściowym uzasadnieniu odkrycia naukowego, negując tym samym ostre empirystyczne przeciwstawienie kontekstu odkrycia i~uzasadnienia. Współcześnie D.J. Chalmers 
%\label{ref:RNDC6dhVUglyH}(Chalmers, 2012, s.~96),
\parencite*[][s.~96]{chalmers_constructing_2012} %
 mówi o~uzasadnieniu, że jest częściowe, gdy ,,pomija pewne elementy'', podczas gdy \textit{pełne uzasadnienie} (\textit{full justification}) ,,zawiera uzasadnienie dla każdego przekonania w~strukturze''. Analogicznie jest ono rozwijane w~teorii prawa karnego w~znaczeniu \textit{częściowej obrony} (\textit{partial defense}), zob. 
%\label{ref:RND5ZfW8ZOwgd}(Eldar, Laist, 2014, ,,Introduction'').
\parencite[][]{eldar_misguided_2014}.%
}.

Wszelkie próby eliminacji metodologicznego koła presupozycyjnego, czy to internalistyczne, czy eksternalistyczne skazane są na niepowodzenie. Próby internalistyczne kończą się ostatecznie zawsze jakąś formą dogmatyzmu lub intuicjonizmu odwołującego się do pre-filozoficznych intuicji określających warunki racjonalności naukowej i~ontycznej
%\label{ref:RNDWutTx8Kwb7}(Życiński, 1985, s.~129.172).
\parencite[][s.~129.172]{zycinski_teizm_1985}. %
 Z~kolei eksternaliści przerywają to koło uznając presupozycje za wynik pozaracjonalnych uwarunkowań kulturowych i~społecznych. Jak zauważa:


\myquote{Przyjęcie którejkolwiek z~tych interpretacji jest w~istocie równoważne uznaniu tezy głoszącej, iż warunki naukowości przyjmowane w~poszczególnych paradygmatach przyrodniczych mają charakter przedzałożeniowy. Jeśli chce się w~sposób dogmatyczny bronić określonej filozofii nauki, nie istnieje wówczas praktyczna możliwość jej falsyfikacji [...]
%\label{ref:RNDW8Z3RvxxMh}(Życiński, 1985, s.~129).
\parencite[][s.~129]{zycinski_teizm_1985}.%
}
Ani internalista, ani eksternalista próbując wyeliminować błędne koło w~uzasadnianiu presupozycji, nie czyni tego w~oparciu o~jakieś bardziej fundamentalne uzasadnione racje, lecz odwołując się do własnych, odmiennych przedzałożeń na temat naukowości. Zatem ich rozwiązania mają charakter kolisty. Z~tego wynika, że wszelka próba obrony stanowiska tradycyjnego internalisty lub eksternalisty możliwa jest wyłącznie jako pewna forma dogmatyzmu, czyli uporczywego trwania przy swoich przedzałożeniach. Przeciw dogmatycznej obronie nie istnieje jednak żaden wiążący kontrargument. Dogmatycznie można bronić dowolnego stanowiska filozoficznego w~nieskończoność
%\label{ref:RND7KVfuwkQS0}(Życiński, 1985, s.~129).
\parencite[][s.~129]{zycinski_teizm_1985}.%


Życiński polemizuje jedynie z~tradycyjnym racjonalistą, który tradycyjnie odrzuca kolistość przedzałożeń jako przejaw irracjonalizmu. Zarzuca mu
%\label{ref:RNDrK0xw63f83}(Życiński, 1985, s.~156n)
\parencite[][s.~156n]{zycinski_teizm_1985} %
 przyjmowanie dwóch niespójnych standardów racjonalności: ,,normalne'' twierdzenia systemu racjonalista ocenia wedle pewnych zasad, ale same te zasady oceniania ocenia już wedle innych metazasad, które są niespójne z~przyjętymi zasadami\footnote{Z~zarzutu tego można wnioskować, że w~przekonaniu Życińskiego tradycyjny racjonalista powinien zgodnie ze swymi deklaracjami przyjmować jeden spójny sposób oceny racjonalności wszystkich twierdzeń naukowych w~duchu stoickiej tradycji racjonalistycznej domagającej się, by wszystkie twierdzenia naukowe były uzasadniane, a~przeciw nowożytnej tradycji racjonalistycznej przyjmującej istnienie pierwszych, dalej nieuzasadnialnych aksjomatów i~zasad.}. Zarzut przyjmowania dwóch odmiennych standardów oceny zasad nie wydaje się jednak argumentem wystarczająco silnym przeciw tradycyjnemu racjonaliście. Można jednak tak przeformułować ten zarzut, by miał on charakter rozstrzygający, przynajmniej według deklarowanych standardów racjonalisty. Można wykazać wewnętrzną sprzeczność i~nieusuwalny dylemat, przed jakim stoi tradycyjny racjonalista.

Dogmatyzm jest logicznie możliwy, niemniej internalista, który broni w~sposób dogmatyczny swej koncepcji, staje się wbrew sobie eksternalistą. Ostatecznym bowiem elementem rozstrzygającym o~dogmatycznej obronie takiego a~nie innego przedzałożenia epistemologicznego są czynniki arbitralno-subiektywne, jakaś forma intuicji i~przekonanie o~jej ostatecznym charakterze, a~nie racje, choćby częściowo uzasadnione. Tym samym postawa dogmatyczna tradycyjnego racjonalisty prowadzi go do wewnętrznej sprzeczności. Broni swego stanowiska w~sposób sprzeczny z~jego deklarowanymi założeniami. Jeśli jednak racjonalista zacznie bronić się przed takim zarzutem, odwołując się na przykład do odmiennego rozumienia pojęcia intuicji, to jego obrona przyjmie charakter kolisty\footnote{ Jeśli ktoś \textit{a~priori} przyjmuje istnienie jakiejś niezawodnej intelektualnej intuicji, to trudno będzie go przekonać argumentami w~duchu Kanta czy Poppera, że intuicja intelektualna albo nie istnieje, albo jest niejasna i~zawodna, i~że odwoływanie się do niej jest przejawem subiektywizmu i~psychologizmu. Co więcej, w~świetle swoich przedzałożeń zawsze znajdzie on jakieś metanaukowe \textit{fakty}, które będą potwierdzały jego tezy. Patrząc ,,z zewnątrz'', takie zachowanie dogmatyka trzeba uznać za zachowanie koliste.}. Będzie musiał odwołać się do swoich przedzałożeń w~celu ich obrony. Racjonalista staje zatem przed dylematem: albo przyjąć kolistość w~uzasadnianiu, albo przyjąć postawę eksternalistyczną i~przestać być racjonalistą\footnote{W~przypadku eksternalisty postawa dogmatyczna jest całkowicie spójna z~jego założeniami: o~wyborze eksternalistycznych przedzałożeń decydują bowiem czynniki zewnętrze, aracjonalne. Wszakże domaganie się przez eksternalistę uznania \textit{jego racji} będzie wyrazem \textit{racji} całkowicie arbitralnych, a~nie merytorycznych. Oczywiście, z~logicznego punktu widzenia skrajny eksternalista nie jest zobowiązany do logicznej spójności swych poglądów, skoro w~jego przekonaniu wszelkie racje, w~tym także racja niesprzeczności, zdeterminowane są całkowicie przez uwarunkowania pozaracjonalne i~przyczynowe. Ale także to ostatnie stwierdzenie pisane jest z~perspektywy nie-eksternalistycznej, gdyż dotyczy logicznej (racjonalnej) spójności eksternalisty.}.

Stwierdzenie nieusuwalności koła presupozycyjnego pozwala Życińskiemu skonkludować pierwszy etap swego argumentu: skoro w~obliczu presupozycyjnego koła jedyną alternatywą dla racjonalisty jest stanowisko wewnętrznie niespójne, zatem akceptacja tego koła przy ocenie podstawowych przedzałożeń jest dla tegoż racjonalisty \textit{koniecznością}
%\label{ref:RNDkaUFXx3cCn}(Życiński, 1985, s.~157).
\parencite[][s.~157]{zycinski_teizm_1985}. %
 Presupozycyjna kolistość przestaje być synonimem irracjonalizmu, a~staje się koniecznym warunkiem nowego, bardziej realistycznego rozumienia racjonalności.

Tak rozumiane metodologiczne koło presupozycyjne, będące koniecznym warunkiem racjonalności, nie prowadzi do wewnętrznej sprzeczności, a~przynajmniej do takiej, która byłaby destrukcyjna dla samej racjonalności. Zatem koło to nie jest ani do końca szkodliwe, ani do końca błędne. W~konsekwencji wyrażenia ‘\textit{circulus vitiosus}' i~‘błędne koło' stosowane przez Życińskiego na określenie tego koła nie wydają się być użyte w~sposób spójny. Tego typu niespójność terminologiczna, nie mając większego znaczenia dla poprawności argumentu Życińskiego, wskazuje jednak na pewne głębsze uwarunkowania pojęcia racjonalności.

Pod wieloma względami rozwiązanie Życińskiego kwestii kolistości jest analogiczne do tego, jakie proponują zwolennicy abdukcyjnej obrony realizmu naukowego. Korzystają oni z~idei Braithwaite'a
%\label{ref:RND9p0TIYUTIg}(1953, s.~274–278)
\parencite*[][s.~274–278]{braithwaite_scientific_1953} %
 rozróżniania \textit{koła szkodliwego} względnie \textit{błędnego} od koła \textit{niewinnego}. Kolistość nie musi być ani szkodliwa, ani błędna. Błędne koło w~uzasadnianiu występuje wówczas, gdy konkluzja zawiera się (\textit{explicite} lub \textit{implicite}) w~przesłankach. Nieszkodliwe względnie niewinne z~kolei jest to koło, w~którym konkluzja nie jest tożsama z~przesłanką, jest natomiast tożsama z~regułą rządzącą wyprowadzeniem konkluzji z~przesłanek. I~tak indukcyjne uzasadnianie indukcji, czy dedukcyjne uzasadnianie dedukcji, względnie abdukcyjne uzasadnianie abdukcji tworzą koło, które nie jest szkodliwe\footnote{Na ten temat zob. 
%\label{ref:RNDs0vlCq8cH0}(Liana, 2003, s.~138n; Psillos, 1999, s.~84; Grobler, 2006, s.~102n.265n).
\parencites[][s.~138n]{liana_naturalistyczne_2003}[][s.~84]{psillos_scientific_1999}[][s.~102n.265n]{grobler_metodologia_2006}.%
}.

Sposób, w~jaki Życiński artykułuje koło presupozycyjne, może wskazywać, że ma on na myśli takie właśnie niewinne koło metodologiczne. Mówi bowiem, że przedzałożenia wyznaczają standardy swego własnego uzasadniania:


\myquote{Przedzałożenia nie mają charakteru irracjonalnego, gdyż mogą być w~pewnym stopniu uzasadnione. Uzasadnienie to jednak nie ma charakteru definitywnego czy rozstrzygającego, zaś sposób uzasadniania określany jest właśnie w~przedzałożeniach, co nadaje całej argumentacji charakter błędnego koła, gdyż w~celu uzasadnienia np. przyjmowanych teorii racjonalności trzeba się uprzednio odwołać do racjonalnych argumentów, których racjonalność oceniania była na podstawie uzasadnianej koncepcji racjonalności
%\label{ref:RNDnwR2nhtFTP}(Życiński, 1985, s.~156 zob. także s.~173).
\parencite[][s.~156, zob. także s.~173]{zycinski_teizm_1985}.%
}
A~następnie kontynuuje: ,,Ten typ \textit{circuli vitiosi} przy analizie przedzałożeń podstawowych wydaje się więc koniecznością [...]''. Ma zatem świadomość, że chodzi o~specyficzne koło, różne od tradycyjnego błędnego koła.

Problem jednak w~tym, że nie każde koło presupozycyjne omawiane przez Życińskiego jest kołem metodologicznym w~sensie Braith\-waite'a. Pisząc o~próbie uzasadnienia realizmu epistemologicznego Życiński mówi:

\myquote{
jakiekolwiek próby uzasadnienia realizmu epistemologicznego implikują element błędnego koła: by uzasadnić realizm, odwołujemy się np. do rozwoju technologicznego, zaś w~celu stwierdzenia tego rozwoju zakładamy \textit{implicite} realizm epistemologiczny i~ontologiczny
%\label{ref:RND99nQGiYZ0M}(Życiński, 1985, s.~173).
\parencite[][s.~173]{zycinski_teizm_1985}.%
}
W~tym kole konkluzja jest tożsama z~przesłanką entymematyczną, zatem mamy do czynienia raczej z~tradycyjnym \textit{circulum vitiosum} niż z~kołem Braithwaite'a. Mimo to Życiński uznaje także to koło za konieczne i~nieszkodliwe dla racjonalisty.

W~tradycyjnej koncepcji racjonalności i~uzasadnienia kolistość w~uzasadnianiu była uznawana za koło błędne dlatego, że prowadziła do wewnętrznej sprzeczności i~tym samym podważała racjonalność przekonań. Życiński odrzuca koncepcje racjonalności negujące wartość zasady sprzeczności \textit{tout court}, w~szczególności w~odniesieniu do treści teorii naukowych\footnote{Zob. np.
%\label{ref:RNDwnPfh5veyl}(Życiński, 1996, s.~218n, 2015, s.~297n),
\parencites[][s.~218n]{zycinski_elementy_1996}[][s.~297n]{zycinski_elementy_2015}, %
 gdzie omawia ten problem w~kontekście koncepcji anarchizmu Feyerabenda.}. Sytuacja wydaje się jednak zgoła odmienna, gdy kolistość i~brak wewnętrznej spójności dotyka wyboru podstawowych przedzałożeń. W~tym szczególnym przypadku nowa koncepcja racjonalności postuluje zmianę metodologicznego statusu metodologicznej kolistości i~wewnętrznej sprzeczności. W~przypadku przedzałożeń kolistość i~brak spójności mają istotny pro-racjonalny charakter, a~uznanie ich konieczności okazuje się niezbędne dla obrony możliwości racjonalności w~ogóle\footnote{O~zmianie metodologicznego statusu zasady sprzeczności w~teorii podstaw logiki i~matematyki Życiński pisze omawiając twierdzenia limitacyjne Gödla. Zob. np. 
%\label{ref:RNDrqSdvQR434}(Życiński, 1985, s.~196, 1988b, s.~131n, 2013, s.~231nn, 1996, s.~266n, 2015, s.~360nn).
\parencites[][s.~196]{zycinski_teizm_1985}[][s.~131n]{zycinski_structure_1988}[][s.~231nn]{zycinski_struktura_2013_liana}[][s.~266n]{zycinski_elementy_1996}[][s.~360nn]{zycinski_elementy_2015}.%
}.

Powstaje pytanie o~(częściowe) uzasadnienie nieszkodliwości presupozycyjnej kolistości. Odpowiedź Życińskiego tkwi \textit{implicite} w~jego tezie o~konieczności akceptacji takiej kolistości przez racjonalistę\footnote{Na temat pragmatycznego charakteru tej konieczności zob. niżej punkt 5.}. Jedyną alternatywą dla akceptacji tej kolistości jest wewnętrzna niespójność na poziomie głoszonych tez i~swych zachowań. Tego typu niespójność jest już jednakże wyrazem irracjonalności, nie cechuje się bowiem koniecznością. Można jej uniknąć akceptując kolistość przedzałożeń. Ponownie, racjonalista ma do wyboru albo zaakceptować kolistość, albo \textit{volens nolens} przejść do obozu sceptyków
%\label{ref:RNDEO7F9t6HiQ}(Życiński, 1985, s.~165).
\parencite[][s.~165]{zycinski_teizm_1985}.%


Neo-racjonalistyczne rozwiązanie problemu kolistości w~uzasadnieniu przedzałożeń przez uznanie konieczności jej akceptacji nie rozwiązuje jednak jeszcze kwestii obecności czynników zewnętrznych w~nauce. Nie pokazuje również, w~jakim sensie racjonalistyczne rozwiązanie problemu kolistości jest lepsze od rozwiązania eksternalistycznego. Uznanie koniecznego charakteru kolistości jest jedynie punktem wyjścia. Racjonalista musi podać dodatkowe argumenty na rzecz swego rozwiązania.

\subsection{b) Epistemologiczna zasada niedookreśloności i~pluralizm filozoficznych programów badawczych}

Aby znaleźć takie argumenty, należy metanaukowy fakt koniecznej presupozycyjnej kolistości poddać analizie i~poszukać jego konsekwencji. Życiński wskazuje na istotny charakter dwóch takich konsekwencji. Są to zasada epistemologicznej niedookreśloności\footnote{W~ten sposób Życiński oddaje angielskie wyrażenie `\textit{underdetermination principle}', zob.
%\label{ref:RND1eqv1HwIVT}(Życiński, 1985, s.~129n).
\parencite[][s.~129n]{zycinski_teizm_1985}. %
 Zasadę tę należy wyraźnie odróżniać od podobnie brzmiącej zasady nieokreśloności (\textit{indeterminacy}) epistemologicznej, tożsamej z~zasadą niepewności. Ta ostatnia zostanie omówiona poniżej w~punkcie 3.} oraz fakt pluralizmu metateoretycznego.

Z~wielu znanych sformułowań zasady niedookreśloności Życiński wybiera to, które mówi, iż ten sam zbiór faktów może być w~sposób równoważny wyjaśniony przez wiele alternatywnych i~konkurencyjnych teorii. Zauważa przy tym, że pojęcie równoważności obserwacyjnej jest nieprecyzyjne ze względu na brak współrzędnej czasowej, dlatego wprowadza do tej zasady element odniesienia czasowego i~modyfikuje ją do wariantu, który określa mianem ,,słabszego'':

\myquote{
W~dowolnej chwili \textit{t}~mogą istnieć alternatywne teorie przyrodnicze, które (a) posiadają częściowe konfirmacje empiryczne i~(b) są wzajemnie sprzeczne ze sobą
%\label{ref:RNDWXsPaZGgnp}(Życiński, 1985, s.~131).
\parencite[][s.~131]{zycinski_teizm_1985}.%
}
Tradycyjna zasada niedookreśloności epistemologicznej implikuje zatem nieunikniony pluralizm teoretyczny w~nauce\footnote{W~\textit{Elementach filozofii nauki}
%\label{ref:RND6pVMIbWOZ3}(Życiński, 1996, s.~106n, 2015, s.~143)
\parencites[][s.~106n]{zycinski_elementy_1996}[][s.~143]{zycinski_elementy_2015} %
 Życiński zauważa, że tego typu obserwacyjna równoważność teorii znana jest już od czasów Duhema i~znana była także Popperowi, prowadząc do komplikacji w~procesie falsyfikacji.}.

Życiński nie widzi jednak powodu, podobnie jak Feyerabend, by tego typu pluralizm ograniczać wyłącznie do teorii przyrodniczych. \textit{A~fortiori} powinien on być obecny także w~metanauce, gdzie element przedzałożeniowy odgrywa znacznie większą rolę niż w~teoriach przyrodniczych, choćby ze względu na brak ,,jednoznacznie określonego zbioru obiektów''
%\label{ref:RNDcuo4B6L07e}(Życiński, 1985, s.~163)
\parencite[][s.~163]{zycinski_teizm_1985}%
\footnote{Życiński 
%\label{ref:RNDd67zjg2PB4}(1988b, s.~143, 2013, s.~254)
\parencites*[][s.~143]{zycinski_structure_1988}[][s.~254]{zycinski_struktura_2013_liana} %
 pisze, że im bardziej ogólne założenia epistemologiczne przyjmujemy, tym bardziej wzrasta stopień ich subiektywizmu i~niepewności.}. Pluralizm metodologiczny, pluralizm tradycji badawczych i~pluralizm koncepcji racjonalności są równie realne jak pluralizm teoretyczny w~naukach przyrodniczych\footnote{O~pluralizmie metodologicznym zob. 
%\label{ref:RNDQucUL7wfuA}(Życiński, 1988b, s.~198, 2013, s.~343),
\parencites[][s.~198]{zycinski_structure_1988}[][s.~343]{zycinski_struktura_2013_liana}, %
 o~pluralizmie tradycji badawczych zob. 
%\label{ref:RNDkPOacUISzE}(Życiński, 1985, s.~164, 1996, s.~184, 2015, s.~250).
\parencites[][s.~164]{zycinski_teizm_1985}[][s.~184]{zycinski_elementy_1996}[][s.~250]{zycinski_elementy_2015}.%
}.

Nawet pobieżna analiza sporów metanaukowych pokazuje pewien fenomen, który zdaniem Życińskiego nie został jeszcze dostatecznie zinterpretowany przez filozofów nauki, a~mianowicie to, że:

\myquote{
Poszczególnym paradygmatom przyrodniczym można przyporządkować nie tylko ciągi różnych programów badawczych [....], lecz także konkurencyjne ciągi filozoficznych programów badawczych rozwijanych w~perspektywach filozofii nauki
%\label{ref:RNDOlrApSQ1Ee}(Życiński, 1985, s.~132).
\parencite[][s.~132]{zycinski_teizm_1985}.%
}

Pomijając omówienie samej koncepcji filozoficznych programów i~tradycji badawczych, należy zauważyć, że także w~przypadku koncepcji filozoficznych występuje swoista równoważność obserwacyjna\footnote{Dotyczy to w~szczególności rozwijanej przez Życińskiego realistycznej koncepcji nauki, która nie ogranicza się do czysto logicznych analiz, lecz testuje je w~oparciu o~metanaukowe fakty z~historii nauki i~metanauki.}. W~przypadku nauk przyrodniczych dotyczy ona faktów empirycznych, w~przypadku metanauki dotyczy faktów metanaukowych, takich jak poszczególne twierdzenia naukowe, a~nawet całe dyscypliny przyrodnicze. Odmienne interpretacje metanaukowe tych samych faktów metanaukowych często prowadzą do odmiennych celów i~strategii rozwoju nauki:

\myquote{
W~perspektywach teorii filozoficznych programów badawczych te same dyscypliny przyrodnicze mogą być interpretowane w~sposób całkowicie różny, tak jak różna jest matematyka platonizmu od matematyki intuicjonizmu. [...] różnice te prowadzą nie tylko do odmiennych interpretacji znanych obecnie twierdzeń przyrodniczych, lecz także do zróżnicowanych programów rozwijania badań i~poszukiwań nowego paradygmatu naukowego
%\label{ref:RND44bBN6lLAu}(Życiński, 1985, s.~133).
\parencite[][s.~133]{zycinski_teizm_1985}.%
}


Życiński wskazuje jednak
%\label{ref:RND5Ve53wlyzV}(Życiński, 1985, s.~133.161-164)
\parencite[][s.~133.161-164]{zycinski_teizm_1985} %
 na pewną brzemienną w~skutki metodologiczną różnicę pomiędzy empirycznym niedookreśleniem teorii w~naukach przyrodniczych a~analogicznym niedookreśleniem teorii w~metanauce. O~ile w~naukach przyrodniczych możliwe jest, pomimo teoretycznego obciążenia faktów, uzgadnianie stanowisk i~swoiste dążenie do unifikacji, o~tyle w~metanauce coś takiego wydaje się całkowicie niemożliwe. Wiara w~możliwość unifikacji metateoretycznej i~w możliwość powszechnej \textit{zgody filozofów} jest co najwyżej wyrazem nierealnego \textit{wishful thinking} i~pozostałością myślenia magicznego w~filozofii\footnote{Jako przykład pozostałości myślenia magicznego Życiński (1985, s.~133) podaje idee J. Monoda zawarte w~jego \textit{Le hasard et la nécessité} \parencite*{monod_hasard_1970}.}. Stoi ona w~sprzeczności z~ideą realistycznej koncepcji nauki respektującej dane historyczne:

\myquote{
Postulat wypracowania uniwersalnych kryteriów metanaukowych pozwalających na jednoznaczną ocenę proponowanych hipotez trzeba uznać za oznakę braku realizmu
%\label{ref:RNDdBvQvK1Nfm}(Życiński, 1985, s.~163).
\parencite[][s.~163]{zycinski_teizm_1985}.%
}


Podobnej ocenie Życiński
%\label{ref:RND78djZP5J1x}(Życiński, 1985, s.~161n, 1988b, s.~140nn, 2013, s.~247nn)
\parencites*[][s.~161n]{zycinski_teizm_1985}[][s.~140nn]{zycinski_structure_1988}[][s.~247nn]{zycinski_struktura_2013_liana} %
 poddaje anty-Feyerabendowskie tezy Poppera wyrażone w~\textit{Normal science and its dangers} 
%\label{ref:RNDwlMisyXmlT}(1970)
\parencite*[][]{popper_normal_1970}, %
 uznając je za wyraz nadmiernego optymizmu epistemologicznego. Koło przedzałożeniowe tkwiące w~schematach pojęciowych nie jest więzieniem Pickwickowskim, jak tego chciał Popper. Nie jest tak, że naukowiec może w~dowolnej chwili przełamać swój schemat pojęciowo-metodologiczny i~wybrać inny, który zawsze będzie ,,lepszy i~wygodniejszy''. Tego typu ocena schematu pojęciowego możliwa jest dopiero \textit{ex post}, gdy już istnieją lepsze rozwiązania alternatywne\footnote{Ze słów Życińskiego można wyprowadzić ironiczną konkluzję na temat Poppera i~jego poglądów na temat kategorii ‘schematu pojęciowego', wyrażających się w~tytule innego jego artykułu i~książki (1994): \textit{The Myth of the Framework} (\textit{Mit schematu pojęciowego}). Artykuł, od którego cały zbiór czerpie swą nazwę, ukazał się znacznie wcześniej, w~1976 roku 
%\label{ref:RNDM57srNyQkh}(zob. Popper, 1994, s.~33).
\parencite[zob.][s.~33]{popper_myth_1994}. %
 W~perspektywie nowej koncepcji racjonalności Życińskiego teza Poppera o~mitologicznym charakterze idei schematu pojęciowego sama okazuje się pozostałością myślenia mitologicznego i~magicznego.}.

Źródłem wspomnianej różnicy tych dwóch typów niedookreśloności epistemologicznej jest w~przekonaniu Życińskiego odmienny charakter referencyjny pojęć i~praw w~naukach przyrodniczych i~w~filo\-zofii. Prawa przyrody odnoszone są do ,,jednoznacznie określonego'' zbioru obiektów, tymczasem denotacje twierdzeń filozoficznych, w~tym metodologicznych, są znacznie bardziej rozmyte
%\label{ref:RNDjJnQHEYsUt}(Życiński, 1985, s.~163).
\parencite[][s.~163]{zycinski_teizm_1985}. %
 Fakty przyrodnicze, pomimo swego teoretycznego obciążenia, posiadają łatwo identyfikowalne desygnaty pozajęzykowe\footnote{Argument ten Życiński przejmuje od Newtona-Smitha 
%\label{ref:RNDdsin7xIJJx}(1981, s.~179),
\parencite*[][s.~179]{newton-smith_rationality_1981}, %
 zob. 
%\label{ref:RND78rndPC2SV}(Życiński, 1996, s.~255, 2015, s.~346).
\parencites[][s.~255]{zycinski_elementy_1996}[][s.~346]{zycinski_elementy_2015}.%
}. W~przypadku faktów metanaukowych sytuacja jest zgoła odmienna. Przykładowo, jak pisze, pomimo wykazania bezpodstawności stosowania idei radykalnego empiryzmu w~fizyce teoretycznej, nadal usiłuje się tę ideę stosować w~socjologii czy psychologii 
%\label{ref:RNDPoo2yXmWeC}(Życiński, 1985, s.~163).
\parencite[][s.~163]{zycinski_teizm_1985}. %
 Sugestia Życińskiego jest jasna. Fakty metanaukowe w~znacznie większym stopniu niż fakty przyrodnicze są teoretycznymi konstrukcjami, gdyż ich desygnatami są językowe zachowania naukowców, a~nie przedmioty empiryczne. Rola przedzałożeń w~filozofii jest zdecydowanie większa niż w~naukach przyrodniczych, a~możliwość powszechnej zgody zdecydowanie mniejsza.

Szczególnym przypadkiem pluralizmu metateoretycznego jest pluralizm koncepcji racjonalności, także metodologicznych, względnie naukowych\footnote{Na temat typów racjonalności według Życińskiego zob. niżej punkt 5., zwł. przypis 63. Należy wszakże odróżnić u~Życińskiego wielość typów racjonalności od wielości zmieniających się i~ewoluujących w~czasie koncepcji racjonalności.}. Kolistość w~uzasadnianiu presupozycji wyznaczających tradycje badawcze oraz uznanie pluralizmu tychże tradycji sprawia, że nie jest możliwe zbudowanie jednej uniwersalnej teorii racjonalności nauk przyrodniczych
%\label{ref:RNDoa7esAZgs7}(Życiński, 1985, s.~206.226).
\parencite[][s.~206.226]{zycinski_teizm_1985}. %
 Analizy z~zakresu historii nauki pokazują ,,czasowe zrelatywizowanie ocen racjonalności'' do zmieniających się porewolucyjnych paradygmatów naukowych 
%\label{ref:RNDWqoSSTtmg6}(Życiński, 1985, s.~196n).
\parencite[][s.~196n]{zycinski_teizm_1985}. %
 Najbardziej wymownym potwierdzeniem takiego zrelatywizowania są fakty z~historii nauki wskazujące na zmianę oceny jakiejś idei z~\textit{szokującej} na w~pełni \textit{naturalną} i~\textit{racjonalną}. Przykładowo, antynomie w~teorii mnogości szokowały na początku dwudziestego wieku Fregego, Russella czy Grellinga, a~już kilkanaście lat później dzięki teorii typów i~koncepcji metajęzyka straciły swój problematyczny charakter. Analogicznym przykładem jest zmiana, jaka w~XX wieku zaszła w~ocenie metodologicznego statusu faktu wewnętrznej sprzeczności jakiejś teorii czy faktu występowania anomalii teoretycznych\footnote{Zob. wyżej, przypis nr 13%
. Szerzej na temat metanaukowego znaczenia terminu ‘szok' zob.
%\label{ref:RNDsEjKV3IP8W}(Liana, 2019, s.~163n.179n).
\parencite[][s.~163n.179n]{liana_nauka_2019_liana}.%
}.

Analizy historyczno-naukowe nie pozostawiają zatem dla racjonalisty pola manewru. Teorie racjonalności są historycznie zmienne, podobnie jak zmienne są teorie przyrodnicze. Co więcej, zmiany na poziomie teoretycznym są ściśle powiązane ze zmianami na poziomie metateoretycznym. Próba budowania jednej standardowej teorii racjonalności wbrew tym metanaukowym faktom byłaby przejawem dogmatycznej obrony starej, nierealistycznej i~nazbyt wyidealizowanej teorii racjonalności:

\myquote{
Teorie racjonalności zmieniają się ze zmianą paradygmatów następującą po rewolucjach naukowych. [...] W~tej sytuacji próby absolutyzowania pojęcia racjonalności oraz dążenie do dychotomicznego podziału interpretacji na racjonalne i~irracjonalne jest już wyrazem pewnej filozofii bronionej w~sposób dogmatyczny, lecz falsyfikowanej przez faktyczną naukę
%\label{ref:RNDzgkWErEQeB}(Życiński, 1985, s.~206).
\parencite[][s.~206]{zycinski_teizm_1985}.%
}


Nowa koncepcja racjonalności, jaką proponuje Życiński, dokonuje podobnie ,,szokujących'' zmian w~standardach racjonalności. Zmienia ona status metodologiczny kolistości w~uzasadnianiu przedzałożeń, status czynników pozaracjonalnych i~innych elementów tradycyjne uznawanych za irracjonalne. Swoją koncepcję racjonalności Życiński nazywa za Newtonem-Smithem ‘racjonalizmem umiarkowanym'
%\label{ref:RNDs2Qn6JUpra}(Życiński, 1985, s.~205)
\parencite[][s.~205]{zycinski_teizm_1985}%
\footnote{Nazwa polska jest odpowiednikiem angielskiego `\textit{temperate rationalism}' 
%\label{ref:RNDbeRgfw7YUQ}(Newton-Smith, 1981, s.~266–273).
\parencite[][s.~266–273]{newton-smith_rationality_1981}. %
 Życiński rozumie jednak racjonalizm umiarkowany nieco odmiennie od Newtona-Smitha. Na ten temat zob. niżej punkt 6.}.

Pluralizm teoretyczny i~metateoretyczny są więc faktami metanaukowymi, a~ich wyjaśnieniem jest epistemologiczna zasada niedookreśloności. Nieuchronność pluralizmu nie musi jednak prowadzić do anarchizmu metodologicznego ani na gruncie nauk przyrodniczych, ani na gruncie metanauki
%\label{ref:RNDNgZeDdqS9o}(zob. np. Życiński, 1983, s.~183nn, 1985, s.~161n.203nn)
\parencites[zob. np.][s.~183nn]{zycinski_jezyk_1983}[][s.~161n.203nn]{zycinski_teizm_1985}%
\footnote{Szczególnie znaczące w~tym kontekście są jego analizy anarchistycznej koncepcji Feyerabenda, w~których podkreśla bezzasadność uznania tej koncepcji za przejaw skrajnego irracjonalizmu. Styl Feyerabenda ma swe pozytywne znaczenie w~postaci przestrogi ,,przed bezkrytycznym kultem jakichkolwiek metod badawczych'' 
%\label{ref:RND3bt7spuBFy}(zob. Życiński, 1996, s.~217–227, zwł. s.~220, 2015, s.~295–308, zwł. s.~299n).
\parencites[zob.][s.~217–227, zwł. s.~220]{zycinski_elementy_1996}[][s.~295–308, zwł. s.~299n]{zycinski_elementy_2015}.%
}. Przeciwnie. Epistemologiczna zasada niedookreśloności stwierdza, że pluralizm teoretyczny i~metateoretyczny jest stanem \textit{naturalnym} ludzkiego poznania. Stan ten wynika z~nieuniknionej obecności przedzałożeń i~z ich metodologicznej kolistości. Realistyczna koncepcja metanaukowa powinna ten fakt uznać i~wyjaśnić. Jego negacja w~imię racjonalności byłaby co najwyżej dogmatyczną próbą obrony tradycyjnego stanowiska, polegającą na eliminacji \textit{ad hoc} niewygodnych faktów.

\subsection{c) Konieczność podmiotowego \textit{commitment} w~wyborze tradycji badawczej}

Metaepistemologiczna zasada niedookreśloności i~fakt pluralizmu metateoretycznego stanowią wystarczające przesłanki do rozwiązania problemu czynników zewnętrznych. Wystarczy jedynie poddać je wnikliwej analizie logicznej.

Niemożliwość ostatecznego uzasadnienia podstawowych przedzałożeń tradycji badawczych i~wynikający stąd nieunikniony pluralizm równoważnych ,,obserwacyjnie'' tradycji badawczych sprawia, że wybór tradycji badawczych i~odpowiadających im przedzałożeń nie może być nigdy do końca obiektywny i~racjonalny w~tradycyjnym rozumieniu tych terminów. Stojąc przed wyborem równoważnych tradycji, naukowiec musi podjąć \textit{decyzję}, w~której element konwencjonalno-subiektywny odgrywa istotną rolę:

\myquote{
W~sytuacji takiej wybór podstawowych złożeń w~teorii wiedzy implikuje z~konieczności pewien element konwencjonalno-subiektywny, w~którym występują dwie składowe: (1) obiektywne trudności z~rozstrzygnięciem określonych kwestii [i] (2) zasadnicza niemożność uprawiania wiedzy z~pozycji czysto neutralnego obserwatora, wolnego od wcześniejszych założeń, czy podmiotowych uprzedzeń
%\label{ref:RNDq1xMrOTlNY}(Życiński, 1985, s.~157n)
\parencite[][s.~157n]{zycinski_teizm_1985}%
\footnote{Stwierdzenie to inspiruje Życińskiego do wypracowania epistemologicznej zasady niepewności. Jako że nie stanowi ona elementu koniecznego w~argumencie Życińskiego, zostanie omówiona poniżej w~punkcie 3.}.
}

\textit{Zasadnicza niemożność} pozbycia się elementu subiektywnego z~nauki przekreśla tradycyjne wyobrażenia o~nauce jako dziele czystego rozumu. Wybór pomiędzy różnymi tradycjami badawczymi i~ich przedzałożeniami musi zostać uzupełniony o~akt podmiotowego \textit{związania} lub \textit{commitment}
%\label{ref:RND27y96wW0K5}(Życiński, 1985, s.~160.164nn.182, 1996, s.~186n.191, 2015, s.~253n.258)
\parencites[][s.~160]{zycinski_teizm_1985}[][s.~186n.191]{zycinski_elementy_1996}[][s.~253n.258]{zycinski_elementy_2015}%
\footnote{Życiński używa też określania ‘osobowe \textit{commitment}' pochodzącego zapewne od Polanyia 
%\label{ref:RNDCd2EdbIWIc}(1962),
\parencite*[][]{polanyi_personal_1962}, %
 zob. np. 
%\label{ref:RNDX0FNqDQW8z}(Życiński, 1985, s.~79.125.165).
\parencite[][s.~79.125.165]{zycinski_teizm_1985}. %
 Podobnie w
%\label{ref:RNDmC7GrxFJcP}(Życiński, 1988b, s.~137.144)
\parencite[][s.~137.144]{zycinski_structure_1988} %
 stosuje wyrażenia ‘\textit{subjective commitment}' i~‘\textit{personal commitment}', jak też zamiennie używa określenia ‘\textit{attachment}' (tamże, s.~143). W~pracach polskich Życiński stosuje termin angielski ‘\textit{commitment}' zamiennie z~polskimi odpowiednikami. Różne polskie znaczenia tego terminu Życiński przedstawia w
%\label{ref:RNDViGm1dQpm9}(Życiński, 1996, s.~191, 2015, s.~259n)
\parencites[][s.~191]{zycinski_elementy_1996}[][s.~259n]{zycinski_elementy_2015} %
 wskazując przy tym na T. Kuhna, W.V.O. Quine'a i~M. Polanyia jako tych, których prace przyczyniły się w~sposób szczególny do rozwoju badań nad rolą związania z~paradygmatem. Ponieważ angielskie \textit{commitment} stało się terminem technicznym, znacznie bardziej jednoznacznym od polskich odpowiedników -- np. \textit{związanie, zaangażowanie, przylgnięcie, zdeklarowanie} -- dlatego w~moim tekście pozostaję zasadniczo przy terminie ‘\textit{commitment}'.}:

\myquote{
Wypracowanie [zasad obiektywnego uzasadniania w~nauce] jest możliwe, jeśli dany badacz związał się z~określoną tradycją badawczą. Sam fakt wyboru jednej z~wielu tradycji, ma jednak znowu uwarunkowania podmiotowe, gdyż nie istnieją obiektywne metazasady pozwalające jednoznacznie oceniać konkurencyjne tradycje. Ten element podmiotowego związania z~tradycją badawczą porównywany jest do związania z~określoną tradycją religijną
%\label{ref:RNDTsOZn4Dp4U}(Życiński, 1985, s.~164).
\parencite[][s.~164]{zycinski_teizm_1985}.%
}


W~analogicznym ujęciu tej samej konkluzji \textit{commitment} jest konieczne, by przerwać nieskończony regres w~przyjmowaniu kolejnych konwencji znaczeniowych w~podstawach teorii racjonalności naukowej:

\myquote{
Rozstrzygnięcie, które z~przyjętych konwencji są bardziej racjonalne, prowadzić musi do przyjęcia nowych konwencji przy ocenie racjonalności twierdzeń. Brak jednolitych metakryteriów racjonalności sprawia [...] iż przy określaniu podstawowych założeń musi być uwikłany element osobowego wyboru -- \textit{commitment} -- który [...] może być \textit{jedynie częściowo uzasadniony}
%\label{ref:RNDyKpH3lBHEJ}(Życiński, 1985, s.~79, podkreślenia oryginalne).
\parencite[][s.~79, podkreślenia oryginalne]{zycinski_teizm_1985}.%
}


Podmiotowy charakter \textit{związania} i~\textit{commitment} podkreśla równoważne określenie go mianem ‘aktu wiary'
%\label{ref:RNDzsqOJxt7Tg}(Życiński, 1985, s.~186)
\parencite[][s.~186]{zycinski_teizm_1985}%
\footnote{O~decydującej roli wiary (\textit{faith}) przy podejmowaniu decyzji przez naukowców mówi Kuhn 
%\label{ref:RNDpadC4alIqX}(1970, s.~158, zob. 2001, s.~274).
\parencites*[][s.~158]{kuhn_structure_1970_liana}[zob.][s.~274]{kuhn_struktura_2001}.%
}. Wprawdzie termin ‘wiara' Życiński rozumie zasadniczo w~sensie angielskiego \textit{belief} (przekonanie), niemniej nawiązując do Barboura 
%\label{ref:RNDp93xgslbs9}(1984),
\parencite*[][]{barbour_mity_1984} %
 wskazuje na podobieństwo pomiędzy metanaukowym \textit{commitment} a~związaniem z~tradycją religijną. W~akcie \textit{commitment} elementy racjonalne ustępują w~dużej mierze miejsca elementom wolitywnym. W~związku z~tym, jak zauważa, mówi się raczej o~,,wyborze'' metodologii naukowej i~o ,,woli wierzenia'' w~określone twierdzenia niż o~ich uzasadnieniu 
%\label{ref:RNDTYDM8ZK1O1}(Życiński, 1985, s.~164).
\parencite[][s.~164]{zycinski_teizm_1985}. %
 \textit{Wiara} nie musi jednak być i~nie jest absolutnym przeciwieństwem \textit{wiedzy}\footnote{Życiński 
%\label{ref:RNDY4bCdwAdhc}(1985, s.~141–156)
\parencite*[][s.~141–156]{zycinski_teizm_1985} %
 poddaje szczegółowej analizie relacje semantyczne pomiędzy operatorami ‘wiedzieć' i~‘wierzyć'. Wykazuje przy tym ich złożone zależności logiczne.} . Obecność podmiotowego aktu wiary w~\textit{commitment} nie implikuje z~konieczności irracjonalnej arbitralności tego aktu. Przeciwnie, obecność ta ma charakter jak najbardziej pro-racjonalny, gdyż jest warunkiem koniecznym racjonalności w~ogóle.

Idea \textit{bycia warunkiem koniecznym racjonalności} stanowi istotny element poszukiwanego rozwiązania problemu czynnika zewnętrznego w~nauce. Brak decyzji wyboru w~obliczu pluralizmu tradycji i~kolistości przedzałożeń byłby kapitulacją rozumu na rzecz sceptycyzmu. Rozum może obronić się wyłącznie w~akcie podmiotowej i~częściowo (koliście) uzasadnionej wiary:

\myquote{
Związanie to nie jest -- w~przypadku ogólnym -- przejawem woluntaryzmu metodologicznego, lecz \textit{następstwem konieczności, w~stosunku do której główną kontrpropozycją byłby sceptycyzm}
%\label{ref:RND2tvUgP4RWN}(Życiński, 1985, s.~165)
\parencite[][s.~165]{zycinski_teizm_1985}%
\footnote{Życiński 
%\label{ref:RNDXEgUMQdMH5}(1988b, s.~143, por. 2013, s.~253)
\parencites*[][s.~143]{zycinski_structure_1988}[zob.][s.~253]{zycinski_struktura_2013_liana} %
 pisze podobnie: ,,W rzeczywistości jednak takie przywiązanie zasadniczo nie jest wyrazem metodologicznego woluntaryzmu, lecz konieczności. Alternatywną dla niego byłby sceptycyzm. Aby go uniknąć, konieczne jest przyjęcie zbioru twierdzeń umożliwiających zarówno prowadzenie dyskursu naukowego, jak i~określenie typu tego dyskursu''. Tego typu argument znajduje się również w
%\label{ref:RNDvseT1eJNu1}(Życiński, 1996, s.~183n, 2015, s.~250).
\parencites[][s.~183n]{zycinski_elementy_1996}[][s.~250]{zycinski_elementy_2015}.%
}.
}


Szczególnie interesujące w~tym tekście wydaje się pozornie paradoksalne odwrócenie logicznej zależności pomiędzy ideą kolistych presupozycji a~ideą sceptycyzmu. W~tradycyjnym ujęciu racjonalności brak trwałych i~ostatecznych podstaw interpretowany był jako argument za sceptycyzmem (irracjonalizmem). Życiński odwraca sytuację i~atakuje tradycyjny sceptycyzm jego własną bronią. Skoro nie istnieje żaden ostateczny, niewzruszony i~niezmienny układ odniesienia dla naszych wypowiedzi i~teorii naukowych -- a~to jest faktem metanaukowym -- zatem konieczne i~zarazem wystarczające jest istnienie przynajmniej tymczasowych, zmiennych historycznie, quasi-uniwersalnych ram paradygmatycznych. Bez nich cały dyskurs rozpłynąłby się w~nieskończonej liczbie całkowicie niewspółmiernych układów odniesienia. W~przypadku braku jakichkolwiek, choćby tymczasowych, i~choćby tylko częściowo powszechnie obowiązujących wzorców, niemożliwy byłby jakikolwiek dyskurs intersubiektywny, prowadząc do skrajnego solipsyzmu i~do tylu koncepcji nauki, ile byłoby naukowców
%\label{ref:RNDSMxHa8t6nI}(Życiński, 1985, s.~185)
\parencite[][s.~185]{zycinski_teizm_1985}%
\footnote{,,Jako niekwestionowalna subiektywnie jawi się jedynie rzeczywistość naszych podmiotowych doznań, której treść dałaby co najwyżej podstawę do wypracowania nowej monadologii. W~rzeczywistości takiej mogłoby istnieć tyle nauk, ilu jest uczonych, gdyż nie istnieje obiektywny gwarant identyczności recepcji świata''. Na temat warunków możliwości dyskursu zob. 
%\label{ref:RND7DjcDdo8TJ}(Życiński, 1985, s.~194).
\parencite[][s.~194]{zycinski_teizm_1985}.%
}. Dyskurs intersubiektywny jest jednak faktem. Uznanie faktu dyskursu intersubiektywnego w~powiązaniu z~uznaniem faktu braku ostatecznego układu odniesienia tego dyskursu skutkuje koniecznością uznania istnienia tymczasowych i~jedynie częściowo uzasadnionych paradygmatycznych ram dyskursu naukowego.

Szczególnym przypadkiem faktu intersubiektywnego dyskursu jest dla Życińskiego współczesna nauka przyrodnicza posługująca się matematyką w~opisie rzeczywistości. Pozwala mu to na jeszcze inne sformułowanie tego samego argumentu za pro-racjonalnym charakterem podmiotowego \textit{commitment} odwołujące się do logicznej możliwości faktu nauki:

\myquote{
Gdyby w~praktyce badawczej uczeni chcieli uwzględniać całą złożoność istniejących stanowisk metaepistemologicznych i~metodologicznych, nauka przestałaby w~ogóle istnieć, zaś dominującą filozofią byłby sceptycyzm. Istnienie i~rozwój nauki jest więc możliwe dzięki temu, iż w~pragmatyce badań braki jednoznacznych argumentów składających się na kontekst uzasadnienia uzupełniane są przez czynniki psycho-społeczne tworzące kontekst akceptacji
%\label{ref:RNDpNHSRz3QtD}(Życiński, 1985, s.~228).
\parencite[][s.~228]{zycinski_teizm_1985}.%
}


Próba absolutnego doprecyzowania przez naukowca argumentów składających się na jego kontekst uzasadnienia unicestwiłaby możliwość intersubiektywnej nauki w~ogóle. W~rzeczywistej nauce wybory naukowców nie są, gdyż nie mogą być, wystarczająco dookreślone ani formalno-logicznie, ani semantycznie. Konieczne jest odwołanie się do pragmatyki uwzględniające kontekstualność podmiotu. Jedyną alternatywą dla takiej pragmatyki byłby solipsystyczny sceptycyzm (zob. niżej punkt 5.).

Idea warunku koniecznego pokazuje pro-racjonalny charakter czynnika subiektywnego w~nauce. Nie neutralizuje jednak argumentu sceptycznego. Sceptyk może bowiem powiedzieć, że podmiotowe \textit{commitment} może być warunkiem koniecznym tak a~nie inaczej rozumianej nauki, czy nawet dyskursu w~ogóle, ale to w~żaden sposób nie podważa jego twierdzenia, że tym, co w~pełni determinuje nasze rozumienie nauki, są czynniki zewnętrzne. Teza o~konieczności \textit{commitment} zdaje się~jedynie potwierdzać jego stanowisko eksternalistyczne. Argument z~warunku koniecznego musi zatem zostać uzupełniony o~argument z~równoważności obserwacyjnej alternatywnych tradycji metanaukowych: sceptycznej i~racjonalistycznej:

\myquote{
Czynnikiem usprawiedliwiającym to związanie jest zarówno to, że żadna z~konkurencyjnych propozycji nie może być uzasadniona w~sposób definitywny, jak i~to, iż można przytaczać racjonalne argumenty na rzecz wybranej interpretacji. W~sytuacji gdy dążenie do absolutnej pewności okazuje się praktycznie niemożliwe do zrealizowania, wyrazem racjonalizmu jest akceptacja dyskursu, w~którym proponowane tłumaczenia wydają się bardziej uzasadnione niż alternatywne propozycje
%\label{ref:RNDA6pO9ozw7D}(Życiński, 1985, s.~165).
\parencite[][s.~165]{zycinski_teizm_1985}.%
}


Spróbujmy wydobyć z~tekstu Życińskiego racje zawarte w~nim \textit{implicite}, wykorzystując metanaukowe kategorie proponowane przez samego Życińskiego.

Skoro jest powszechnie uznaną prawdą, że nie ma poznania bezzałożeniowego, to także sceptycyzm jest wynikiem przyjęcia \textit{implicite} określonych przedzałożeń sceptycznych. Z~konieczności zatem także deklarowany sceptycyzm jest wynikiem podmiotowego \textit{commitment} do określonej koncepcji racjonalności. Zgodnie z~zasadą niedookreśloności metanaukowy fakt przedzałożeniowego koła nie dookreśla wystarczająco (to znaczy: nie \textit{determinuje}) żadnej interpretacji metateoretycznej, ani racjonalistycznej, ani sceptycznej\footnote{Ani fakt obecności w~nauce czynników pozaracjonalnych, ani fakt niedowodliwych presupozycji nie implikuje konieczności wyjaśnienia eksternalistycznego, chyba że już wcześniej przyjęło się dodatkowe, eksternalistyczne przedzałożenia na temat racjonalności. Por.
%\label{ref:RND178P6mRxlg}(Życiński, 1996, s.~184, 2015, s.~250);
\parencites[][s.~184]{zycinski_elementy_1996}[][s.~250]{zycinski_elementy_2015}; %
 o~solipsyzmie Bridgmana zob. 
%\label{ref:RND7pVakaVtUm}(Życiński, 1996, s.~186–188, 2015, s.~253–255).
\parencites[][s.~186–188]{zycinski_elementy_1996}[][s.~253–255]{zycinski_elementy_2015}.%
}. Zatem wybór między tymi interpretacjami kolistości przedzałożeń musi być powodowany podmiotowym \textit{commitment} z~odmiennymi przedzałożeniami na temat racjonalności.

Mając zatem do wyboru między obserwacyjnie równoważnymi interpretacjami metanaukowego faktu presupozycyjnego koła, sceptyczną i~racjonalistyczną, racjonalista chcąc pozostać racjonalistą musi wybrać interpretację racjonalistyczną. Ta jednak z~konieczności musi uwzględniać nieusuwalny element \textit{podmiotowego commitment}. Nie jest bowiem możliwy logicznie wybór tradycji w~oparciu o~same przesłanki racjonalne, bez podmiotowego \textit{commitment}, skoro zostało przyjęte, że uzasadnienie tych przesłanek nie może być ostateczne, lecz jest swoiście koliste. W~przeciwnym wypadku racjonalista popada w~wewnętrzną sprzeczność, a~jego podmiotowe \textit{commitment}, które do niej prowadzi, nosi wszelkie znamiona sceptycyzmu.

Nieco inaczej wygląda sytuacja zadeklarowanego eksternalisty i~sceptyka. Zgodnie z~racjonalistyczną interpretacją podmiotowego \textit{commitment}, także sceptyczne \textit{commitment} z~tradycją sceptyczną jest w~pewnym podstawowym sensie racjonalne, gdyż w~ten właśnie sposób działa ludzki rozum, zarówno rozum racjonalisty, jak i~rozum sceptyka. Ale jest to wyłącznie racjonalność na poziomie nieuświadomionych zachowań. Tymczasem racjonaliście chodzi o~obronę racjonalności na poziomie teoretycznych artykulacji i~świadomych wyborów. Na tym poziomie eksternalista i~sceptyk w~perspektywie nowego racjonalizmu jedynie artykułuje i~potwierdza swoje własne \textit{commitment} do eksternalistycznych i~sceptycznych przedzałożeń.

Równoważność obserwacyjna sprawia, że z~logicznego punktu widzenia wybór między racjonalizmem a~sceptycyzmem jest kwestią podmiotowego \textit{commitment}: racjonalista może i~powinien wybrać racjonalizm, sceptyk może wybrać i~zapewne wybierze sceptycyzm. Swego czasu Poincaré
%\label{ref:RNDjmQkd1iyEO}(1905)
\parencite*[][]{poincare_valeur_1905} %
 mówił, że deterministą jest się z~własnej woli 
%\label{ref:RNDvy9Ff5ru3q}(Szumilewicz-Lachman, 1978, s.~276.278).
\parencite[][s.~276.278]{szumilewicz-lachman_poincare_1978}. %
 Podobną zasadę zdaje się wyznawać Życiński. Racjonalistą -- podobnie jak eksternalistą i~sceptykiem -- jest się z~własnego wyboru. Przedzałożenie racjonalizmu wyznacza tradycję racjonalistyczną. Jego wybór musi ostatecznie być wynikiem podmiotowego \textit{commitment} do idei racjonalności.

Także uzasadnienie tego rozwiązania, wyprowadzane z~faktu nieeliminowalności przedzałożeniowego koła i~pluralizmu metateoretycznego, nie może mieć charakteru ostatecznego rozstrzygnięcia, a~jedynie częściowego uzasadnienia. Także musi być koliste. Ze względu na zasadę niedookreśloności nie jest ono możliwe bez uprzedniego podmiotowego związania się z~określonymi wzorcami racjonalności i~bez przyjęcia aksjologicznie uwarunkowanej hierarchii ważności\footnote{Na temat aksjologii zob. niżej punkt 4.} różnych typów racjonalności:

\myquote{
Zwłaszcza [...] wybór określonej koncepcji racjonalności nie może być całkowicie i~obiektywnie uzasadniony, lecz wymaga odwołania się do podmiotowego \textit{commitment}, w~którym podejmuje się decyzję o~wyborze określonego zespołu wskaźników racjonalności lub o~priorytecie pewnego typu racjonalności. Z~tej racji [...] w~dyskusjach o~racjonalności wiedzy dużą rolę odgrywają podmiotowe przedzałożenia określające hierarchię typów racjonalności oraz kryteria ustalania tej hierarchii
%\label{ref:RNDehhUb2RaQI}(Życiński, 1985, s.~206n).
\parencite[][s.~206n]{zycinski_teizm_1985}.%
}

\enlargethispage{-\baselineskip}
Argument Życińskiego prowadzi zatem do podwójnej konkluzji. Po pierwsze, racjonalista, chcąc pozostać racjonalistą, musi zaakceptować element \textit{commitment} w~samych podstawach racjonalności. Po drugie, sceptyczny argument eksternalisty nie ma mocy wiążącej dla racjonalisty.

\section{Epistemologiczna zasada niepewności i~doxalogia}
Życiński wiąże swój argument za koniecznością \textit{commitment} z~istotnym elementem \textit{niepewności}, jaki pojawił się nauce porewolucyjnej, i~z wypracowaną przez siebie zasadą niepewności. Wszakże z~punktu widzenia logicznej rekonstrukcji zasada \textit{niepewności} jest raczej konsekwencją niż przesłanką tego argumentu. Subiektywny charakter \textit{commitment} wyboru leżący u~podstaw nauki i~racjonalności prowadzi do \textit{niepewności}, jaka towarzyszy fundamentalnym wyborom. Konieczność wyboru w~sytuacji niemożliwości ostatecznego uzasadnienia takiego wyboru sprawia, że wybór ten charakteryzuje się logiczną niepewnością. Epistemologiczna zasada niepewności stanowi metateoretyczną próbę zracjonalizowania tej niepewności\footnote{Życiński rozwija swą epistemologiczną zasadę niepewności stosunkowo niezależnie od zasady niedookreśloności. Por.
%\label{ref:RNDWpO2LkCq8r}(Życiński, 1988b, s.~145, 2013, s.~256).
\parencites[][s.~145]{zycinski_structure_1988}[][s.~256]{zycinski_struktura_2013_liana}.%
}.

Niepewność można rozumieć czysto psychologicznie jako pewien stan podmiotowego umysłu. Tak też rozumiał ją Bridgman
%\label{ref:RNDQxteQ1VYVf}(1950),
\parencite*[][]{bridgman_reflections_1950}, %
 którego pierwsza zasada dynamiki myślowej stała się inspiracją dla Życińskiego, by zaproponować nową epistemologiczną zasadę niepewności 
%\label{ref:RND6kczIlrP0P}(Życiński, 1985, s.~158–161, 1988b, s.~137–139, 2013, s.~243–247)
\parencites[][s.~158–161]{zycinski_teizm_1985}[][s.~137–139]{zycinski_structure_1988}[][s.~243–247]{zycinski_struktura_2013_liana}%
\footnote{Życiński 
%\label{ref:RND2Urh3qBpQy}(1985, s.~159, przypis 78)
\parencite*[][s.~159 przypis 78]{zycinski_teizm_1985} %
 zaznacza, że polska nazwa tej zasady pochodzi od niego.}. Życiński odrzuca jednak psychologiczne rozumienie niepewności, uznając je za nieinteresujące dla filozofa nauki i~prowadzące zbyt łatwo do idei poznawczego solipsyzmu, jak w~przypadku Bridgmana. Niepewność w~rozumieniu Życińskiego ma charakter epistemologiczny i~metateoretyczny, a~zatem do pewnego stopnia obiektywny. Podobnie jak inne podmiotowe kategorie metateoretyczne posiada ona charakter pro-racjonalny i~pro-intersubiektywny, a~nie pro-sceptyczny i~pro-solipsystyczny. Nowy umiarkowany racjonalizm przejmuje kategorie sceptyczne i~adaptuje je do swych przedzałożeń.

\enlargethispage{-\baselineskip}
Jako kategoria pro-racjonalna epistemologiczna kategoria ‘niepewności' ma swe źródła w~twierdzeniach limitacyjnych metalogiki i~w zasadzie nieoznaczoności Heisenberga i~w innych zasadach leżących u~podstaw mechaniki kwantowej\footnote{Na temat zasady Heisenberga zob.
%\label{ref:RNDXvTifl6z0D}(Życiński, 1985, s.~158n, 1988b, s.~138nn, 2013, s.~246n).
\parencites[][s.~158n]{zycinski_teizm_1985}[][s.~138nn]{zycinski_structure_1988}[][s.~246n]{zycinski_struktura_2013_liana}. %
 Na temat twierdzeń limitacyjnych i~związanej z~nimi niepewności zob. 
%\label{ref:RNDDYUol0OeO1}(Życiński, 1985, s.~158, 1988b, s.~18–46, b, s.~102–113, 2013, s.~183–201, 1996, s.~262–276, 2015, s.~355–374)
\parencites[][s.~158]{zycinski_teizm_1985}[][s.~18–46]{zycinski_teizm_1988}[][s.~102–113]{zycinski_structure_1988}[][s.~183–201]{zycinski_struktura_2013_liana}[][s.~262–276]{zycinski_elementy_1996}[][s.~355–374]{zycinski_elementy_2015}.
}.
%Odwołania do tej samej czy innej pracy?
W~obu tych przypadkach mamy do czynienia z~niepewnością uwarunkowaną obiektywnie. Jej źródłem jest nieunikniona nieokreśloność lub niedookreślenie pewnych sytuacji badawczych. Twierdzenia limitacyjne pokazują niemożliwość jednoczesnego określenia niezupełności i~niesprzeczności bogatych systemów logicznych\footnote{Inny typ niepewności eksploatowany metanaukowo przez Życińskiego wiąże się z~górnym twierdzeniem Skolema–Löwenheima i~z tak zwaną przezeń ‘skolemizacją' języka. Wielość niezamierzonych modeli semantycznych, często drastycznie różnych od zamierzonych, jest obiektywną cechą każdego wystarczająco bogatego języka, w~tym języka nauki. Na temat tego twierdzenia i~jego konsekwencji filozoficznych zob.
%\label{ref:RNDlJXIc3Y13s}(Życiński, 1988a, s.~22–46, 1996, s.~270–276, 2015, s.~369–373).
\parencites[][s.~22–46]{zycinski_structure_1988}[][s.~270–276]{zycinski_elementy_1996}[][s.~369–373]{zycinski_elementy_2015}. %
 Życiński porównuje twierdzenia limitacyjne do zasady nieoznaczoności Heisenberga, a~zatem pośrednio także do epistemologicznej zasady niepewności. Życiński 
%\label{ref:RNDBkyipNpImm}(1988b, s.~109, 2013, s.~194)
\parencites*[][s.~109]{zycinski_structure_1988}[][s.~194]{zycinski_struktura_2013_liana} %
 pisze: ,,Gödel, Skolem i~Church odegrali we współczesnej matematyce tę samą rolę co Heisenberg w~mechanice kwantowej''. Na ten temat zob. też 
%\label{ref:RNDDkBzlMrnvK}(Życiński, 1996, s.~273, 2015, s.~371).
\parencites[][s.~273]{zycinski_elementy_1996}[][s.~371]{zycinski_elementy_2015}.%
}. Zasada nieoznaczoności Hei\-senberga wskazuje z~kolei na obiektywną niemożliwość dowolnie dokładnej charakterystyki zdarzeń na poziomie mikroświata i~na swoiste splątanie lub komplementarność mierzonych wielkości.

\enlargethispage{-\baselineskip}
Nie-psychologiczne rozumienie epistemologicznej zasady niepewności wiąże się też z~genezą samej nazwy. Pochodzi ona z~fizyki. Nazwa ‘\textit{principle of uncertainty}' przyjęła się w~języku angielskim jako nazwa na zasadę nieoznaczoności Heisenberga\footnote{Innym, ale znacznie rzadziej stosowanym określeniem jest ‘\textit{principle of indeterminacy}'. Zob. np. hasło ,,Uncertainty principle'' w~\textit{Encyclopaedia Britannica}
%\label{ref:RNDdiEIbVd7hr}(2020).
\parencite*[][]{noauthor_uncertainty_2020}.%
}. Sam Heisenberg w~swym oryginalnym artykule 
%\label{ref:RNDFcP31f31lK}(1927, s.~179nn)
\parencite*[][s.~179nn]{heisenberg_uber_1927} %
 nie mówi o~zasadzie, a~jedynie o~\textit{nieokreśloności} (\textit{Unbestimmtheit}) pewnych sytuacji pomiaru kwantowego\footnote{Dokładniej mówi on o~‘nieokreśloności' (\textit{Unbestimmtheit}) występującej w~eksperymentach, w~których pragnie się podać jednoczesne określenie (\textit{simultane Bestimmung}) dwóch kanonicznie powiązanych wielkości. Stopień tej nieokreśloności wyrażony jest z~kolei w~podanej przez niego matematycznej relacji (\textit{Relation}) lub równaniu (\textit{Gleichung}) 
%\label{ref:RNDaBKElykLvm}(Heisenberg, 1927, s.~179.181).
\parencite[][s.~179.181]{heisenberg_uber_1927}.%
}. Angielskie określenie ‘\textit{principle of uncertainty}' stanowi zatem swoistą interpretację epistemologiczną twierdzenia Heisenberga. Życiński wykorzystuje ten fakt językowy, by nazwie zaczerpniętej z~fizyki nadać sens epistemologiczny. W~angielskiej \textit{The Structure of the Metascientific Revolution} 
%\label{ref:RNDAnieyw9FmQ}(1988b)
\parencite*[][]{zycinski_structure_1988} %
 konsekwentnie stosuje określenie ‘\textit{principle of uncertainty}' zarówno w~odniesieniu do fizycznej zasady Heisenberga, jak i~do jej epistemologicznego analogonu. W~książkach polskich brak takiego zdecydowania. Fizyczną zasadę Heisenberga nazywa w~pierwszym tomie \textit{Teizmu} zasadą \textit{nieokreśloności} 
%\label{ref:RNDtOel8hs11X}(Życiński, 1985, s.~118.159),
\parencite[][s.~118.159]{zycinski_teizm_1985}, %
 w~tomie drugim raz zasadą \textit{nieoznaczoności} 
%\label{ref:RNDuJAnlbY1tw}(Życiński, 1988a, s.~24),
\parencite[][s.~24]{zycinski_teizm_1988}, %
 innym razem zasadą \textit{nieokreśloności} 
%\label{ref:RND6VZU4WeYfr}(Życiński, 1988a, s.~63),
\parencite[][s.~63]{zycinski_teizm_1988}, %
 natomiast w~\textit{Elementach filozofii nauki} 
%\label{ref:RND6qJI0labnN}(1996, 2015)
\parencites*[][]{zycinski_elementy_1996}[][]{zycinski_elementy_2015} %
 mówi już konsekwentnie o~zasadzie \textit{nieoznaczoności}. Swoją własną zasadę epistemologiczną nazywa równoważnie \textit{epistemologiczną zasadą niepewności} i~\textit{metateoretyczną zasadą nieokreśloności epistemologicznej} 
%\label{ref:RNDN1COxt98uh}(Życiński, 1985, s.~159)
\parencite[][s.~159]{zycinski_teizm_1985}%
\footnote{W~\textit{Elementach filozofii nauki} 
%\label{ref:RND9KadZo7XXP}(1996, 2015)
\parencites*[][]{zycinski_elementy_1996}[][]{zycinski_elementy_2015} %
 Życiński nie wspomina o~niej. W~jej miejsce odwołuje się do natomiast do idei Polanyia \textit{milczącego poznania} i~\textit{wiedzy osobowej}.}.

\enlargethispage{-\baselineskip}
Istnieje zbieżność pomiędzy konsekwencjami filozoficznymi, jakie Heisenberg wyciąga ze swego odkrycia, a~epistemologicznymi wnioskami wyciąganymi z~niego przez Życińskiego. Heisenberg stwierdza, że skoro nie jest możliwe poznanie współczesności (\textit{Gegenwart}) we wszystkich jej możliwych determinantach (\textit{Bestimmungen}), to znaczy to, że każda obserwacja jest ,,jakimś wyborem'' (\textit{eine Auswahl}) spośród mnóstwa możliwości i~że zarazem jest ona ograniczeniem przyszłych możliwości
%\label{ref:RNDJenhHbD3MF}(Heisenberg, 1927, s.~197).
\parencite[][s.~197]{heisenberg_uber_1927}. %
 Życiński interpretuje ją podobnie. Stwierdza 
%\label{ref:RND3FzukxAWAI}(Życiński, 1985, s.~159n),
\parencite[][s.~159n]{zycinski_teizm_1985}, %
 że w~sytuacji obiektywnej i~koniecznej nieokreśloności w~mechanice kwantowej musi wystąpić ,,element subiektywny'' w~postaci ,,podmiotowej decyzji naukowca'', którą z~dwóch kanonicznie powiązanych wielkości badać z~większą dokładnością. Stwierdza też, że element subiektywny łączy się tutaj nierozerwalnie z~elementem obiektywnym i~że analogiczna sytuacja zachodzi w~epistemologii. Właśnie ta sytuacja wyrażana jest za pomocą epistemologicznej zasady niepewności.

\enlargethispage{-\baselineskip}
Życiński podaje kilka różnych ujęć swej epistemologicznej zasady niepewności. W~ujęciu ogólnym mówi ona, że ,,wraz z~rozwojem ludzkiego poznania rozwija się nie tylko jego informacyjna zawartość, lecz także metateoretyczna świadomość jego braków i~ograniczeń uniemożliwiających osiągnięcie pewności przy najbardziej podstawowych założeniach''
%\label{ref:RNDLF01Op8WVM}(Życiński, 1985, s.~159).
\parencite[][s.~159]{zycinski_teizm_1985}. %
 W~ujęciu metajęzykowym mówi ona z~kolei, że ,,dany zbiór twierdzeń nie jest niezależny od innych twierdzeń, lecz pozostaje powiązany z~założeniami metafizycznymi mającymi istotny wpływ na formę i~treść akceptowanych twierdzeń. [I że] twierdzenia obiektywne i~założenia metafizyczne są ze sobą sprzężone na kształt pewnych parametrów mikroświata'' 
%\label{ref:RNDBTX7mN3OPz}(Życiński, 1988b, s.~139, 2013, s.~246).
\parencites[][s.~139]{zycinski_structure_1988}[][s.~246]{zycinski_struktura_2013_liana}. %
 W~jeszcze innym, najbardziej interesującym z~metanaukowego punktu widzenia ujęciu i~najbliższym Heisenbergowi, obecnym u~Życińskiego jedynie \textit{implicite}, mówi ona o~nieuniknionym sprzężeniu w~poznaniu ludzkim dwóch, powiedzielibyśmy, kanonicznie powiązanych wielkości: obiektywnego elementu racjonalnego z~subiektywnym elementem \textit{decyzji} i~\textit{commitment}:

\myquote{
W~procedurach pomiarowych mikroprocesów, przy określaniu, czy bardziej dokładnie będzie mierzony pęd cząstki, czy też jej współrzędne, łączy się czynnik obiektywny z~subiektywną podmiotową decyzją. Podobne uwarunkowanie występuje jako specyficzna cecha epistemologii i~jednym z~jej przejawów jest np. to, że decyzja o~wprowadzeniu rygorystycznych kanonów metodologicznych prowadzi do dokładniejszej wiedzy w~pewnych dziedzinach i~do całkowitego ,,rozmycia'' innych dziedzin, które zostają uznane za ,,metafizyczne'' lub ,,niepoznawalne''
%\label{ref:RNDNbMe5IgKQG}(Życiński, 1985, s.~160).
\parencite[][s.~160]{zycinski_teizm_1985}.%
}
W~przypadku epistemologii podobnie jak u~Heisenberga decyzje ograniczają zakres przyszłych możliwości poznawczych.

Niepewność wynikająca z~niedookreśloności epistemologicznej i~metodologicznej tradycji badawczych wyraża się również w~idei prawdopodobnej \textit{doxy}, analogicznie jak \textit{pewność} była cechą tradycyjnej \textit{episteme}. Swą \textit{Strukturę rewolucji metanaukowej} Życiński konkluduje znamiennym stwierdzeniem, że ,,gatunek ludzki żyje transcendentaliami i~(doksatyczną) niepewnością''
%\label{ref:RNDeGAOfCeBL7}(Życiński, 1988b, s.~204, 2013, s.~354).
\parencites[][s.~204]{zycinski_structure_1988}[][s.~354]{zycinski_struktura_2013_liana}. %
 To inny sposób wyrażenia nieeliminowalnej obecności i~konieczności epistemologicznej niepewności w~podstawach nauki i~ludzkiego dyskursu. Z~tego względu realistyczna koncepcja ludzkiej wiedzy i~nauki powinna raczej nazywać się \textit{doxalogią} niż \textit{epistemologią} 
%\label{ref:RNDPNjbXFBV36}(Życiński, 1985, s.~227, 1988b, s.~175, 2013, s.~305, 1996, s.~129, 2015, s.~174)
\parencites[][s.~227]{zycinski_teizm_1985}[][s.~305]{zycinski_structure_1988}[][s.~305]{zycinski_struktura_2013_liana}[][s.~129]{zycinski_elementy_1996}[][s.~174]{zycinski_elementy_2015}%
\footnote{Wynika z~tego, że terminy ‘epistemologia' i~‘epistemologiczny' używane przez Życińskiego mają charakter kontekstualny, podobnie jak wszystkie metateoretyczne kategorie wypracowane w~paradygmacie tradycyjnego racjonalizmu i~sceptycyzmu. Co innego znaczą w~kontekście tradycyjnego racjonalizmu i~sceptycyzmu, a~co innego w~kontekście racjonalizmu umiarkowanego.}.

\section[Aksjologia metanaukowa i~próby domknięcia luki racjonalności]{Aksjologia metanaukowa i~próby domknięcia luki racjonalności\footnote{\ Wyrażeniem ‘aksjologia metanaukowa' Życiński posługuje się w~kontekście omawiania Kuhnowskiej idei związania z~paradygmatem, zob.
%\label{ref:RNDQbycyIn8QK}(Życiński, 1985, s.~161).
\parencite[][s.~161]{zycinski_teizm_1985}. %
 Wyrażenie i~problem ‘domknięcia luki racjonalności' czerpię od A. Groblera 
%\label{ref:RNDGylWKl1LgQ}(1993, s.~11–22).
\parencite*[][s.~11–22]{grobler_prawda_1993}.%
}}
Dla lepszego zrozumienia rozwiązania problemu racjonalności naukowej, jakie zaproponował Życiński, cenne może okazać się pokazanie tej koncepcji w~szerszym metanaukowym kontekście i~porównanie jej z~analogicznym rozwiązaniem zaproponowanym przez Laudana. W~ujęciu Groblera jest to próba domknięcia luki racjonalności, jaka pojawiła się w~dwudziestowiecznej metodologii za sprawą dogmatyzmu aksjologicznego
%\label{ref:RNDFRuULHJmNl}(Grobler, 1993, s.~12nn).
\parencite[][s.~12nn]{grobler_prawda_1993}.%


Pojęcie \textit{commitment} pełni w~koncepcji Życińskiego funkcję analogiczną do pojęcia \textit{decyzji} \textit{metodologicznej}, które zostało wypracowane na gruncie metodologicznego konwencjonalizmu
%\label{ref:RND0GlmHbDv1e}(por. Życiński, 1985, s.~157.159n).
\parencite[por.][s.~157.159n]{zycinski_teizm_1985}. %
 Wypracowano je w~celu domknięcia metateoretycznej luki, jaka pojawiła się w~nauce w~wyniku odkrycia w~fizyce niedookreślenia teorii przez obserwację. W~sytuacji, gdy żadne doświadczenie nie może rozstrzygnąć, która z~przyjętych hipotez została obalona, warunkiem koniecznym rozwoju nauki okazuje się podjęcie przez naukowca decyzji metodologicznej. Aby decyzja taka była racjonalna, musi stosować się do odpowiednich reguł. Krytycznym opracowaniem takich reguł zajmuje się metodologia\footnote{Takie rozumienie metodologii wprowadził Popper 
%\label{ref:RNDN52lzm40ir}(zob. np. Popper, 1979, s.~5.111).
\parencite[zob. np.][s.~5.111]{popper_beiden_1979}. %
 Na temat decyzji konwencjonalistycznych zob. uwagi Poincarégo w
%\label{ref:RNDuAOQQADXPD}(Szumilewicz-Lachman, 1978, s.~261–269).
\parencite[][s.~261–269]{szumilewicz-lachman_poincare_1978}.%
}.

Pojęcie \textit{commitment} jest jednakże próbą domknięcia innej, bardziej fundamentalnej luki niż luka teoretyczna. Nowa luka pojawiła się w~metanauce, gdy uświadomiono sobie, że także teorie metodologiczne podlegają niedookreśleniu przez fakty z~historii nauki i~metanauki i~że konieczne jest podejmowanie decyzji metametodologicznych. W~sytuacji, gdy dane metanaukowe nie pozwalają rozstrzygnąć, która z~równoważnych teorii metodologicznych została obalona, a~którą należy zachować, metodolog musi podjąć decyzję metametodologiczną. Problem w~tym, że budowanie w~nieskończoność kolejnych metametodologii, pozwalających określić warunki racjonalności tego typu wyborów byłoby pozbawione sensu. W~tradycyjnej metodologii XX wieku przyjęło się zatem przecinać ten regres przyjmując, że wybory metametodologiczne dokonywane są w~oparciu o~dalej nieuzasadnialne wartości epistemiczne lub kognitywne. Jest to postawa aksjologicznego dogmatyzmu. Za jej głównego przedstawiciela uważa się K. Poppera
%\label{ref:RNDTkvpBvHhg5}(zob. Grobler, 1993, s.~12).
\parencite[zob.][s.~12]{grobler_prawda_1993}.%


W~znanym ujęciu tego problemu, jaki pochodzi od Laudana
%\label{ref:RNDhbWLFe7WlS}(1984),
\parencite*[][]{laudan_science_1984}, %
 mowa jest o~tzw. hierarchicznym modelu racjonalności, jaki zdominował metodologię XX wieku. Według tego modelu dyskusja w~nauce toczy się na trzech niesprowadzalnych do siebie poziomach: faktualnym, metodologicznym i~aksjologicznym. Pierwszy poziom, poziom nauki przedmiotowej, dotyczy świata danego w~doświadczeniu empirycznym. Wszelkie dyskusje na temat tego, jaki jest świat, w~tym dyskusje teoretyczne, rozstrzygane są przez odwołanie do faktów empirycznych. Ale już spory na temat reguł rozstrzygania empirycznego wypracowywane są i~rozstrzygane w~nie-empirycznej metodologii. W~końcu nierzadkie spory metodologiczne na temat reguł empirycznych rozstrzygane są przez odwołanie do trzeciego, najwyższego poziomu, poziomu wyznawanych hierarchii wartości kognitywnych, wyrażających się w~przyjmowanych celach nauki. Wedle tego modelu cele i~wartości nie podlegają dalszemu uzasadnieniu, lecz są jedynie wyznawane. Wybór zatem wartości pozostaje wyborem czysto subiektywnym i~całkowicie pozaracjonalnym 
%\label{ref:RNDX8WyveF3Yd}(zob. Laudan, 1984, s.~23-26.39-41.47-50; Grobler, 1993, s.~20–22).
\parencites[zob.][s.~23–26.39–41.47–50]{laudan_science_1984}[][s.~2022]{grobler_prawda_1993}.%


Właśnie ten brak racjonalności wyboru wartości definiuje wspomnianą metanaukową lukę, którą należało dopełnić, by zamknąć drogę sceptycyzmowi. Laudan zaproponował jako rozwiązanie tej kwestii tak zwany siateczkowy model racjonalności
%\label{ref:RNDRfSup8slW8}(Laudan, 1984, s.~62–66; Grobler, 1993, s.~20n)
\parencites[][s.~62–66]{laudan_science_1984}[][s.~20n]{grobler_prawda_1993}%
\footnote{Przekład wyrażenia Laudana ‘\textit{reticulated model of rationality}' pochodzi od A. Groblera 
%\label{ref:RNDNkfPWt13Pb}(1993).
\parencite*[][]{grobler_prawda_1993}.%
}. Jest to model swoiście kolisty, w~którym wszystkie trzy elementy trójkąta metodologicznego: element faktualny, metodologiczny i~aksjologiczny wzajemnie się uzasadniają\footnote{Laudan 
%\label{ref:RNDaUl1Z9HuEh}(1984, s.~63)
\parencite*[][s.~63]{laudan_science_1984} %
 mówi o~tzw. \textit{trójelementowej sieci uzasadnienia} (\textit{tradic network of justification}), w~której ,,aksjologia, metodologia i~twierdzenia faktualne są ze sobą w~sposób nieunikniony splecione za pomocą relacji wzajemnej zależności''. Grobler 
%\label{ref:RNDQ9b8cwaGrY}(1993, s.~21)
\parencite*[][s.~21]{grobler_prawda_1993}, %
 nawiązując zapewne do załączonego przez Laudana graficznego schematu tej sieci, mówi o~\textit{trójkącie} uzasadnienia.}. W~terminologii Życińskiego tego typu relację wzajemnego uzasadniania należy uznać za uzasadnienie koliste i~częściowe. Sam też Laudan mówi w~tym kontekście o~próbie ,,zamknięcia koła ocen'' w~celu usunięcia rozbieżności dotyczących wartości kognitywnych 
%\label{ref:RNDxkpNF61Kze}(Laudan, 1984, s.~42).
\parencite[][s.~42]{laudan_science_1984}.%


Pod wieloma względami koncepcja częściowego i~kolistego uzasadnienia przedzałożeń zaproponowana przez Życińskiego jest rozwiązaniem bardzo podobnym do koncepcji Laudana\footnote{Pomimo podobieństw istnieje jednak, jak zobaczymy, wiele istotnych różnic. Ponadto chronologia powstania tych dzieł zdaje się wykluczać możliwość znajomości przez Życińskiego koncepcji Laudana. Swój model racjonalności Laudan przedstawił w~\textit{Science and Values}, która ukazała się w~roku 1984. Życiński swoją koncepcję opublikował w~roku 1985. Nawet gdyby zapoznał się z~książką Laudana, to i~tak trudno przypuszczać, by analizy, jakie zawarł w~\textit{Teizmie}, mogły powstać w~zaledwie klika miesięcy i~być jeszcze w~tamtych czasach równie szybko opublikowane. Warto zauważyć, że także w~swych późniejszych książkach z~filozofii nauki Życiński nie odwołuje się do \textit{Science and Values} Laudana, a~jedynie do jego wcześniejszej \textit{Progress and Its Problems}
%\label{ref:RNDSK7XAhBEvw}(1977).
\parencite*[][]{laudan_progress_1977}.%
}. Wyrażona jest jednak w~odmiennej terminologii \textit{presupozycji} i~\textit{commitment}, która bardziej wskazuje na \textit{Personal Knowledge} Polanyia 
%\label{ref:RNDwFBn3fLGym}(1962)
\parencite*[][]{polanyi_personal_1962} %
 niż \textit{Science and Values} Laudana 
%\label{ref:RND9UI6uo7MT1}(1984)
\parencite*[][]{laudan_science_1984} %
 jako jej źródło. Wprawdzie Życiński nie tematyzuje kwestii aksjologicznej luki w~racjonalności, niemniej kwestia aksjologicznego uwarunkowania wyboru fundamentalnych przedzałożeń stanowi istotny element jego koncepcji racjonalności i~polemiki z~racjonalistycznym optymizmem metanaukowym.

Omawiając nieuniknioną wielość koncepcji metanaukowych, w~szczególności koncepcji racjonalności i~konieczności podmiotowego \textit{commitment} w~ich wyborze, Życiński mówi o~,,aksjologicznym uwarunkowaniu'' takich wyborów. Przyjmuje ono postać akceptowanej hierarchii przedzałożeń: ,,akceptowany system wartościowań prowadzi do określonej hierarchii'' tych twierdzeń
%\label{ref:RNDUU3LMRxLMB}(Życiński, 1985, s.~225).
\parencite[][s.~225]{zycinski_teizm_1985}. %
 Życiński w~przeciwieństwie do Laudana nie ogranicza się jednak do samych wartości kognitywnych określanych przez aksjologię metanaukową, lecz mówi również o~wartościach etycznych i~estetycznych obecnych w~akcie podmiotowego \textit{commitment}\footnote{Zob. 
%\label{ref:RNDWRO4yGbJPh}(Życiński, 1988b, s.~147–156, 2013, s.~257–274),
\parencites[][s.~147–156]{zycinski_structure_1988}[][s.~257–274]{zycinski_struktura_2013_liana}, %
 gdzie mówi o~relacji nauki do aksjologii humanistycznej. Jest to aksjologia różna od aksjologii metanaukowej. Na temat wartości omawianych w~aksjologii metanaukowej i~problemu hierarchizacji przedzałożeń zob. też 
%\label{ref:RNDXZYDxOPXAF}(Życiński, 1985, s.~161.169n).
\parencite[][s.~161.169n]{zycinski_teizm_1985}.%
}.

Ideę metanaukowego wartościowania Życiński wiąże z~pojęciem kontekstu akceptacji, który odróżnia od kontekstu odkrycia i~od kontekstu uzasadnienia
%\label{ref:RNDbdnKgA4NJn}(zob. Życiński, 1985, s.~225)
\parencite[zob.][s.~225]{zycinski_teizm_1985}%
\footnote{Relację między kontekstem uzasadnienia a~kontekstem akceptacji Życiński omawia najobszerniej w
%\label{ref:RNDa8zVqANSqx}(Życiński, 1985, s.~216–232).
\parencite[][s.~216–232]{zycinski_teizm_1985}. %
 }. Jego \textit{kontekst akceptacji} różni się jednak zasadniczo od analogicznego \textit{context of acceptance} Laudana 
%\label{ref:RNDqiX8WYsx7l}(1977, s.~108nn).
\parencite*[][s.~108nn]{laudan_progress_1977}. %
 Laudan również wiąże ten kontekst z~\textit{commitment}: naukowiec musi choćby na próbę zaangażować się w~akceptację (\textit{to commit himself to the acceptance}) jednej z~tradycji badawczych, ale Laudan pojmuje akceptację na sposób racjonalistycznej tradycji metodologicznej XX wieku jako racjonalne ‘\textit{acceptability}' i~‘\textit{warranted assertibility}' 
%\label{ref:RNDFx4k1oQYeE}(Laudan, 1977, s.~110).
\parencite[][s.~110]{laudan_progress_1977}. %
 Co więcej, Laudan zarzuca koncepcji akceptacji teorii zaproponowanej przez Lakatosa brak racjonalnego charakteru akceptacji 
%\label{ref:RNDCuq49glQFi}(Laudan, 1977, s.~77n).
\parencite[][s.~77n]{laudan_progress_1977}. %
 Sam proponuje bardziej racjonalną koncepcję: należy wybierać te teorie i~te tradycje badawcze, które ,,lepiej rozwiązują problemy od swych konkurentek'' 
%\label{ref:RNDCdfevDPzJd}(Laudan, 1977, s.~109)
\parencite[][s.~109]{laudan_progress_1977}%
\footnote{To zapewne usprawiedliwia milczenie Życińskiego na temat koncepcji Laudana, pomimo iż jego \textit{Progress} 
%\label{ref:RNDFfDGvCPF0a}(1977)
\parencite*[][]{laudan_progress_1977} %
 przywołuje w~swym \textit{Teizmie} 
%\label{ref:RNDgO5ibm9vG0}(1985)
\parencite*[][]{zycinski_teizm_1985} %
 co najmniej kilkukrotnie omawiając inne zagadnienia.}.

Kategoria ‘kontekstu akceptacji' Życińskiego ma zadanie wyjaśniać sposób funkcjonowania czynników pozaracjonalnych w~wyborach, jakich musi dokonywać naukowiec i~filozof stojący w~obliczu równoważnych i~alternatywnych interpretacji
%\label{ref:RNDgfQIpozFca}(Życiński, 1985, s.~225nn).
\parencite[][s.~225nn]{zycinski_teizm_1985}. %
 Główną rolę w~takim wyborze pełnią zasady pozwalające przyporządkować oceny wartościujące konkurencyjnym interpretacjom\footnote{Podobne stwierdzenie znajdujemy w~%
%\label{ref:RNDGn40dsMp4z}(Życiński, 1988b, s.~137, 2013, s.~242n):
\parencites[][s.~137]{zycinski_structure_1988}[][s.~242n]{zycinski_struktura_2013_liana}: %
 ponieważ kontekst uzasadnienia jest jedynie kontekstem częściowego uzasadnienia, zatem musi zostać uzupełniony o~element kontekstu akceptacji, w~którym ,,czynnik częściowego racjonalnego uzasadnienia wiąże się z~czynnikiem osobowego \textit{commitment}'' w~celu ustalenia zarówno ,,podstawowych zasad interpretacji'', jak i~,,specyficznej hierarchii ważności poszczególnych zasad''. Polski przekład 
%\label{ref:RND5NOLIC6GqB}(Życiński, 2013, s.~243)
\parencite[][s.~243]{zycinski_struktura_2013_liana} %
 używa w~tym miejscu wyrażenia ‘kontekst przyjęcia'. Sam Życiński używa jednak polskiego wyrażenia ‘kontekst akceptacji'.}. Współcześnie wiadomo jednak, że wypracowanie jednej uniwersalnie akceptowanej hierarchii przedzałożeń jest nieosiągalne nawet w~logice i~matematyce. Nauka jako taka ,,milczy o~wartościach'', zatem przyszły rozwój nauki niczego nie zmieni. Ewentualna zgoda uczonych na hierarchię wartościowań może mieć co najwyżej charakter socjologiczny, nie zaś logiczny. Kolejne pokolenia uczonych nie są w~żaden sposób zobligowane logicznie do respektowania hierarchii wartościowań swych poprzedników. Współcześnie nie podziela się wielu elementów średniowiecznego czy oświeceniowego obrazu świata i~nauki\footnote{Nie jest to jednak w~przekonaniu Życińskiego argument za niewspółmiernością paradygmatów. Całkowita niewspółmierność byłaby wyrazem metanaukowego antyrealizmu. W~perspektywie racjonalizmu umiarkowanego Życińskiego antyrealizm jest tak samo zdeterminowany przez antyrealistyczne przedzałożenia, jak realizm przez założenia realistyczne. Co więcej -- jest to oczywiście uzasadnienie tylko częściowe -- antyrealizm stoi w~sprzeczności z~fundamentalnym faktem występowania międzyparadygmatycznego porozumienia i~zrozumienia.}. Wiara w~możliwość ustalenia jednej hierarchii wartościowań dotyczących poszczególnych przedzałożeń czy reguł metodologicznych byłaby przejawem braku realizmu metanaukowego:

\myquote{
wyrazem nierealistycznego optymizmu byłoby oczekiwanie na powszechną zgodę filozofów odnośnie do hierarchii wartościowań dotyczących poszczególnych przedzałożeń czy reguł metodologicznych
%\label{ref:RNDK5O2oIIwOH}(Życiński, 1985, s.~226).
\parencite[][s.~226]{zycinski_teizm_1985}.%
}
Tego typu optymizm byłby kolejnym przedzałożeniem, analogicznym do pozostałych. Byłoby to jednak przedzałożenie sfalsyfikowane przez rzeczywistą naukę i~jej historię. Zatem podtrzymywanie go byłoby dzisiaj przejawem dogmatyzmu i~swoistego epistemologicznego antyrealizmu.

Wybór wartości w~rozwiązaniu Życińskiego nie jest wszakże wyborem irracjonalnym i~całkowicie subiektywnym. Życiński nie unika uzasadniania wartości kognitywnych, jakkolwiek nie nazywa tego w~ten sposób. Powyższa krytyka optymizmu metanaukowego pokazuje, że także aksjologiczne \textit{commitment} podlega racjonalnej ocenie i~(częściowemu) uzasadnieniu. Wielokrotnie również Życiński poddaje racjonalnej dyskusji kwestię właściwego celu nauki, który w~koncepcji Laudana stanowi jedną z~podstawowych wartości kognitywnych każdej tradycji badawczej\footnote{Laudan
%\label{ref:RNDL61hUlRw3J}(1984, s.~44n)
\parencite*[][s.~44n]{laudan_science_1984} %
 wiąże aksjologię nauki z~dyskusją kognitywnych celów wspólnoty naukowej oraz z~oceną tychże celów (‘\textit{cognitive goals}', ‘\textit{goal evaluation}', ‘\textit{basic aims and goals}').}. W~tym wypadku jednak Życiński staje po stronie Poppera i~realizmu, a~przeciw Laudanowi i~antyrealizmowi.

W~\textit{Progress and its Problems} Laudan
%\label{ref:RNDhtBfzIbZcA}(1977)
\parencite*[][]{laudan_progress_1977} %
 uznał, że pojęcie prawdy jako cel nauki jest logicznie nie do utrzymania w~obliczu kłopotów z~Popperowskim pojęciem przybliżania się do prawdy. Postuluje tam zatem, by porzucić klasyczne, nie-pragmatyczne pojęcie prawdy i~przybliżenia do prawdy i~zbudować antyrealistyczną koncepcję racjonalności i~racjonalnego postępu w~oparciu o~pojęcie \textit{problem-solving} 
%\label{ref:RNDx4LQQidUw4}(Laudan, 1977, s.~109.121-125)
\parencite[][s.~109.121--125]{laudan_progress_1977}%
\footnote{Życiński 
%\label{ref:RNDSDke1gzBZQ}(1985, s.~203)
\parencite*[][s.~203]{zycinski_teizm_1985} %
 mówi o~Laudanie, że do swej koncepcji nie wprowadza pojęcia prawdy i~przybliżenia do prawdy, lecz ,,odwołuje się do wskaźnika efektywności rozwiązywania problemów''.}. Życiński potwierdza wszystkie logiczne słabości Popperowskich pojęć uprawdopodobnienia (\textit{verisimilitude}) i~przybliżenia do prawdy (\textit{approximation to truth})\footnote{Koncepcję \textit{verisimilitude} (uprawdopodobnienia) Życiński omawia szeroko w~
%\label{ref:RNDGoLltfWABA}(1996, s.~110–119, 2015, s.~147–160).
\parencites*[][s.~110–119]{zycinski_elementy_1996}[][s.~147–160]{zycinski_elementy_2015}.%
}. Co więcej, uznaje, że rozwiązanie Laudana przedstawia w~obecnej sytuacji metateoretycznej najlepsze perspektywy rozwoju racjonalistycznej tradycji badawczej w~metanauce dlatego, że akcentuje doniosłość pozbawionych wyjaśnienia sytuacji problemowych. Odrzuca jednak propozycje Laudana pozbycia się prawdy i~przybliżenia do prawdy i~przyjęcia czysto instrumentalistycznej interpretacji teorii naukowych i~metanaukowych. Na przekór samemu Laudanowi postuluje, by jego pragmatyczne idee, podobnie jak idee jego intelektualnego oponenta Newtona-Smitha 
%\label{ref:RNDt65nJbE1Nh}(1981),
\parencite*[][]{newton-smith_rationality_1981}, %
 wykorzystać w~celu lepszego opracowania pojęcia przybliżenia do prawdy 
%\label{ref:RND0AX408dGww}(Życiński, 1996, s.~117n, 2015, s.~156nn)
\parencites[][s.~117n]{zycinski_elementy_1996}[][s.~156nn]{zycinski_elementy_2015}%
\footnote{Występuje pewna niezgodność między interpretacją Życińskiego a~sformułowaniami samego Laudana. Życiński mówi 
%\label{ref:RNDT9rW4OUZFJ}(1996, s.~117, 2015, s.~156)
\parencites*[][s.~117]{zycinski_elementy_1996}[][s.~156]{zycinski_elementy_2015} %
 o~koncepcji Laudana, że przyjmuje za ,,podstawę wartościowań zawartość treściową teorii pozwalającą na rozwiązywanie problemów''. Tymczasem Laudan 
%\label{ref:RNDR3ujUDOZm5}(1977, s.~77)
\parencite*[][s.~77]{laudan_progress_1977} %
 zarzuca \textit{explicite} koncepcji programów badawczych Lakatosa i~uznaje za jej słabość to, że odwołuje się do pojęcia Tarskiego i~Poppera ‘zawartości empirycznej i~logicznej' teorii (\textit{empircial and logical content}). Za Grünbaumem powtarza, że próba określenia miary zawartości teorii naukowych jest wysoce problematyczna, jeśli w~ogóle możliwa. Ocena tej rozbieżności wymagałaby bliższych analiz porównawczych.}.

Odmiennie od Laudana Życiński opowiada się również za realizmem naukowym i~teoretycznym. Ma świadomość antyrealistycznych implikacji epistemologicznej zasady niedookreśloności teorii, ale nie uważa, by były to implikacje konieczne
%\label{ref:RNDe9BkHT28aq}(Życiński, 1985, s.~130).
\parencite[][s.~130]{zycinski_teizm_1985}. %
 Świadomie też opowiada się (\textit{commitment}) za realizmem i~tego przedzałożenia broni. Czyni to w~sposób jak najbardziej racjonalny, uznając je za jeden z~warunków koniecznych racjonalizmu naukowego w~ogóle 
%\label{ref:RND3BOkVWKKj8}(Życiński, 1985, s.~170–195).
\parencite[][s.~170–195]{zycinski_teizm_1985}.%


Dyskutując liczne argumenty za i~przeciw realizmowi ontycznemu i~poznawczemu Życiński zauważa, że argumenty te nie są nigdy rozstrzygające, ani w~jedną, ani w~drugą stronę\footnote{Przykładem może być tutaj jego polemika z~konstruktywnym empiryzmem van Fraassena
%\label{ref:RNDsprN8sOvis}(zob. Życiński, 1985, s.~183, 1988b, s.~118n, 2013, s.~209nn).
\parencites[zob.][s.~183]{zycinski_teizm_1985}[][s.~118n]{zycinski_structure_1988}[][s.~209nn]{zycinski_struktura_2013_liana}. %
 }. Ponieważ jednak odrzucenie realizmu musiałoby skutkować solipsyzmem poznawczym, zatem zwolennik tradycji broniącej racjonalności musi aktem wiary i~\textit{commitment} przyjąć realistyczne przedzałożenia:

\myquote{
Podczas gdy alternatywą dla teizmu był agnostycyzm, alternatywą dla realizmu jest solipsyzm. [...] w~sytuacji braku rozstrzygających argumentów trzeba stwierdzić, iż uznanie realizmu ontologicznego i~epistemologicznego zawiera w~sobie element ryzyka podobny do ryzyka zakładu Pascalowskiego. Istnieją pragmatyczne racje, aby przyjąć realizm ze względu na doniosłość jego konsekwencji. Uzasadnienie podmiotowego wyboru między realizmem i~antyrealizmem nie jest jednak obiektywnym argumentem przemawiającym na korzyść realizmu
%\label{ref:RND43CpxDMTWN}(Życiński, 1985, s.~175 i~186).
\parencite[][s.~175 i~186]{zycinski_teizm_1985}.%
}
I~nieco dalej konkluduje:

\myquote{
Leżące u~podstaw wszelkiej wiedzy dążenie do poznania pozasubiektywnej rzeczywistości wymaga, aby najpierw aktem wiary przyjąć tezę o~istnieniu tej rzeczywistości
%\label{ref:RNDhZTJw6uUMx}(Życiński, 1985, s.~186).
\parencite[][s.~186]{zycinski_teizm_1985}.%
}
Podobnie argumentuje za przedzałożeniem racjonalności ontycznej:

\myquote{
trudny do wyjaśnienia fakt ontycznej racjonalności świata jawi się jako warunek konieczny możliwości wszelkiego dyskursu zarówno w~płaszczyźnie przyrodniczej, jak i~filozoficznej. [...] Tezy tej nie daje się uzasadnić przez odwołanie do bardziej podstawowych własności przyrody natomiast jej uznanie jest konieczne dla uprawomocnienia obszernego zbioru innych twierdzeń. W~przypadku przedzałożenia racjonalności świata pragmatyka badawcza okazuje się znowu ważniejsza od racjonalnych uzasadnień
%\label{ref:RNDg0GSXc5DAP}(Życiński, 1985, s.~194–195).
\parencite[][s.~194–195]{zycinski_teizm_1985}.%
}

Z~tych wszystkich analiz Życiński wyprowadza jedną istotną konkluzję. Nie istnieje jeden realizm. Stary, naiwny, przyjmujący ideę swoistego \textit{izomorfizmu} między poznawaną rzeczywistością a~naszym jej poznaniem, musi zostać zarzucony w~obliczu rozlicznych kontrargumentów. Odpowiedzią na zasadną krytykę tradycyjnego realizmu nie jest jednak jego porzucenie -- nie ma takiej logicznej konieczności -- lecz wypracowanie nowej ,,pośredniej'' koncepcji wykraczającej poza ,,tradycyjną opozycję między realizmem i~różnymi wariantami antyrealizmu''
%\label{ref:RNDORPssVXNrx}(Życiński, 1985, s.~184)
\parencite[][s.~184]{zycinski_teizm_1985}%
\footnote{Powraca tutaj idea nowego rozumienia starych kategorii.}.

Oczywiście wszystkie argumenty, jakie przedstawia Życiński za realizmem, prawdą i~uprawdopodobnieniem mają charakter uzasadniania jedynie częściowego i~ostatecznie kolistego. Nie może to być jednak zarzutem, gdyż kolistość w~uzasadnianiu przedzałożeń jest istotnym elementem tradycji badawczej, z~którą wiąże się Życiński. W~sytuacji, gdy do wyboru mamy dwie zasadniczo równoważne tradycje racjonalności, Życiński opowiada się za tradycją, w~której racjonalność wiąże się z~prawdą jako celem poznania i~z umiarkowanym realizmem jako nową koncepcją tej prawdy. Wraz z~Newtonem-Smithem
%\label{ref:RNDF4NGv3QNVZ}(Newton-Smith, 1981, s.~273),
\parencite[][s.~273]{newton-smith_rationality_1981}, %
 a~zatem przeciw Laudanowi, z~którego koncepcją Newton-Smith polemizuje, konkluduje mówiąc, że ,,realizm jest prawdą a~umiarkowany racjonalizm -- drogą do prawdy'' 
%\label{ref:RNDTqCW4MDzzM}(Życiński, 1985, s.~205, 1988b, s.~135, 2013, s.~239)
\parencites[][s.~205]{zycinski_teizm_1985}[][s.~135]{zycinski_structure_1988}[][s.~239]{zycinski_struktura_2013_liana}%
\footnote{Podobne podejście do Laudanowskiej koncepcji celów nauki Laudana zaproponował wcześniej A. Grobler 
%\label{ref:RNDYps8mmrkOh}(1993, s.~23.35-42).
\parencite*[][s.~23.35-42]{grobler_prawda_1993}. %
 O~ile jednak Życiński omawia koncepcję Laudana w~kontekście \textit{verisimilitude}, o~tyle Grobler w~szerszym kontekście aksjologicznej luki.}.

Podobnie też do Newtona-Smitha Życiński zdaje się twierdzić, że wprawdzie koncepcje racjonalności metodologicznej ewoluują, ale główny cel nauki pozostaje ten sam, choć nie taki sam. Celem tym jest prawda, jakkolwiek samo rozumienie tego, jak ta prawda powinna być rozumiana, zmienia się wraz z~paradygmatami. Newton-Smith odrzuca Laudanowską ewolucję celów nauki rozumianą jako zmiana podstawowych wartości, ale już nie ewolucję rzeczywistego celu nauki pojmowaną jako modyfikacja rozumienia tego celu\footnote{Na temat celu nauki, który pozostaje ten sam ale nie taki sam, zob.
%\label{ref:RNDUK6tOd4iiN}(Newton-Smith, 1981, s.~221nn.269n).
\parencite[][s.~221nn.269n]{newton-smith_rationality_1981}.%
}. Podobnie postępuje Życiński: świadome opowiedzenie się za racjonalną nauką oraz za prawdą i~realizmem gwarantuje zasadniczą trwałość podstawowych celów nauki, ale nie oznacza ich braku podatności na modyfikację. Jego koncepcja ewolucji pojęcia racjonalności i~pojęcia realizmu, od ujęć naiwnych do bardziej umiarkowanych, jest tego dostatecznym potwierdzeniem.

Analizy powyższe pokazują, że w~ujęciu Życińskiego hierarchia wartości decydujących o~podmiotowym \textit{commitment} do określonej tradycji i~określonego zbioru przedzałożeń oraz sam wybór tych wartości również są niedookreślone faktami z~dziedziny historii nauki i~metanauki. Wybór aksjologiczny musi zatem ostatecznie podlegać podmiotowemu \textit{commitment}, analogicznie jak w~przypadku wyboru przedzałożeń. Musi także podlegać częściowemu uzasadnieniu, jeśli ma to być wybór racjonalizmu a~nie sceptycyzmu. Widać więc, że podobnie jak Laudan Życiński odrzuca hierarchiczny model uzasadniania w~metanauce. Wartości poznawcze nie wyłamują się z~metodologicznego koła presupozycji. Nie stoją ponad nim w~swego rodzaju dogmatycznej i~nieuzasadnialnej przestrzeni aksjologicznej. Wprawdzie kategorie ‘\textit{commitment}' i~‘aktu wiary' mogłyby sugerować przyjęcie takiego aksjologiczne fundamentu, ale byłaby to sugestia błędna. Nie uwzględniałaby bowiem tego, że Życiński nie traktuje ani pojęcia \textit{commitment}, ani pojęcia wiary skrajnie subiektywistycznie, czyli jako wyrazu irracjonalności, w~duchu tradycyjnego racjonalizmu. Przeciwnie, traktuje je analogicznie do innych kategorii podmiotowych, czyli w~perspektywie nowej koncepcji racjonalizmu umiarkowanego\footnote{Zob. wyżej przypis 30 na temat operatorów ‘wiedzieć' i~‘wierzyć' oraz uwagę o~‘epistemologii' w~przypisie 42.}. Z~jednej strony są one nie w~pełni uzasadnionym podmiotowym elementem każdej postawy metanaukowej, w~szczególności racjonalizmu. Z~drugiej podlegają częściowemu i~kolistemu uzasadnieniu.

W~stosunku do rozwiązania Laudana idea presupozycyjnego \textit{commitment} wydaje się mieć pewną przewagę, a~na pewno ma ją z~punktu widzenia samego Życińskiego. Po pierwsze, podkreśla pro-racjonalną rolę czynników zewnętrznych. Tym samym pokazuje, jak uniknąć nieskończonych i~nierozstrzygalnych dyskusji w~aksjologicznym kole Laudana. Mówiąc o~wartościach kognitywnych i~o ich racjonalnej dyskusji Laudan nie wskazuje żadnej z~nich jako fundamentalnej. Wszystkie wartości kognitywne są u~niego logicznie równoważne i~wszystkie mogą być równie dobrze ,,obalone''
%\label{ref:RNDSoHXsajfnI}(zob. Grobler, 1993, s.~23).
\parencite[zob.][s.~23]{grobler_prawda_1993}. %
 Sytuacja taka prowadzi do aksjologicznego błądzenia, które trudno pogodzić z~ideą postępu, rozwijaną przez Laudana 
%\label{ref:RND840AlRHNks}(Grobler, 1993, s.~32–35).
\parencite[][s.~32–35]{grobler_prawda_1993}. %
 Bez pojęcia \textit{dążenia do prawdy} trudno wytłumaczyć nie tylko rozwój nauki, ale i~rozwój metanauki\footnote{Pomimo krytyki metanaukowego optymizmu Poppera i~innych autorów Życiński przyjmuje istnienie ograniczonego postępu w~rozwoju filozofii, zob. np. 
%\label{ref:RND5RvAGsOsbs}(Życiński, 1988a, s.~47–53).
\parencite[][s.~47–53]{zycinski_teizm_1988}. %
 Jest to przejaw swoistego realizmu metateoretycznego.}. Ukazanie (częściowo) racjonalnego charakteru opowiedzenia się za prawdą i~uprawdopodobnieniem to druga z~przewag koncepcji Życińskiego nad koncepcją Laudana.

Ale sama akceptacja prawdy jako celu nauki nie wystarcza do rozwiązania istotnego problemu metanaukowego, jakim jest częściowo pozaracjonalny charakter takiej akceptacji. Z~konieczności także ona jest do pewnego stopnia dziełem czynników zewnętrznych. W~każdej akceptacji zawarty jest element podmiotowego \textit{commitment}, analogicznie jak w~decyzji metodologicznej konwencjonalistów. W~przekonaniu Życińskiego, przeciwnie do Laudana, dopiero uwzględnienie pro-racjonalnego charakteru \textit{commitment} w~kontekście akceptacji pozwala na w~pełni realistyczne wyjaśnienie racjonalności naukowej. Dopiero akceptacja pro-racjonalnego charakteru pozaracjonalnego \textit{commitment} i~pozaracjonalnej \textit{akceptacji} pozwala racjonaliście opanować kolistość presupozycyjną i~niejako zaprząc ją do swych celów.

\section{Racjonalność pragmatyczna i~zakład Pascala}
Niejasną kwestią u~Życińskiego pozostaje kwestia racjonalności pragmatycznej. Racjonalizm umiarkowany i~doksatyczny wydaje się zawierać w~sobie istotny element racjonalności pragmatycznej. W~pierwszym tomie \textit{Teizmu i~filozofii analitycznej}
%\label{ref:RNDhsCAwCNS6s}(1985)
\parencite*[][]{zycinski_teizm_1985} %
 Życiński zalicza ten typ racjonalności do szeroko rozumianej racjonalności poznawczej\footnote{Omówieniu różnych typów racjonalności poznawczej Życiński poświęca wiele miejsca w
%\label{ref:RND9thEwL5Ypb}(Życiński, 1985, s.~195–216, 1988b, s.~128–135, 2013, s.~276–239, zob. także 1993, s.~14.235-240).
\parencites[][s.~195–216]{zycinski_teizm_1985}[][s.~128–135]{zycinski_structure_1988}[][s.~276–239]{zycinski_struktura_2013_liana}[zob. także][s.~14.235–240]{zycinski_granice_1993}. %
 Racjonalność pragmatyczną omawia szeroko w~
%\label{ref:RNDKFRW8sTxhD}(Życiński, 1985, s.~205n.207-216).
\parencite[][s.~205n.207–216]{zycinski_teizm_1985}. %
 Uznanie racjonalności metodologicznej za podzbiór pragmatycznej (tamże, s.~205) sprawia, że pojęcie racjonalności poznawczej i~pragmatycznej krzyżują się. W~podziale racjonalności Życińskiego występuje zatem pewna niespójność wynikająca z~jednoczesnego posługiwania się szerokim i~wąskim rozumieniem dzielonego pojęcia i~członów podziału. Należy też zauważyć, że w~
%\label{ref:RNDC3RVCj0G9D}(Życiński, 1988b, s.~132, por. 2013, s.~234)
\parencites[][s.~132]{zycinski_structure_1988}[por.][s.~234]{zycinski_struktura_2013_liana} %
 ‘racjonalność treściowa' oddana jest angielskim terminem ‘\textit{objective rationality}' oraz że angielska wersja przeszła z~czasem do tekstów polskich. W~\textit{Elementach} 
%\label{ref:RND3cw1oLgBB5}(1996, s.~248–251, 2015, s.~336–340)
\parencites*[][s.~248–251]{zycinski_elementy_1996}[][s.~336–340]{zycinski_elementy_2015} %
 Życiński mówi już wyłącznie o~racjonalności obiektywnej.}. W~obrębie tej ostatniej wyróżnia kolejno racjonalność \textit{formalną} (wewnętrzna niesprzeczność systemu), \textit{treściową} (wzajemna spójność podstawowych założeń systemu), \textit{zdroworozsądkową}, \textit{metodologiczną} (problem racjonalnego rozwoju nauki i~uzasadnienia twierdzeń) oraz \textit{pragmatyczną}. Ponadto racjonalność metodologiczną uznaje za szczególny przypadek racjonalności pragmatycznej. Jednakże z~czasem zmienia swe stanowisko. Trzy lata po ukazaniu się pierwszego tomu \textit{Teizmu} stwierdza \textit{explicite}, że racjonalność pragmatyczna, polegająca na dobieraniu środków do celu działania, ma niewielkie znaczenie dla filozofii nauki i~dlatego pomija jej omówienie 
%\label{ref:RNDnQWVxMDiHM}(Życiński, 1988b, s.~134, 2013, s.~237).
\parencites[][s.~134]{zycinski_structure_1988}[][s.~237]{zycinski_struktura_2013_liana}.%


Jednak w~swych argumentach za racjonalnymi rozwiązaniami metanaukowymi Życiński nie przestał odwoływać się do pragmatyki badawczej jako ostatecznej instancji rozstrzygającej (częściowo) o~wartości argumentu metanaukowego. W~\textit{Teizmie}
%\label{ref:RNDeRDlnHdonW}(1985, s.~194)
\parencite*[][s.~194]{zycinski_teizm_1985} %
 konkluduje swój argument za wyborem racjonalności w~następujący sposób: ,,w przypadku przedzałożenia racjonalności świata \textit{pragmatyka badawcza} okazuje się \textit{znowu} ważniejsza od racjonalnych uzasadnień''\footnote{Podkreślenie moje.  Wyrażenie ‘znowu' wskazuje na niejednostkowy i~trwały charakter tego typu sytuacji w~nauce. Metanauka realistyczna nie powinna zbywać tego typu faktów milczeniem, lecz uwzględniać w~wyjaśnianiu samej racjonalności.}. Podobną argumentację zastosuje dziesięć lat później oceniając koncepcję programów badawczych Lakatosa. To, że uwzględnia ona sytuacje, w~których ,,pragmatyka badań góruje nad wewnętrzną logiką nauki'', Życiński uznaje za ,,przejaw realistycznego stosunku wobec danych historii nauki'' 
%\label{ref:RNDnkTEgXJ5NG}(Życiński, 1996, s.~243, 2015, s.~330).
\parencites[][s.~243]{zycinski_elementy_1996}[][s.~330]{zycinski_elementy_2015}.%


Ponadto rozwiązanie problemu kolistości przez Życińskiego przypomina do pewnego stopnia pragmatyczne rozwiązanie kwestii antynomii przez Russella i~Tarskiego, jak też argument Heisenberga za koniecznością zasady nieoznaczoności. Można przyjąć, że u~podstaw matematyki tkwią nieusuwalne antynomie i~odrzucić całą matematykę, gdyż antynomie prowadzą do sprzeczności. Chcąc jednak zachować matematykę należy przyjąć, że zakazana jest budowa zdań, w~których język miesza się z~metajęzykiem, a~zbiory mogą być elementem samych siebie. Należy też przyjąć, że pewien rodzaj kolistości nie może być szkodliwy dla matematyki\footnote{Podobnie argumentuje Heisenberg w~swej z~pracy
%\label{ref:RNDRLOGuLAHoO}(1927, s.~179n).
\parencite*[][s.~179n]{heisenberg_uber_1927}. %
 Nieokreśloność kwantowa jest konieczna, gdyż bez niej niemożliwa byłaby w~ogóle mechanika kwantowa, podobnie jak teoria względności nie byłaby możliwa, gdyby możliwa była definicja absolutnej, a~nie tylko względnej, równoczesności.}. Analogicznie w~epistemologii: albo przyjmiemy nowe metodologiczne rozumienie kolistości i~wewnętrznej sprzeczności, albo będziemy musieli zrezygnować z~nauki i~zaakceptować sceptycyzm. Taki sam typ argumentu występuje u~Życińskiego, gdy mówi o~konieczności przyjęcia przedzałożenia racjonalności świata (zob. wyżej).

W~argumentach tego typu nie mamy do czynienia z~koniecznością absolutną, a~jedynie z~koniecznością warunkową, zależną od przyjętego punktu wyjścia. Można ten typ argumentacji nazwać za Kantem argumentem transcendentalnym. Ale nie będzie to transcendentalizm Kantowski, lecz co najwyżej Popperowski. Dotyczy on konwencji znaczeniowych i~definicji i~jako taki nie ma charakteru rozstrzygającego w~sposób ostateczny\footnote{Zob. np. wyżej punkt 1(c), cytat z
%\label{ref:RNDJcjjhdmx3n}(Życiński, 1985, s.~79).
\parencite[][s.~79]{zycinski_teizm_1985}. %
 Szerzej na temat tego typu metody zob. 
%\label{ref:RNDdOyndd6Uou}(Liana, 2019, s.~153n).
\parencite[][s.~153n]{liana_nauka_2019_liana}.%
}. Odwołanie się do pragmatyki badań w~tego typu argumentach pozwala mówić o~ich pragmatyczno-racjonalnym charakterze, różnym od racjonalności czysto formalnej i~treściowej.

Dodatkowo, w~\textit{Teizmie} Życiński wielokrotnie mówi o~,,Pascalowskim ryzyku'', jakie podejmuje naukowiec i~filozof wiążąc się podmiotowym \textit{commitment} z~określoną tradycją badawczą
%\label{ref:RNDjD6gzXNJF3}(Życiński, 1985, s.~135.86.228.230).
\parencite[][s.~135.86.228.230]{zycinski_teizm_1985}. %
 Z~kolei zakład Pascala jest dla niego typowym przykładem racjonalności pragmatycznej 
%\label{ref:RNDFJzwxxv4Kn}(Życiński, 1985, s.~207nn)
\parencite[][s.~207nn]{zycinski_teizm_1985}%
\footnote{Faktem jest też, że w~kolejnych książkach Życińskiego temat zakładu Pascala nie pojawia się.}. Owszem, analizy Życińskiego tego zakładu mają na celu zasadniczo pragmatyczną obronę teizmu przeciw agnostycyzmowi, niemniej wyprowadza on z~nich także konkluzje ogólne. Bez trudu można je odnieść do podmiotowego \textit{commitment} w~wyborze racjonalistycznych przedzałożeń:

\myquote{
Metodyka pragmatycznej racjonalności sugerowana przez Pascala i~Jamesa ma za cel przeciwdziałanie sceptycyzmowi przy rozwiązywaniu zagadnień najbardziej doniosłych. Jej przyjęcie nie zmienia prawdopodobieństwa w~pewność, lecz wskazuje interpretację, którą można uznać za najbardziej racjonalną pragmatycznie
%\label{ref:RNDVdPkjuKaRT}(Życiński, 1985, s.~216).
\parencite[][s.~216]{zycinski_teizm_1985}.%
}
Bez wątpienia do takich doniosłych zagadnień można zaliczyć także kwestię racjonalności i~obrony nauki przed sceptycyzmem. Także te zagadnienia w~sytuacji braku ostatecznego uzasadnienia obiektywnego warte są podjęcia Pascalowskiego ryzyka \textit{commitment} i~przyjęcia przedzałożeń w~metanaukowym akcie wiary. Życiński nie odwołuje tych tez w~swych kolejnych książkach. Nie do końca wiadomo zatem, dlaczego z~czasem uznał racjonalność pragmatyczną za nieistotną dla filozofii nauki. Można jedynie domyślać się i~wysuwać hipotezy, które z~konieczności będą trudne do obalenia. Bez wątpienia bardziej interesująca byłaby taka sytuacja, w~której Życiński zamiast zawężać swoje pojęcie racjonalności pragmatycznej dokonałby jego pogłębienia, by lepiej współgrało z~koncepcją racjonalizmu umiarkowanego\footnote{Być może pozwoliłoby mu to uwzględnić i~wykorzystać ideę naturalizmu metodologicznego Laudana
%\label{ref:RNDoGj7Gy7523}(1987),
\parencite*[][]{laudan_progress_1987}, %
 czy też reliabilistyczne koncepcje nieszkodliwego (\textit{harmless}) naturalizmu metodologicznego, zob. 
%\label{ref:RND4xXm3tu5Ko}(Almeder, 1998; Liana, 2003, s.~136nn).
\parencites[][]{almeder_harmless_1998}[][s.~136nn]{liana_naturalistyczne_2003}. %
 Życiński ogranicza swe rozumienie naturalizmu epistemologicznego do naturalizacji epistemologii w~stylu Quine'a i~Szkoły Edynburskiej 
%\label{ref:RNDQMpb7Y5sq6}(zob. np. Życiński, 1985, s.~228, 1996, s.~134.157nn, 2015, s.~181.214nn).
\parencites[zob. np.][s.~228]{zycinski_teizm_1985}[][s.~134.157nn]{zycinski_elementy_1996}[][s.~181.214nn]{zycinski_elementy_2015}.%
}.

Przykładowo, cytowane wyżej słowa Życińskiego zwracają uwagę na jeden istotny aspekt zakładu Pascala i~racjonalności pragmatycznej. Życiński pokazuje, że zakład Pascala można rozumieć jako element czysto matematycznej teorii gier stosującej się do dowolnego działania w~sytuacji niepełnego określenia dostępnych alternatyw. Ale wtedy byłaby to jedynie zwykła gra prawdopodobieństw o~niewielkim znaczeniu egzystencjalnym i~aksjologicznym. Życiński podkreśla wszakże wagę specyficznego kontekstu odniesienia tego zakładu. Chodzi o~rozwiązywanie zagadnień najbardziej doniosłych. W~domyśle: doniosłych egzystencjalnie. Chodzi też o~takie sytuacje, w~których nie jest możliwe powstrzymanie się od zajęcia stanowiska, gdyż także powstrzymanie się od zajęcia stanowiska jest już zajęciem jakiegoś stanowiska mającym istotne konsekwencje egzystencjalne. Chodzi zatem o~sytuacje, które z~logicznego punktu widzenia są ,,bez wyjścia'', ale praktyka życia wymusza opowiedzenie się po jednej ze stron, świadomie lub nieświadomie. W~tak rozumianym pragmatycznym sensie Życiński mówi o~konieczności wyboru i~podmiotowego \textit{commitment}: ,,Ten Pascalowski wybór jest koniecznością, w~stosunku do której alternatywą byłby w~nauce sceptycyzm, w~religii -- agnostycyzm''
%\label{ref:RND4OwqxkQrts}(Życiński, 1985, s.~230).
\parencite[][s.~230]{zycinski_teizm_1985}.%


W~takiej pragmatycznej perspektywie teza o~konieczności podmiotowego \textit{commitment} w~wyborze przedzałożeń i~tradycji racjonalistycznej nie ma do końca charakteru obiektywnej zasady metodologicznej, lecz raczej ,,heurystycznej zasady racjonalnego działania''
%\label{ref:RNDkTz9px49om}(Życiński, 1985, s.~211).
\parencite[][s.~211]{zycinski_teizm_1985}. %
 Pozwala uniknąć sceptycyzmu w~sytuacji, gdy argumenty wysuwane przeciw racjonalności i~realizmowi są praktycznie niefalsyfikowalne i~są wynikiem równie ,,arbitralnego'' wyboru. Wszakże ostre przeciwstawienie obiektywnej zasady metodologicznej pragmatycznej zasadzie heurystycznej nie do końca wydaje się właściwe w~sytuacji wyboru podstawowych przedzałożeń i~w~sytuacji, gdy ostre rozróżnienie na kontekst odkrycia i~kontekst uzasadnienia jest nie do utrzymania. U~podstaw nauki i~racjonalności, analogicznie jak u~podstaw rzeczywistości kwantów, znikają proste rozróżnienia na odkrycie, uzasadnienie i~pragmatykę.

Omawiając pragmatyczny aspekt umiarkowanego racjonalizmu Życińskiego warto zwrócić jeszcze uwagę na podobieństwa i~różnice w~stosunku do idei abdukcyjnego wnioskowania do najlepszego wyjaśnienia, w~szczególności w~kontekście prób naturalistycznego uzasadnienia realizmu naukowego
%\label{ref:RNDkkGaTDxl2Z}(zob. Grobler, 2006, s.~102n.265; Liana, 2003).
\parencites[zob.][s.~102n.265]{grobler_metodologia_2006}[][]{liana_naturalistyczne_2003}. %
 Abdukcyjna zasada wnioskowania do najlepszego wyjaśniania mówi, że w~sytuacji posiadania wielu równoważnych empirycznie wyjaśnień jakiegoś zaskakującego zjawiska, naukowiec powinien poszukiwać jego najlepszego wyjaśnienia, i~że tym, co decyduje, że jakieś wyjaśnienie jest najlepsze, jest to, czy znosi ono zaskakujący charakter wyjaśnianego problemu. W~przypadku Życińskiego zaskakującym lub, używając jego ulubionego określania, szokującym zjawiskiem wymagającym wyjaśnienia jest fakt obecności czynników pozaracjonalnych w~nauce. W~sytuacji alternatywy rozłącznej: albo racjonalna akceptacja podmiotowego \textit{commitment}, albo sceptycyzm, Życiński wybiera człon pierwszy i~argumentuje, dlaczego jest to wybór najlepszy.

Zachodzi jednak pewna istotna różnica w~stosunku do tradycyjnego argumentu abdukcyjnego. Także drugi człon powyższej alternatywy znosi zaskakujący charakter wyjściowego problemu. Mamy zatem do czynienia z~obserwacyjną i~psychologiczną równoważnością dwóch wyjaśnień metanaukowych. Uzasadnienie wyboru w~tej sytuacji musi mieć inny charakter niż proste zniesienie elementu szoku. Odpowiedź na pytanie, które z~tych dwóch wyjaśnień lepiej znosi wyjściowy szok, zakłada odwołanie się do założeń określających kryteria ,,lepszości'', a~te zawarte są w~przedzałożeniach determinujących poszczególne alternatywne wyjaśnienia metanaukowe. Argumentacja za ,,lepszością'' będzie zatem z~konieczności kolista. Ponownie zatem powraca kwestia kolistości przedzałożeń.

Całą tę sytuację problemową można zrekonstruować w~jeszcze w~inny sposób. Powyższa interpretacja zachodzi, jeśli przyjmie się, że tym, co szokuje i~wymaga (najlepszego) wyjaśnienia jest fakt nieusuwalnej obecności czynnika pozaracjonalnego w~rozwoju nauki. Gdy jednak przyjmie się, że tym, co bardziej wywołuje szok i~wymaga (najlepszego) wyjaśnienia, jest sam fakt istnienia nauki pomimo obecności czynnika pozaracjonalnego, wówczas sceptycyzm i~antyrealizm faktycznie wydają się na swój sposób wyjaśnieniem ,,cudownym''\footnote{Na temat Putnamowskiej kategorii ‘cudu' w~kontekście idei wnioskowania do najlepszego wyjaśnienia zob.
%\label{ref:RNDvzgFSKZ86M}(Psillos, 1999, s.~70n; Grobler, 2006, s.~265; Liana, 2003, s.~133).
\parencites[][s.~70n]{psillos_scientific_1999}[][s.~265]{grobler_metodologia_2006}[][s.~133]{liana_naturalistyczne_2003}.%
}, by nie rzec ,,cudacznym'', mogącym generować kolejny szok. Problem jednak w~tym, że rozstrzygnięcie, który szok jest bardziej szokujący i~dla kogo, ponownie stawia kwestię wyboru podstawowych przedzałożeń i~hierarchii wartości, prowadząc do nieuniknionej kolistości. Zatem częściowo (koliście) uzasadnialne \textit{commitment} wydaje się z~punktu widzenia racjonalisty najlepszym wyjaśnieniem problemu obecności czynnika subiektywnego w~nauce.

\section{Wiedza osobowa: próba przekroczenia dualizmu subiektywizm -- obiektywizm}
Można bez obawy przesady powiedzieć, że całe rozwiązanie Życińskiego metanaukowej kwestii czynników pozaracjonalnych bazuje na pojęciu podmiotowego \textit{commitment}. W~tym względzie racjonalizm umiarkowany Życińskiego jest jednak bardziej radykalny od \textit{temperate rationalism} Newtona-Smitha. Obaj w~myśl Kuhna, a~przeciw tradycyjnemu racjonalizmowi, starają się potraktować element subiektywny jako część ,,natury nauki'', a~nie tylko jako ułomność ludzkiego poznania. Czynią to jednak odmiennie\footnote{Kuhn
%\label{ref:RNDrvtK0blF4X}(1970, s.~151, por. 2001, s.~263)
\parencites*[][s.~151]{kuhn_structure_1970_liana}[por.][s.~263]{kuhn_struktura_2001} %
 pisze, iż obecność elementu podmiotowego w~wyborze teorii należy uznać raczej za wyraz lub wskaźnik (\textit{index}) natury (\textit{nature}) naukowego badania i~naukowej wiedzy niż za wyraz ludzkiej słabości. Jeszcze dobitniej robi to w~
%\label{ref:RNDfI3BIGFZbl}(Kuhn, 1977, s.~325n, por. 1985, s.~447).
\parencites[][s.~325n]{kuhn_objectivity_1977}[por.][s.~447]{kuhn_obiektywnosc_1985}. %
 Newton-Smith i~Życiński wprawdzie polemizują z~(wczesnym) Kuhnem, ale obaj uznają konieczność wprowadzenia do teorii racjonalności naukowej elementu subiektywnego. W~odniesieniu do Newtona-Smitha zob. np. jego wypowiedź 
%\label{ref:RNDd3J9gJijYm}(Newton-Smith, 1981, s.~270).
\parencite[][s.~270]{newton-smith_rationality_1981}.%
}.

Newton-Smith
%\label{ref:RNDQs0ClezAZo}(1981)
\parencite*[][]{newton-smith_rationality_1981} %
 rozwija koncepcję podmiotowego osądu (\textit{judgment})\footnote{Por. tytuł artykułu Kuhna 
%\label{ref:RNDaUYqVFjuOY}(1977).
\parencite*[][]{kuhn_objectivity_1977}.%
} i~jego konieczności w~rozwoju nauki. Zdolność osądu definiuje jako zdolność podejmowania decyzji w~sytuacji braku możliwości wyartykułowania uzasadnienia w~\textit{danym} czasie 
%\label{ref:RNDQEF9JTftzd}(Newton-Smith, 1981, s.~234).
\parencite[][s.~234]{newton-smith_rationality_1981}. %
 W~jego przekonaniu dotychczasowe racjonalistyczne rozwiązania Poppera, Lakatosa i~Laudana nie uwzględniały tego elementu osądu. Newton-Smith stawia sobie za cel ukazanie dynamicznego modelu nauki, w~którym oprócz teorii zmieniają się także metody i~reguły metodologiczne. Idea osądu pozwala mu racjonalnie domknąć rozwiązanie metanaukowego problemu wyboru jednej spośród alternatywnych i~wykluczających się metod rozwoju nauki 
%\label{ref:RNDBlYBJ93707}(Newton-Smith, 1981, s.~270).
\parencite[][s.~270]{newton-smith_rationality_1981}. %
 Swe analizy osądu ogranicza on jednak do wskazania miejsc, w~których tego typu podmiotowy osąd okazuje się nieeliminowalny bez wchodzenia w~analizę logiczną samego osądu\footnote{Są to kolejno: decyzje językowe, sytuacje eksperymentalne wymagające określonych sprawności (\textit{skills}) intelektualnych, oraz sytuacje wyboru jednej spośród alternatywnych teorii, także metodologicznych 
%\label{ref:RNDse6ZZ1x916}(zob. Newton-Smith, 1981, s.~232–235).
\parencite[zob.][s.~232–235]{newton-smith_rationality_1981}.%
}. Ogranicza się do wykazania możliwości racjonalnego charakteru nauki w~sytuacji występowania w~niej nieusuwalnego elementu subiektywnego osądu. Rolę czynnika racjonalizującego podmiotowe osądy pełni w~jego koncepcji predykcyjny sukces takiego osądu osiągany w~dłuższej perspektywie czasowej. Tego rodzaju sukces ,,kontroluje ewolucję innych czynników na zasadzie mechanizmów akcji i~reakcji''\footnote{Newton-Smith 
%\label{ref:RND6ILXJTaLyA}(1981, s.~270):
\parencite*[][s.~270]{newton-smith_rationality_1981}: %
 ,,\textit{long-range predictive success}''. Niemniej nadmierny sukces może doprowadzić do ryzykowanej sytuacji, w~której naukowiec nazbyt uwierzy w~genialność swoich własnych ,,sądów intuicyjnych'', zob. 
%\label{ref:RND6yc9MkcDNt}(Newton-Smith, 1981, s.~234n).
\parencite[][s.~234n]{newton-smith_rationality_1981}.%
}. Z~jego perspektywy interesujące jest zatem tylko to, że elementy subiektywne mają charakter tymczasowy i~że z~czasem pojawia się racjonalne uzasadnienie pierwotnie intuicyjnych osądów i~decyzji. Problem przedzałożeń, tak ważny dla Życińskiego, w~jego koncepcji zajmuje miejsce marginalne. Wprawdzie wspomina o~nich omawiając ontologiczne \textit{commitments} realizmu 
%\label{ref:RNDOg5R14exLD}(Newton-Smith, 1981, s.~38),
\parencite[][s.~38]{newton-smith_rationality_1981}, %
 ale nie zajmuje się ich epistemologiczną funkcją i~analizą towarzyszącego im \textit{commitment}.

Dla Życińskiego problem wyboru przedzałożeń stanowi centralne miejsce w~jego koncepcji racjonalizmu umiarkowanego. W~konsekwencji nie tyle interesuje go wykazanie racjonalności w~długim okresie czasu, ile jej zakorzenienie w~\textit{hic et nunc} podmiotowego wyboru przedzałożeń. Rozwiązanie problemu racjonalności w~dłuższym okresie czasu stanowi dla niego jedynie konsekwencję rozwiązania problemu podmiotowego \textit{commitment}. Traktuje on element podmiotowy znacznie bardziej fundamentalnie od Newtona-Smitha i~wprowadza go znacznie głębiej do natury nauki i~racjonalności\footnote{Różnica w~podejściu wynika zapewne z~odmiennych celów, jakim podporządkowują oni swe uprawianie filozofii. Dla Życińskiego celem jest wykazanie możliwej racjonalności wiary religijnej i~teizmu. Filozofia nauki dostarcza w~tym wypadku jedynie niezbędnych narzędzi pozwalających wykazać arbitralność pozytywistycznego przeciwstawienia wiary i~rozumu. Dla Newtona-Smitha celem jest sama filozofia nauki.}.

Swoje rozumienie elementu podmiotowego występującego w~nauce i~racjonalności Życiński rozwija wykorzystując pojęcie \textit{wiedzy osobowej} Polanyia
%\label{ref:RNDHjmbGZoXxp}(1962)
\parencite*[][]{polanyi_personal_1962}%
\footnote{Życiński rozwija swoje pojęcie wiedzy osobowej w~
%\label{ref:RNDVhdEuu50Zr}(Życiński, 1996, s.~179–187, 2015, s.~243–254).
\parencites[][s.~179–187]{zycinski_elementy_1996}[][s.~243–254]{zycinski_elementy_2015}.%
}. Polanyi definiuje \textit{wiedzę osobową} za pomocą kategorii ‘\textit{commitment}': jest ona ,,intelektualnym \textit{commitment}'' 
%\label{ref:RNDv4LqKNaqwL}(Polanyi, 1962, s.~IV).
\parencite[][s.~IV]{polanyi_personal_1962}. %
 Z~kolei \textit{commitment} to swoiste ,,zawierzenie'' (\textit{reliance}) jakiemuś narzędziu, także intelektualnemu, warunkujące możliwość jego użycia. Tego typu zawierzenie stanowi istotny element sprawności (\textit{skills}) w~używaniu narzędzi 
%\label{ref:RNDdJyvJNqn6i}(Polanyi, 1962, s.~63).
\parencite[][s.~63]{polanyi_personal_1962}. %
 Dla Życińskiego bardziej istotne jest jednak stwierdzenie Polanyia, że wiedza osobowa będąc wiedzą podmiotową nie jest ani czysto subiektywna, ani czysto obiektywna, lecz ,,transcenduje'' poza tradycyjny dualizm podmiotu i~przedmiotu. W~ten sposób kategorie ‘\textit{commitment}' i~‘wiedzy osobowej' wpisują się w~program Życińskiego pokonania dualizmu internalizm -- eksternalizm\footnote{Por. cytat z~\textit{Personal Knowledge} Polanyia podany przez Życińskiego 
%\label{ref:RNDd6vFMbp9p3}(Życiński, 1996, s.~183, przypis 276, 2015, s.~249, przypis 11):
\parencites[][s.~183, przypis 276]{zycinski_elementy_1996}[][s.~249, przypis 11]{zycinski_elementy_2015}: %
 ,,In so far as the personal submits to requirements acknowledged by itself as independent of itself, is not subjective; but in so far as it is an (\textit{sic}!) action guided by individual passions, it is not objective either. It transcends the disjunction between subjective and objective.'' W~obu tekstach Życińskiego zamiast ‘an' jest błędnie ‘not'. W~cytowaniu Życińskiego zachodzi dodatkowo niezgodność stron. Strona, na którą podaje Życiński: s.~300, jest w~wydaniu z~1962~r. (Routledge) -- na które się powołuje w~\textit{Elementach} -- stroną 316. Strona 300 występuje natomiast w~wydaniu Chicagowskim z~1958~r. i~w~kolejnych.}.

W~\textit{Elementach filozofii nauki}
%\label{ref:RNDcq0YHvC6fb}(1996, 2015)
\parencites*[][]{zycinski_elementy_1996}[][]{zycinski_elementy_2015} %
 Życiński podejmuje próbę wyartykułowania sposobu transcendowania wiedzy osobowej poza tradycyjnie rozumianą subiektywność i~obiektywność. Zauważa, że specyfika epistemologiczna wiedzy osobowej tkwi w~osobowym \textit{commitment} naukowca \textit{z~tradycją badawczą}. To ostatnie dopełnienie jest kluczowe. Podmiotowe związanie z~tradycją samo w~sobie zawiera istotny element pozasubiektywny, gdyż uczony wiąże się z~czymś, co z~definicji jest obiektywne w~sensie Popperowskim. Tradycja badawcza rozumiana realistycznie zajmuje się twierdzeniami posiadającymi określoną zawartość treściową (obiektywną). Spory między różnymi tradycjami badawczymi dotyczą takiej obiektywnej treści twierdzeń i~teorii. W~przypadku osobowego wyboru i~\textit{commitment} zasad akceptowanych uniwersalnie i~w sposób konieczny warunkujących możliwość jakiejkolwiek refleksji intersubiektywnej, pozasubiektywny charakter osobowego wyboru i~\textit{commitment} wydaje się bezsporny. Ale nawet wówczas, gdy uczony wybiera wzorce interpretacji, które nie są akceptowane uniwersalnie, tylko przez pewną grupę, to jego wybór i~wiedza osobowa także są ,,prawomocne'', czyli nie-subiektywne. Dlaczego? Ponieważ, zdaniem Życińskiego, jest to zawsze wybór spośród wyjaśnień ,,alternatywnych'' i~,,współistniejących'' 
%\label{ref:RNDX4HH5Qz2IK}(Życiński, 1996, s.~184, 2015, s.~250)
\parencites[][s.~184]{zycinski_elementy_1996}[][s.~250]{zycinski_elementy_2015}%
\footnote{Widać tutaj wyraźnie różnicę z~koncepcją Newtona-Smitha.} w~tym samym czasie w~różnych tradycjach badawczych. Otóż wielość współistniejących tradycji badawczych nie jest wynikiem irracjonalnego woluntaryzmu, lecz logiczną (konieczną) konsekwencją metody prób i~błędów w~nauce oraz obiektywnej złożoności sytuacji badawczych\footnote{Jako przykłady takiej wiedzy osobowej, Życiński podaje konieczność wyboru między różnymi tradycjami ewolucjonizmu, oraz kłopoty z~ustaleniem definicji gatunku biologicznego 
%\label{ref:RNDZHuxvpo6X7}(zob. Życiński, 1996, s.~184n, 2015, s.~250nn).
\parencites[zob.][s.~184n]{zycinski_elementy_1996}[][s.~250nn]{zycinski_elementy_2015}.%
}. To inne ujęcie zasady niedookreśloności teoretycznej, metodologicznej i~presupozycyjnej. Konieczność podmiotowego wyboru jednej spośród takich alternatywnych i~równoważnych tradycji jest wymuszona sytuacją obiektywną, sam wybór zatem nie jest czysto arbitralnym aktem. Rzeczywista nauka ze swej natury wymaga osobowego wyboru, by w~ogóle istnieć i~by się rozwijać. Zarówno natura obserwacji empirycznej, jak i~swoisty brak domknięcia formy logicznej języka teoretycznego wymuszają na naukowcu podejmowanie różnych decyzji, w~tym tych najbardziej fundamentalnych w~oparciu o~\textit{commitment}. Tym samym nadają tak motywowanym wyborom charakter pozasubiektywny.

Można więc powiedzieć, że w~tej mierze, w~jakiej \textit{commitment} jest obiektywnie konieczne, jest też pozasubiektywne i~ma charakter epistemologiczny. Sytuacja zmienia się jednak zasadniczo, gdy pod uwagę bierzemy motywacje skłaniające naukowca do związania się z~\textit{tą a~nie inną} tradycją. Według Życińskiego, który podąża tutaj za Polanyi'em, konkretny wybór dokonuje się zawsze w~oparciu o~intuicyjne, a~zatem typowo subiektywne odczucia, i~pozakonceptualne wartościowania:

\myquote{
Czynnikiem decydującym o~związaniu badacza z~określoną tradycją będzie w~podobnym przypadku element intuicyjnych odczuć czy pozakonceptualnych wartościowań stanowiących wiedzę osobową
%\label{ref:RNDlQqxUeraGa}(Życiński, 1996, s.~185, 2015, s.~252)
\parencites[][s.~185]{zycinski_elementy_1996}[][s.~252]{zycinski_elementy_2015}%
\footnote{Stwierdzenie to jest w~pełni zgodne z~zacytowanymi wyżej w~przypisie 76. słowami M. Polanyia.}.
}

Próbą wyjaśnienia takiego paradoksalnego statusu epistemologicznego wiedzy osobowej mogłoby być rozróżnienie w~wiedzy osobowej poszczególnych osób dwóch elementów ,,materialnego'' i~,,formalnego''. Elementem materialnym byłaby w~tym wypadu subiektywna, jednostkowa i~intuicyjna treść konkretnej wiedzy osobowej; elementem formalnym natomiast pozasubiektywne i~swoiście transcendujące podmiot odniesienie konkretnej wiedzy osobowej do obiektywnej sytuacji wyboru. To samo można wyrazić w~języku metateoretycznym. Istnieją dwa różne, aczkolwiek powiązane ze sobą, znaczenia terminu ‘wiedza osobowa'. W~użyciu referencyjnym i~realnym oznacza on pewne \textit{konkretne} stany umysłu indywidualnych osób, tylko im właściwe intuicje, poglądy, upodobania etc. Z~kolei w~użyciu epistemologicznym jest to nazwa ogólna nazywająca to, co \textit{wspólne} wszystkim indywidualnym \textit{wiedzom osobowym} (\textit{sit venia verbo}!), to znaczy konieczne odniesienie każdej konkretnej wiedzy osobowej do (możliwej i/lub rzeczywistej) obiektywnej sytuacji koniecznego wyboru\footnote{Przykładowo, w~znaczeniu pierwszym, materialnym, wiedza osobowa Einsteina jest całkowicie różna od wiedzy osobowej Newtona. W~znaczeniu drugim, formalnym, wiedza osobowa Newtona i~wiedza osobowa Einsteina ujawniają pewną wspólną cechę, dzięki której ich stany umysłowe nazywane są za pomocą tej samej kategorii epistemologicznej (doksalogicznej) ‘wiedza osobowa'.}. W~perspektywie realistycznej byłby to swoisty powrót do idei Kanta w~nowych okolicznościach racjonalizmu umiarkowanego.

Wiedza osobowa jest zatem tym miejscem, w~którym czynnik tradycyjnie określany jako \textit{zewnętrzny} odgrywa decydującą rolę. Z~tym, że w~nowym, doksatycznym ujęciu racjonalności i~epistemologii nie można go już dalej nazywać po prostu czynnikiem \textit{zewnętrznym}. Stał się bowiem integralną częścią podstaw racjonalności i~koniecznym warunkiem możliwości rozwoju nauki.

\begin{center}
$ {\ast}\,{\ast}\,{\ast} $
\end{center}

Istotna rola czynnika pozaracjonalnego u~podstaw racjonalności nie oznacza jednak, że czynnikowi temu można pozwalać ,,na zbyt wiele'' w~samej nauce i~w jej wyjaśnianiu. Podobnie jak w~fizyce, gdzie zasada nieoznaczoności obowiązuje jedynie w~świecie kwantów, ale już nie w~świecie makroskopowym, tak samo w~\textit{doxalogii} Życińskiego \textit{commitment} i~intuicja odrywają istotną rolę w~ustalaniu podstaw racjonalności naukowej, ale już nie w~jej historycznej artykulacji. Tutaj ich rola przestaje być istotna i~decydująca, a~staje się, na mocy wyboru, drugorzędna. Wybór przedzałożeń racjonalności i~realizmu narzuca wymóg, by naukę uprawiać i~wyjaśniać zasadniczo racjonalnie i~realistycznie. To właśnie taki a~nie inny wybór przedzałożeń pozwala Życińskiemu zarzucić irracjonalizm koncepcji Toulmina traktującej paradygmaty w~kategoriach gier językowych i~wskazać jako przyczynę to, iż przypisuje ona ,,nieproporcjonalnie wielką wagę pozaracjonalnym składnikom nauki'' w~jej wyjaśnianiu
%\label{ref:RNDEXl0Wwkdiy}(Życiński, 1996, s.~202, 2015, s.~275).
\parencites[][s.~202]{zycinski_elementy_1996}[][s.~275]{zycinski_elementy_2015}.%


Wybór przedzałożeń usprawiedliwia również i~zarazem narzuca wymóg wypracowywania takich elementów metanaukowych, które będą podkreślały zasadniczo racjonalny charakter wyborów teoretycznych i~rozwoju nauki. W~ujęciu Życińskiego takimi elementami są: zasada aracjonalności, zasada naturalności interdyscyplinarnej, koncepcja filozoficznych tradycji badawczych oraz ideatów, koncepcja międzyparadygmatycznej współmierności, czy w~końcu koncepcja racjonalności obiektywnej i~jej rozwoju jako filozoficznego ideatu. Ale te zagadnienia, także ze względu na swą obszerność, muszą stać się przedmiotem innego opracowania.

\section{Zakończenie}
Koncepcja Życińskiego ma ponad trzydzieści lat. Z~tego punktu widzenia jest to koncepcja historyczna. Niemniej problem obecności czynnika pozaracjonalnego w~rozwoju nauki jest nadal aktualny i~daleki od ostatecznego rozwiązania. Konieczność metanaukowej racjonalizacji tego czynnika dostrzegli nie tylko obrońcy tradycji realistycznej w~filozofii nauki, tacy jak Życiński, lecz także zwolennicy tradycji antyrealistycznej, tacy jak Bas van Fraassen. Jego praca \textit{The Empirical Stance}
%\label{ref:RNDQijT0UUuvo}(2002)
\parencite*[][]{van_fraassen_empirical_2002} %
 zawiera woluntarystyczne rozwiązanie tego problemu oraz powiązanego z~nim problemu pluralizmu tradycji badawczych lub, w~jego własnej terminologii, \textit{stances}, stanowisk lub postaw\footnote{Pojęcie to wyjaśnia Teller 
%\label{ref:RND1sjSzsaVUG}(2004).
\parencite*[][]{teller_discussion_2004}. %
 Pojęcie \textit{epistemic stance} rozwija z~kolei Chakravartty 
%\label{ref:RNDALw8ekf9iV}(2011).
\parencite*[][]{chakravartty_puzzle_2011}.%
}. Wywołała ona szeroką dyskusję wśród filozofów nauki na temat woluntaryzmu epistemicznego i~epistemologicznego i~jego możliwych konsekwencji. Dyskusja ta skutkowała z~kolei wydaniem specjalnego numeru \textit{Synthese} (2011), zawierającego kilkanaście tekstów, z~których wiele porusza %
zagadnienie \textit{commitment} w~kontekście wyboru zarówno konkretnych twierdzeń, jak i~podstawowych stanowisk filozoficznych, w~tym samego woluntaryzmu. Pokazuje ona, że zwolennicy antyrealizmu mają świadomość zgubnych konsekwencji, jakie mogą być następstwem nieusuwalnej obecności czynnika pozaracjonalnego u~podstaw wszelkich przekonań (\textit{beliefs}) w~połączeniu z~ideą metodologicznej równoważności alternatywnych stanowisk. Według van Fraassena
%\label{ref:RNDhYqXruiOPa}(2011, s.~158)
\parencite*[][s.~158]{van_fraassen_stance_2011} %
 konsekwencje te mogą okazać się autodestrukcyjne zarówno dla głoszonego przezeń woluntaryzmu, jak i~dla samej demokracji.

Życiński wypowiada się negatywnie o~woluntaryzmie (zob. wyżej przypis 31), ale ma on na myśli inny typ woluntaryzmu, woluntaryzmu dotyczącego wyboru samego \textit{commitment}, a~nie wyboru pomiędzy alternatywnymi i~równoważnymi tradycjami. Z~tego ostatniego punktu widzenia także jego rozwiązanie jest woluntarystyczne w~sensie van Fraassena, gdyż o~wyborze alternatywnych i~równoważnych tradycji nie decydują w~ostateczności przesłanki racjonalne, lecz czynnik subiektywny, pozaracjonalny, w~tym także wolitywny (zob. wyżej punkt 1.c.). Wraz z~van Fraassenem podziela on ideę racjonalności lokalnej, według której także sceptyk i~dogmatyk są w~pewnym minimalnym sensie osobami racjonalnymi mogącymi bronić swych stanowisk bez końca (zob. powyżej punkty 1.b. i~1.c.).

Otóż van Fraassen
%\label{ref:RNDs1mcs7rJST}(2011, s.~157n)
\parencite*[][s.~157n]{van_fraassen_stance_2011} %
 skłania się do tezy, że wyartykułowane \textit{commitment} przestaje nim być, gdyż w~artykulacji zawsze zachodzi jakaś alienacja w~stosunku do przedmiotu \textit{commitment}. Prowadzi go to pesymistycznych konkluzji o~możliwym autodestrukcyjnym charakterze świadomego woluntaryzmu metanaukowego w~dłuższej perspektywie czasowej\footnote{Jeszcze dobitniej obawy te artykułuje 
%\label{ref:RNDXKMOmLbVW3}(Teller, 2011, s.~65).
\parencite[][s.~65]{teller_learning_2011}.%
}. Woluntarysta nie będzie bronił skutecznie poglądów, do których nie może być autentycznie przywiązany, o~ile chce pozostać wierny swym przekonaniom o~metodologicznej równoważności odmiennych tradycji. W~pluralistycznym świecie tradycji metanaukowych i~\textit{stances} nie ma żadnych gwarancji \textit{a~priori} przetrwania tradycji naukowej i~racjonalnej.

Analiza koncepcji \textit{commitment} Życińskiego z~tej perspektywy pozwoliłaby zobaczyć, czy jego umiarkowany racjonalizm dysponuje skutecznymi narzędziami obrony przed autodestrukcją, o~której mówi van Fraassen. Inaczej mówiąc, by zobaczyć, czy racjonalizm umiarkowany może sprawdzić się jako trwały światopogląd kształtujący postawy i~zachowania, a~nie pozostać jedynie czystą teorią. Tradycja badawcza, jeśli chce przetrwać, musi umieć przekonywać do siebie członków różnych alternatywnych tradycji. Wyartykułowanie racjonalnego i~realistycznego \textit{commitment} powinno dać się pogodzić z~zaangażowaniem w~ich obronie. Wydaje się jednak, że Życiński nie podziela obaw van Fraassena przed radykalną alienacją związaną z~artykulacją \textit{commitment}. Odwołuje się do idei \textit{wiary}, która nie boi się artykulacji, przeciwnie, w~niej się najpełniej realizuje.

Realistyczna koncepcja ,,woluntaryzmu'' Życińskiego stanowi istotną alternatywę dla koncepcji van Fraassena\footnote{Innym realistycznym rozwiązaniem jest realistyczna ,,rewizja'' siateczkowego modelu racjonalności Laudana zaproponowana przez Groblera
%\label{ref:RNDDX5I0AgSwI}(1993, s.~35nn).
\parencite*[][s.~35nn]{grobler_prawda_1993}. %
 Trudno jednak byłoby zakwalifikować ją jako rozwiązanie woluntarystyczne w~rozumieniu van Fraassena.}. Nie tylko pokazuje, że antyrealizm teoretyczny nie jest jedynym możliwym wyjściem z~epistemologicznego i~metodologicznego impasu wywołanego przez fakt niedookreślenia teorii przez obserwację\footnote{Warto też zauważyć, że fakt ten jest przedmiotem ciągłych dyskusji wśród filozofów nauki i~daleko do pełnej zgody 
%\label{ref:RNDW94Udw94eD}(zob. Stanford, 2017).
\parencite[zob.][]{stanford_underdetermination_2017}.%
} i~przez problem czynników pozaracjonalnych, lecz ukazuje również potencjalności pojęcia \textit{commitment} niedostrzegane przez przekonanego antyrealistę. Choć zapewne z~punktu widzenia antyrealisty takie związanie z~ideą prawdy lub prawdopodbieństwa jest jedynie przejawem irracjonalnej metafizycznej wiary 
%\label{ref:RNDPSErbpq1CL}(Chakravartty, 2011, s.~41n).
\parencite[][s.~41n]{chakravartty_puzzle_2011}.%


Zagadnienie \textit{commitment} posiada również interesujące aspekty kognitywistyczne, pomijane całkowicie przez Życińskiego. Teller
%\label{ref:RNDkydTdwxCLk}(2011, s.~65n),
\parencite*[][s.~65n]{teller_learning_2011}, %
 zwolennik woluntaryzmu van Fraassena, analizuje prospołeczną funkcję \textit{commitment} jako swoistego antidotum na pesymizm van Fraassena. Naturalistyczny argument Tellera można wykorzystać dla wzmocnienia tezy Życińskiego o~koniecznym charakterze \textit{commitment} w~kształtowaniu się i~trwaniu intersubiektywnego rozumu i~nauki. Nie mniej ważna wydaje się też idea \textit{common ground}, o~której mówi Teller. Chodzi o~wspólną platformę porozumienia przedstawicieli różnych tradycji i~różnych epistemologicznych i~aksjologicznych \textit{commitments}. Nawet społeczność multitradycyjna musi posiadać jakieś narzędzia dochodzenia do minimalnego niezbędnego konsensu. Teller zdaje się dostrzegać taki \textit{common ground}, podobnie jak Thomas Kuhn w~\textit{Postscriptum} 
%\label{ref:RNDF6tAx10kNI}(2001, s.~346n),
\parencite*[][s.~346n]{kuhn_struktura_2001}, %
 w~najgłębiej osadzonych \textit{commitments} gatunku ludzkiego. Pytanie, jakie miejsce w~zbiorze takich podstawowych \textit{commitments} zajmują przywiązania do idei realizmu i~racjonalności?

Na koniec o~pewnej pułapce językowej, jaka czyha na zwolenników racjonalizmu umiarkowanego. Według Życińskiego w~nowej koncepcji racjonalności konieczne jest przezwyciężenie tradycyjnych dualizmów pojęciowych, jakie rządzą myśleniem filozoficznym od jego greckich początków. Niezbędne i~konieczne w~poznaniu przedmiotowym, okazują się zbyt uproszczonymi narzędziami do opisu i~analizy podstawowych przedzałożeń tradycji badawczych. Tutaj właśnie tkwi pułapka językowa. Łatwo jest zanegować deklaratywnie tradycyjne dualizmy, o~wiele trudniej wypracować nie-dualistyczny język, w~którym można by precyzyjnie odróżniać różne tradycje badawcze. Pytanie, czy w~ogóle jest to możliwe na poziomie języka metanauki, na poziomie pojęć jednoznacznych i~wyraźnych. Także Życiński posługuje się pojęciami dualistycznymi, by móc odróżnić swoje koncepcje od koncepcji skrajnie sceptycznych czy antyrealistycznych. W~takim dualistycznym znaczeniu używa on na przykład określenia ‘woluntaryzm' (zob. wyżej przypis 31 i~tekst) i~to pomimo uznania istotnej roli woli w~akcie związania się z~tradycją. Z~kolei \textit{woluntaryzm} w~rozumieniu van Fraassena jest, jak widzieliśmy, jedynie jedną z~alternatywnych prób pokonania tradycyjnego dualizmu \textit{episteme} i~\textit{skepsis}. Jak się wydaje, zwolennicy racjonalizmu umiarkowanego skazani są na język dualistyczny, muszą zatem konsekwentnie rozróżniać użycie kwalifikowane od użycia absolutnego terminów obciążonych dualistyczną tradycją.

\end{artplenv}\label{liana-ende}