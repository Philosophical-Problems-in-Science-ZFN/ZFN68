\begin{artengenv}{Jerzy Dadaczyński, Robert Piechowicz}
	{Maxime Bôcher's concept of complementary philosophy of mathematics}
	{Maxime Bôcher's concept of complementary philosophy of mathematics}
	{Maxime Bôcher's concept of complementary philosophy\\of mathematics}
	{Pontifical University of John Paul II in Krakow}
	{The main purpose of the present paper is to demonstrate that as early as 1904 pre-eminent American mathematician Maxime Bôcher was an adherent to the presently relevant argument of reasonableness, or even necessity of parallel development of two philosophical methods of reflection on mathematics, so that its essence could be more fully comprehended. The goal of the research gives rise to the question: what two types of philosophical deliberation on mathematics were proposed by Bôcher?}
	{philosophy, mathematics, Maxim Bôcher, metamathematics.}
	




\section{Introduction}
\lettrine[loversize=0.13,lines=2,lraise=-0.05,nindent=0em,findent=0.2pt]%
{T}{}he main purpose of the present paper is to demonstrate that as early as 1904 pre-eminent American mathematician Maxime Bôcher\footnote{Maxime Bôcher (1867-1918) was a son of an American scientist of French origin. In the years 1883-1888 he studied, \textit{inter alia}, mathematics at Harvard. In the years 1888-1891, in Göttingen he was a student of Klein's, under the direction of whom he wrote and defended his doctoral dissertation. After his return to the USA he was an assistant professor, and as of 1904 a professor of mathematics at Harvard. In the years 1909-1910 he served as president of the American Mathematical Society. Bôcher specialised chiefly in mathematical analysis, and his most famous paper is \textit{Introduction to Higher Algebra}
%\label{ref:RNDc7ZxUmWWTi}(Bôcher, 1907).
\parencite[][]{bocher_introduction_1907}.
To honour the memory of this mathematician, who died prematurely, Bôcher Memorial Prize was endowed; it is awarded at intervals of several years for achievements in the field of mathematical analysis in the period of the six years leading up to the prize-giving year. One of the prize-winners was von Neumann (in the 1930s).} was an adherent to the presently relevant argument of reasonableness, or even necessity of parallel development of two philosophical---complementary---methods of reflection on mathematics, so that its essence could be more fully comprehended.

Thus outlined, the main goal of the research gives rise to the question: what two---complementary---types of philosophical deliberation on mathematics were proposed by Bôcher? They were ``adumbrated'' in the text entitled \textit{The Fundamental Conceptions and Methods of Mathematics}
%\label{ref:RNDNx8Z5TUKyi}(Bôcher, 1904).
\parencite[][]{bocher_fundamental_1904}.


In the first, principal part of his text---which has already been analysed in
%\label{ref:RNDBGQjug7NKV}(Dadaczyński, 2015)
\parencite[][]{dadaczynski_tendencje_2015}---Bôcher made a direct reference to logicism, proposing its corrections, so that the reductionist programme could be appropriately realised.\footnote{Bôcher was doubtful as to the possibility of realising the programme of logicism the way it had been delineated by Frege and Russell. His objection was that logic---as viewed by the originators of logicism---was too ``weak'' to ``produce'' an infinite set of natural numbers. Bôcher's predictions proved right while work on \textit{Principia Mathematica} was underway: it was necessary to ``add'' some form of an axiom of infinity---in the ontological version providing for the existence of an infinitude of individuals---to obtain natural numbers (of any type). Such an axiom did not ``fit within'' Frege's logic or Russell's ``starting'' logic. The American mathematician formulated his objections as follows: ``Nevertheless, in view of the fact that the system of finite positive integers is necessary in almost all branches of mathematics (we cannot speak of a triangle or a hexagon without having the numbers three and six at our disposal), it seems extremely desirable that the system of logical principles which we lay at the foundation of all mathematics be assumed, if possible, broad enough so that the existence of positive integers---at least finite integers---follows from it; and there seems little doubt that this can be done in a satisfactory manner. When this has been done we shall perhaps be able to regard, with Russell, pure mathematics as consisting exclusively of deductions by ‘logical principles from logical principles'' 
%\label{ref:RNDHPWwCxIM4I}(Bôcher, 1904, p.132).
\parencite[][p.132]{bocher_fundamental_1904}.%
} He made an \textit{explicite} demand for future demonstration of non-contradiction of all mathematics liberated---and such liberation he anticipated---of known antinomies, and not just ``partial'' demands for proof of non-contradiction of the arithmetic of real numbers 
%\label{ref:RNDwQcEhP4x06}(Hilbert, 1900),
\parencite[][]{hilbert_uber_1900},
or the arithmetic of natural numbers 
%\label{ref:RNDQN0fkQatZT}(Hilbert, 1905).
\parencite[][]{hilbert_uber_1905}.
In a sense---which was demonstrated in the paper 
%\label{ref:RND8LFBlHFrCY}(Dadaczyński, 2015)
\parencite[][]{dadaczynski_tendencje_2015}---the first part of Bôcher's text contains, \textit{inter alia}, an ``outline,'' though an unordered one, of the 1904 state of philosophy of mathematics confederated with the study of its foundations.\footnote{The passage below serves to bring back the conclusions from the paper 
%\label{ref:RNDLccZyjKs8M}(Dadaczyński, 2015).
\parencite[][]{dadaczynski_tendencje_2015}.
 In 1904 logicism was already a well-developed direction in the study of the foundations of mathematics. Bôcher could see its intrinsic difficulties concerned with the establishment of antinomies, but he was sure that corrections of the logic underlying the foundations of mathematics would make it possible to remove those difficulties. Hence, in a sense, he ``announced'' the ``second'' logicism---the one of Russell's. Two points excerpted from the 1904 text foreshadowed Hilbert's formalism of the 1920s. They are, firstly, methodical nominalism, and, secondly, posing a question whether it is possible to establish---having removed known antinomies---a global non-contradiction of mathematics. The latter issue was Bôcher's clear reference to Hilbert's second problem (non-contradiction of the arithmetic of real numbers) and Hilbert's first attempts at proving the non-contradiction of the arithmetic of natural numbers. The positive solution to the problem of the global non-contradiction of mathematics---which was emphasised by, \textit{inter alia}, Bôcher---later on became the goal of the formalist research programme, the attainment of which being the rationale for the emergence of metamathematics. Methodological nominalism became Hilbert's answer to the question about semantics of the fully axiomatised and formalised classical mathematics. Bôcher's text of 1904 negatively treats Kantianism as the foundation on which to construct mathematics. Nevertheless, it is a source that confirms that Kantianism was still relevant in many mathematical milieus in the early 20\textsuperscript{th} century. That was the ground on which, several years after the publication of Bôcher's text and following the adoption of appropriate corrections and modifications, intuitionism emerged.} Noteworthily, Bôcher termed his own way of deliberating over mathematics, which he presented in the first, principal part of his work, ``discussion [...] of the so-called foundations'' 
%\label{ref:RNDTfXSIWcrjs}(Bôcher, 1904, pp.132–133),
\parencite[][pp.132–133]{bocher_fundamental_1904},
 which focuses on the deductive aspect of mathematics. His---to use his own term---``discussion [...] of the so-called foundations'' of mathematics, in which he develops some of his own threads, \textit{inter alia}, attempting to specify what mathematics is, is a kind of reflection markedly determined by the way (and ``contents'') of Frege's, Russell's and Hilbert's deliberation on mathematics.

Since the first way of deliberation on mathematics, which was proposed by Bôcher, has already been subjected to analysis and identified, demonstrating that his metaphilosophy contained a demand for inclusion of the two---complementary---types of philosophy of mathematics becomes reduced to showing that Bôcher was outlining an alternative---though naturally not in the sense of an exclusive alternative---concept of philosophical reflection on mathematics, which in his opinion was to complement philosophy confederated with the study of the foundations of mathematics.

With a view to the attainment of the above-indicated research objective, in the first place, presentation will be made of the main outlines of \textit{quasi}-empiricism---philosophy of mathematics initiated with Lakatos' publications of the first half of the 1960s. The fundamental imperatives of \textit{quasi}-empiricism shall serve here as points of reference for analysis and identification of the content of the alternative variant of philosophy of mathematics outlined by Bôcher.

This place calls for a crucial remark. Originated with Lakatos' texts, \textit{quasi}-empiricism strongly emphasised some threads of philosophy of mathematics which, although constantly present in deliberation on mathematics, for more than half a century were ``suppressed'' by study concerned with foundations (logicism, formalism and intuitionism). Mention will be made here of the prehistory of \textit{quasi}-empiricism, and one of partial propositions will be to show that Bôcher should be reckoned among its representatives. The choice of \textit{quasi}-empiricism as a point of reference for the study of an alternative thread of Bôcher's philosophy of mathematics may seem to be an ahistorical approach exactly because it does not take account of the above-mentioned chronology. However, the above measure has been chosen, because \textit{quasi}-empiricism is a popularly known version of philosophy of mathematics as an alternative to philosophy confederated with the study of foundations, and a reference to it will make it relatively easy to pinpoint the ``locus'' of the other way of deliberation on mathematics---presented by Bôcher---on the ``map'' of variants of philosophy of mathematics.

\section{An outline of \textit{quasi}-empiricism}
The beginnings of quasi-empiricism are related to a four-part paper entitled \textit{Proofs and Refutations}
%\label{ref:RNDlCDsZHJWUQ}(Lakatos, 1963),
\parencite[][]{lakatos_proofs_1963},
 which Lakatos published in ``The British Journal of the Philosophy of Science'' in 1963-1964. A coherent description of the direction originated with Lakatos' publications is not an easy task. The reason may be that it is not so much about a uniform direction in the philosophy of mathematics, as about a wave of---essentially uncoordinated---reactions to the way philosophy of mathematics had been pursued before.

It was Tymoczko who attempted a certain characterisation which consisted in enumerating some tendencies, not necessarily common to all the representatives of the ``new'' philosophy of mathematics. Amongst the manifestations of the ``new'' philosophy pointed out by Tymoczko, he mentions: focusing attention on informal proofs in mathematics and on explanation which is juxtaposed with formal proofs, the use of computers to prove mathematical theorems, the issues concerned with errors in mathematics, drawing attention to the history of mathematics, and in particular to the reconstruction of essential discoveries, addressing the issue of communication within the community of mathematicians.\footnote{,,If we look at mathematics without prejudice, many features will stand out as relevant that were ignored by the foundationalist: informal proofs, historical development, the possibility of mathematical error, mathematical explanations (in contrast to proofs), communication among mathematicians, the use of computers in modern mathematics, and many more''
%\label{ref:RNDYOKoeMEYyv}(Tymoczko, 1986, p.xvi).
\parencite[][p.xvi]{tymoczko_introduction_1986}.%
}

It appears that the manifestations of \textit{quasi}-empiricism mentioned by Tymoczko should also necessarily include two aspects of the new philosophy of mathematics which in their first presentation were strongly emphasised by I. Lakatos.

Firstly, it is about fallibility of mathematical theorems
%\label{ref:RNDEwhe4aJ9qw}(Lakatos, 1976, p.139).
\parencite[][p.139]{lakatos_proofs_1976}.%
\footnote{,,In fact Lakatos's \textit{quasi}-empiricism consists in the fallibility thesis. Although their subject matter is different, mathematical theories and empirical theories have in common the fact that they are fallible'' 
%\label{ref:RNDUrfHOQp6DB}(Koetsier, 1991, p.4).
\parencite[][p.4]{koetsier_lakatos_1991}.%
} Because since Popper attention has been paid to fallibility of statements advanced by empirical sciences, Lakatos' proposition about the fallibility of mathematical theorems is the first reason for choosing a name for the new direction in the philosophy of mathematics. Obviously, during the checking and testing stage, before they are provided with a correct ``Hilbertian'' proof, mathematical theorems are fallible.

Therefore, one can clearly see that with the scope of its investigation, the new philosophy of mathematics encompasses that which can be termed a ‘context of discovery.' And the differentiation between the context of discovery and the context of substantiation seems to provide a good tool for distinguishing a traditional philosophy of mathematics from \textit{quasi}-empiricism: while the traditional philosophy focused on the context of substantiation, the ``new'' philosophy found the context of discovery of mathematical theorems to be the fundamental object of its investigation, focusing its attention on the ``creative'' mathematician.

Secondly, Lakatos noted the reasonableness, the great relevance of the history of mathematics for studies concerned with philosophy of mathematics
%\label{ref:RNDWl4Qz050nC}(Lakatos, 1976, p.2).
\parencite[][p.2]{lakatos_proofs_1976}.
 It is not only and not so much about the reconstruction of crucial discoveries, on which Tymoczko laid so much emphasis, or about the history of the context of substantiation, which contains theorems with their correct and ready-made proofs. The emphasis was laid first and foremost on the history of the context of discovery, within which hypotheses were suggested, tested, and sometimes rejected; flawed proofs were presented for correct theorems, and if only in scientific correspondence and notes not intended for publication motives behind undertaken mathematical investigations were presented.

Understandably enough, it is interesting to ask whether the ``new'' philosophy of mathematics actually had its origins only in Lakatos' above-mentioned publications in the 1960s, or was there any pre-history to it. While asking about the pre-history of \textit{quasi}-empiricism, one should first and foremost take note of the authors who inspired Lakatos. These include Hegel, with his historism and dialectic, Popper, who emphasised the fallibility of statements made by empirical sciences, and Pólya, who specialised in heuristics of mathematics. Since the philosophy of mathematics was not among the pursuits of the former two, only the Hungarian mathematician and philosopher can be reckoned among those who originated the pre-history of \textit{quasi}-empiricism. Also, Wittgenstein, as represented by the late period of his activity, and Quine are usually mentioned. Certainly, account should also be taken of the lifework of Hadamard
%\label{ref:RNDwSdgR32pBO}(1945)
\parencite*[][]{hadamard_psychology_1945}
 and Poincaré 
%\label{ref:RNDkRcc7QWJnb}(1904),
\parencite*[][]{poincare_valeur_1904},
 who pursued, \textit{inter alia}, ``non-deductive,'' ``psychological'' aspects of mathematicians' ``output.''

Therefore, as already stated, even though the threads of the ``new'' philosophy were invariably present in the history of deliberation on mathematics, for more than half a century, until Lakatos' publications, they were ``muffled'' by philosophy confederated with the study of the foundations\footnote{It must be emphasised, once again, that the ``Creative Mathematician'' relying on intuition and imagination has never been out of interest for the philosophy of mathematics. This aspect of considering mathematics was represented by Husserl, and under his influence by the French school of philosophy of mathematics---starting from Poincaré. Husserl's phenomenology (intuition) was invoked in Gödel's attempt to explain the ``perception'' of the objective infinite hierarchy of infinite sets. However, none of them stressed as strongly the fallibility of the theorems of the mathematics as Bôcher and Lakatos did. Since the thesis of the fallibility of mathematical theorems is the essence of Lakatos' \textit{quasi}-empiricism, the roots of \textit{quasi}-empiricism are to be found in Bôcher.}.

\section{Elements characteristic of the later \textit{quasi}-empiricism in Bôcher's work}
It has already been mentioned that the first, fundamental part of the text by the American mathematician is distinctly determined by the way (and in part by the ``content'') of Frege's, Russell's and Hilbert's deliberation on mathematics. This part culminates in specifying mathematics with objects it investigates and with the method applied in it. Bôcher concludes that mathematics is a science of abstract objects, where (a method of) deductive reasoning is used.\footnote{``[…] I have insisted merely on the rigidly deductive form of reasoning used and the purely abstract character of the objects considered in mathematics''
%\label{ref:RNDyycHgIMS7K}(Bôcher, 1904, p.132).
\parencite[][p.132]{bocher_fundamental_1904}.%
}

\subsection{3.0. A metaphilosophical declaration}

Once the first, main part of his work is over, Bôcher immediately goes on to state that the approach to mathematics presented in his work is a ``discussion'' of the ``so-called foundations.''\footnote{``In fact I should like to subscribe most heartily to the view that in mathematics, as elsewhere, the discussion of such fundamental matters derives its interest mainly from the importance of the theory of which they are the so-called foundations''
%\label{ref:RNDmVdNhCmc3H}(Bôcher, 1904, pp.132–133).
\parencite[][pp.132–133]{bocher_fundamental_1904}.%
} Right there and then he adds that such a perception of mathematics is of ``an extremely one-sided character.''\footnote{``I fear that many of you will think that what I have been saying is of an extremely one-sided character'' 
%\label{ref:RNDRM5C9AWYYy}(Bôcher, 1904, p.132).
\parencite[][p.132]{bocher_fundamental_1904}.%
} That is why within the text under analysis he proposes an outline of yet another way of reflection on mathematics. He titles it: ``non-deductive elements in mathematics.''\footnote{Seventh section of Bôcher's paper was entitled ``The Non-Deductive Elements in Mathematics'' 
%\label{ref:RNDVbp83L8c52}(Bôcher, 1904, p.132).
\parencite[][p.132]{bocher_fundamental_1904}.%
}

With the benefit of the above remarks as well as the very layout of Bôcher's text one can presuppose that his metaphilosophy of mathematics ``contained'' the following propositions:

\begin{enumerate}
\item The ``discussion [...] of the so-called foundations'' is a new and essential direction of reflection on mathematics.
\item However, it is extremely one-sided and inadequate for complete philosophical deliberation on mathematics;
\item That is why it needs to be complemented by another perception of mathematics, one that takes into account---to stay in keeping with Bôcher's terminology---(only) that which is non-deductive in it.
\end{enumerate}
With the benefit of the contextual terminology, which was introduced before, and the remark whereby logicism, formalism and intuitionism were focused on the study of the context of substantiation in mathematics, as well as the results of the analyses performed in the paper
%\label{ref:RND15HRIVaRP6}(Dadaczyński, 2015),
\parencite[][]{dadaczynski_tendencje_2015},
 one can say that what Bôcher outlines in the first part of his work---as an approach to mathematics that is a ``discussion [...] of the so-called foundations''---is in fact a (one-sided) proposal to study the context of substantiation. This context can also be referred to as---to elaborate Bôcher's terminology (the term ``non-deductive'' to be specific) –a deductive aspect of mathematics.

The alternative reflection on mathematics proposed by Bôcher is supposed to take into account---according to his metaphilosophical declaration---(only) a non-deductive aspect of mathematics. Further analysis in the present paper is aimed at answering the question: does the non-deductive aspect of mathematics, singled out by Bôcher, ``coincide'' with the context of discovery in mathematics?

\subsection{3.1. A ``creative mathematician''---significance of imagination}

Bôcher begins his characterisation of the non-deductive elements of mathematics with a comparison. He writes that he likes to perceive mathematics as more of an art than science. In his opinion, a mathematician is an artist led by, though not controlled, by the external, sensual world. Creativity thus construed bears a real resemblance to the activity engaged in by an artist, e.g. a painter. Bôcher likens the mathematician's capacity to pursue deductive reasoning to the artist's skill at mastering painting techniques. When properly developed, the said capacity is a \textit{sine qua non} for being a mathematician, just like the skill at mastering suitable painting techniques is a \textit{sine qua non} for becoming a painter. However, these capabilities are not sufficient conditions for becoming either a ``full-blooded'' mathematician or a ``full-blooded'' painter; what is more---in Bôcher's opinion---they are not even the most important skills conditioning becoming either of the two. Other predispositions play a crucial role, but above all imagination---in both cases---is a more vital factor ``making'' someone a ``good mathematician'' or a ``good painter.''\footnote{``I like to look at mathematics almost more as an art than as a science; for the activity of the mathematician, constantly creating as he is, guided though not controlled by the external world of the senses, bears a resemblance, not fanciful I believe but real, to the activity of an artist, of a painter let us say. Rigorous deductive reasoning on the part of the mathematician may be likened here to technical skill in drawing on the part of the painter. Just as no one can become a good painter without a certain amount of this skill, so no one can become a mathematician without the power to reason accurately up to a certain point. Yet these qualities, fundamental though they are, do not make a painter or a mathematician worthy of the name, nor indeed are they the most important factors in the case. Other qualities of a far more subtle sort, chief among which in both cases is imagination, go to the making of the good artist or good mathematician''
%\label{ref:RNDfe2UhwlTxs}(Bôcher, 1904, p.133).
\parencite[][p.133]{bocher_fundamental_1904}.%
}

The fact that Bôcher regards a (good) mathematician primarily as an artist is of relevance to the research in hand. And here apparently lies the fundamental similarity between Bôcher's reflection on mathematics and later reflection by many representatives of \textit{quasi}-empiricism. The latter ones were primarily interested in the ``creative'' mathematician, his process of obtaining crucial results. They tried to describe the ``creative'' mathematician and to isolate---as far as possible---the procedures constituting mathematical creativity.

In his brief outline of the issue, Bôcher undertakes essentially the same task. His results could be summarised as follows: mathematics is more of an art than science; there are real similarities between the mathematician's work and the work done by an artist (e.g. a painter); in both cases it is (creative) imagination that is of crucial relevance for the value of the results.

It needs to be noted that Bôcher does not go on to belabour the comparison between a mathematician and an artist. But it appears that further conclusions with regard to the ``creative'' mathematician might be drawn from this comparison. Bôcher merely remarks that such a perception of the ``creative'' mathematician essentially allows for introduction of the notion of value among the instruments for evaluating mathematical achievements, just like it is done when evaluating a work of art.\footnote{,,Other qualities of a far more subtle sort, chief among which in both cases is imagination, go to the making of the good artist or good mathematician. I must content myself by merely recalling to you this somewhat vague and difficult though interesting field of speculation which arises when we attempt to attach \textit{value} to mathematical work, a field which is familiar enough to us all in the analogous case of artistic or literary criticism''
%\label{ref:RNDp1pZGnaAyW}(Bôcher, 1904, p.133).
\parencite[][p.133]{bocher_fundamental_1904}.%
} He himself speaks about a ``good mathematician'' in the sense in which one speaks about a ``good artist.''

In this case Bôcher does not follow up on the thread. With the expression of a ``good mathematician'' he suggests that a ``creative'' mathematician should be subjected to assessment, but obviously the assessment must be made by means of evaluating his creative output, that is by means of evaluating ``created'' mathematical objects, their entire structures and perceived properties of these objects, captured in the form of theorems, or alternatively hypotheses. One might only surmise that an evaluation of a mathematician's ``product'' could include such criteria as: boldness, conventionality/unconventionality, a (new) perspective when approaching the existing issues, a level of abstraction.

At any rate---and this should be strongly emphasised---given the present study indicating that Bôcher addressed essential threads characteristic of the much later \textit{quasi}-empiricism, it is vital to show that Bôcher stressed the question of a ``creative'' mathematician, which was of such great import for that direction. One might add that it was exactly addressing this issue that later on came to bear on the shift of emphasis in the philosophy of mathematics effected by representatives of \textit{quasi}-empiricism in relation to the earlier philosophy of mathematics pursued in ``the spirit of foundations.''

But it is exactly the latter statement that is potentially debatable. After all, it was within the framework of intuitionism---as well as other constructivist trends---that unusually dogmatic emphasis was placed on the issues concerned with creation---constructing mathematical objects. And it was the subject who was supposed to do the constructing.

However, leaving aside the fact that there was no consensus with regard to specifying the construction, it must be concluded that within the framework of intuitionism there was no answer to the question: how does the subject (mathematician) do the constructing, and to be more precise, what are the subjective determinants of being a ``creative'' (constructing) mathematician. It was exactly these questions that were posed, and attempts were made at answering them within the framework of \textit{quasi}-empiricism and in Bôcher's text. Besides, constructivism would sometimes even ``hinder'' intuitionists by generating ``proof-theory'' restrictions. It is sufficient to point to the rejection---for constructivist reasons---of indirect proofs of existential theorems. A mathematician hindered by those principles certainly did not match the vision of a ``liberated'' mathematician-artist as outlined by Bôcher.

\subsection{3.2. Intuition}

As he goes on describing an alternative way of reflection on mathematics, which is different from the perception of mathematics from the perspective of the ``so-called foundations,'' Bôcher notes the role of intuition among the instruments used by a ``creative'' mathematician. But here his remarks are only sketchy. As a matter of fact, he only singles out geometric, mechanical and physical intuition, saying that the significance of intuition among the instruments used by a creative mathematician is so popularly known that a mere mention of it is enough.\footnote{,,A discussion and analysis of the non-deductive methods which the creative mathematician really uses would be both interesting and instructive. Here I must content myself with the enumeration of a few of them. First and foremost, there is the use of intuition, whether geometric, mechanical, or physical. The great service which this method has rendered and is still rendering to mathematics both pure and applied is so well known that a mere mention is sufficient''
%\label{ref:RND4OfAwXmUjB}(Bôcher, 1904, p.134).
\parencite[][p.134]{bocher_fundamental_1904}.%
}

Bôcher wrote the text before the emergence of intuitionism, in which (pre-)intuitionism of time takes on a special significance as a reference to the Kantian \textit{a priori} forms of immediacy of time, and is supposed to be the only \textit{a priori} on which Brouwer was trying to construct his mathematics. That is why as one tries to clarify the meaning of the term \textit{intuition} in Bôcher's work, one should refer to his traditional, epistemological meanings which had accompanied the term already in Descartes' and Leibniz's works. It is about direct, non-discursive cognition, and sometimes even about something that can be termed intellectual obviousness. Leaving aside the classical terminology, one could even speak about a skill at guessing, or ``seeing'' mathematical states of affairs expressed only later on with the aid of theorems and proofs appended (discursive cognition), or about some kind of a mathematical ``nose.''

At any rate, with regard to the present discussion, it is essential to state that in reflection on mathematics, from an alternative viewpoint in relation to the perception of mathematics from the perspective of the ``so-called foundations,'' Bôcher very firmly called for inclusion of such a crucial element as intuition among the instruments used by a creative mathematician.

\subsection{3.3. Experiments in mathematics}

In his alternative perception of mathematics, Bôcher takes account of the role of experiment, which he finds to be essential, in a mathematician's working practices. It must be firmly stressed that in one sentence he writes about both physical experiments (in a laboratory) and arithmetic experiments, and in both cases he uses the same English term of \textit{experiment}
%\label{ref:RNDHHiG9oGXtb}(Bôcher, 1904, p.134).
\parencite[][p.134]{bocher_fundamental_1904}.
 Therefore, it is reasonable to conclude that he could see, just like Lakatos later on, some essential relations between the (laboratory) procedures applied in physics and some procedures adopted in arithmetic.

Bôcher stresses that he means above all experiments in number theory (as well as in analysis). He says that experiment records---which, as one might believe, corroborate some statements (hypotheses) in number theory---often used to be contained in mathematicians' publications. He goes on to add that the fact that in his times the practice of publishing the records of such activities was abandoned does not change, in his opinion, the fact that an arithmetical experiment---one might add: in the context of mathematical discovery (to be more precise: of a number-theory character), in study practice---was as common as before.\footnote{,,Then there is the method of experiment; not merely the physical experiments of the laboratory or the geometric experiments I had occasion to speak of a few minutes ago, but also arithmetical experiments, numerous examples of which are found in the theory of numbers and in analysis. The mathematicians of the past frequently used this method in their printed works. That this is now seldom done must not be taken to indicate that the method itself is not used as much as ever''
%\label{ref:RNDQdKDMpRbg8}(Bôcher, 1904, p.134).
\parencite[][p.134]{bocher_fundamental_1904}.%
}

It is worth subjecting the above statement of Bôcher's, concerned with a mathematical (number-theory) experiment, to more accurate analysis. Such an experiment often takes the following form: a number-theory statement of a form, for instance, like this:
\begin{equation}
\prod a,b,c\ (F(a,b,c) = 0),
\end{equation}
where $a$, $b$, $c$ are variables belonging to the set of natural numbers, is attributed a certain degree of probability. And then substituting concrete natural numbers for variables and performing numerical operations are used to check whether the expression contained within the universal quantifier is true.

In general, two cases are possible, and they always need to be taken into account by him who is performing the number-theory experiment: with the natural numbers substituted for the variables the expression (1) is either true, or it is not. In the first case corroboration (to use Popper's terminology) of the statement (1) takes place, while the second case is one of falsification (to use the same terminology) of the ``experimental''---and hence hypothetical---statement (1).

Falsification process for statement (1) is conducted according to the law of \textit{modus tollens}:
\begin{equation}
\begin{split}
(\prod a,b,c\ (F(a,b,c) = 0) \to F(a_{k}, b_{l}, c_{m}) = 0)\\
\land\ \neg (F(a_{k}, b_{l}, c_{m}) = 0)
\to \neg \prod a,b,c\ (F(a,b,c) = 0),
\end{split}
\end{equation}
where $a_{k}$, $b_{l}$, $c_{m}$ are fixed natural numbers.

With regard to the present study, it is essential that as Bôcher fully accepts an experiment among the mathematician's instruments, above all but not only in the field of number theory investigations, and \textit{explicite} declares its relations with a physical experiment in a laboratory, he points, \textit{implicite} – but as one might surmise consciously---to the fact that in mathematical practice some mathematical statements are accepted ``experimentally'' as hypotheses and---as one might add today: in the context of mathematical discovery---some elements of mathematical knowledge (some statements in which this knowledge is expressed) are fallible.

As demonstrated before, the statement whereby some mathematical propositions are fallible and as such may be subjected to testing was the main theorem of Lakatos' \textit{quasi}-empiricism, inspired by the thesis of fallibility of physical knowledge and the thesis of the actual application of falsification processes in the field of physics, the theses having been formulated by Popper. And therefore, by advancing---though only \textit{implicite} by addressing the relevance of an experiment in mathematics (and in particular in number theory)---the issue of fallibility of some parts of mathematical knowledge, Bôcher emphasised another crucial thesis characteristic of the later \textit{quasi}-empiricism of the 1960s.

Obviously, one should remember that experiments of this kind have featured and will always feature in some branches of mathematics. And it is no discovery by Bôcher or Lakatos. Something else is to be owed to them: pointing to the necessity to take account of mathematical experiments and hypotheses in the field of philosophical study of mathematics. Quite obviously they belong to the scope of the context of discovery, and within the study of classical directions of the philosophy of mathematics that focused on the context of substantiation they were not addressed.

\subsection{3.4. (Incomplete) induction}

Another element of Bôcher's perception of mathematics from an alternative viewpoint is about taking into account inductive reasoning that actually features among the mathematician's ``instruments.'' Bôcher writes about ``the ordinary method of induction,'' thereby distinguishing it---as one might suppose---from mathematical induction and juxtaposing it with deduction, which he used to characterise mathematics from the viewpoint of ``the so-called foundations.'' Bôcher finds that applying induction in mathematical practice is often correlated with the use of experiment, which is analysed in the present paper above.\footnote{``Closely allied to this method of experiment is the method of analogy which assumes that something true of a considerable number of cases will probably be true in analogous cases. This is, of course, nothing but the ordinary method of induction. But in mathematics induction may be employed not merely in connection with the experimental method, but also to extend results won by deductive methods to other analogous cases. This use of induction has often been unconscious and sometimes overbold, as, for instance, when the operations of ordinary algebra were extended without scruple to infinite series''
%\label{ref:RNDAcpy86tM1U}(Bôcher, 1904, p.134).
\parencite[][p.134]{bocher_fundamental_1904}.%
} Indeed, for instance in number theory---one should add: in the work done in the context of discovery---to make a certain statement probable, say one like (1), it is put to a series of tests involving concrete substitutions, which can be regarded as an application of ``ordinary'' induction.

\subsection{3.5. ``A method of optimism''}

Another element of Bôcher's alternative perception of mathematics is about taking into account that which he calls ``a method of optimism.'' It makes it possible to ``shut our eyes to the possibility of evil''---as Bôcher claims---and a rapid development of many branches of mathematics may be the benefit here. Bôcher provides the following example of the application of this method: ``I know that I have no right to divide by zero; but there are so many other values which the expression by which I am dividing might have that I will assume that the Evil One has not thrown a zero in my denominator this time.''\footnote{``Finally, there is what may perhaps be called the method of optimism which leads us either willfully or instinctively to shut our eyes to the possibility of evil. Thus the optimist who treats a problem in algebra or analytic geometry will say, if he stops to reflect on what he is doing : ‘I know that I have no right to divide by zero; but there are so many other values which the expression by which I am dividing might have that I will assume that the Evil One has not thrown a zero in my denominator this time'. This method, if a proceeding often unconscious can be called a method, has been of great service in the rapid development of many branches of mathematics, though it may well be doubted whether in a subject as highly developed as is ordinary algebra it has not now survived its usefulness''
%\label{ref:RNDP8LXuuoscK}(Bôcher, 1904, pp.134–135).
\parencite[][pp.134–135]{bocher_fundamental_1904}.%
}

In this way one can illustrate the ``method of optimism'' that Bôcher discerns in mathematical practice; in his opinion the method was instrumental in the rapid development of many branches of mathematics. Noteworthily, he notes that in many specific cases mathematicians were not aware of the application of this ``method'' as a method, as they tried to achieve projected results as quickly as possible. Bôcher goes as far as to claim that some kind of instinct might have been involved. Bearing in mind the previous remarks, one might add that the instinct might have been connected with the intuition and imagination of a mathematician who ``sees'' the goal of his actions and who believes that technical improvement of the ``road'' leading to the goal is of great but not utmost importance.

Such actions, inspired by the ``method of optimism,'' are completely different from the course of action pursued in formalism, logicism and intuitionism. The former two did not allow any ``gaps'' in proving---however, Frege lucidly defined a proof in logic, and Hilbert clearly indicated what a proof in mathematics is. Both worked with axiomatic systems, unlike Brouwer, who claimed that a subject's creative thought cannot be axiomatised. Still, the way of proving in Brouwer's intuitionism was in a sense even more restrictive and rigorous. It is sufficient to mention the rejection of indirect proofs of existential theorems on account of the fact that they did not provide the construction of the object the existence of which they were supposed to substantiate. However, it must be borne in mind that normative schools of philosophy of mathematics set themselves tasks completely different than the ``upward'' development of mathematics, or exploring its new areas. They were pursuing their own philosophical goals concerned with its foundations, and so they had to be rigorous and meticulous in their---and this needs to be strongly emphasised---reconstructions of mathematics as actually pursued.

This point calls for a recapitulation of the study conducted thus far. As early as the beginning of the 20\textsuperscript{th} century Bôcher did not hold with perceiving mathematics solely in the spirit of---to use his own expression---``the so-called foundations.''\footnote{``This explains how, again and again, it has come about, that the most important mathematical developments have taken place by methods which cannot be wholly justified by our present canons of mathematical rigor, the logical ``foundation'' having been supplied only long after the superstructure had been raised. A discussion and analysis of the non-deductive methods which the creative mathematician really uses would be both interesting and instructive''
%\label{ref:RNDBjpQdSo63Y}(Bôcher, 1904, p.134).
\parencite[][p.134]{bocher_fundamental_1904}.%
} He claimed that the analysis of the phenomenon of mathematics should also necessarily include a phenomenon of a ``creative'' mathematician. In his opinion, imagination and intuition are extremely important in the description of a ``creative'' mathematician. Owing to the former, creative pursuit of mathematics is similar to an art rather than science. He considered taking into account non-discursive cognitive methods---intuition---in the research field of philosophy of mathematics to be unquestionable. Bôcher claimed that in their quests mathematicians refer to experiments. It must be stressed that in one sentence he mentioned a physical experiment performed in a laboratory and an arithmetical experiment. His pointing to number-theory experiments makes it possible, as part of the analyses performed in the present paper, to establish that Bôcher, at least \textit{implicite}, agreed to (temporal) fallibility of some parts of (number-theory) mathematical knowledge and, in at least this scope, he accepted the application of falsification processes in mathematics. Thus, in the research field of his alternative philosophy of mathematics he included mathematical hypotheses. The American scholar pointed to the application of processes of induction---as opposed to deduction---in the practice of mathematical research. He also outlined the so-called ``method of optimism,'' which in his opinion played a really crucial role among the instruments used by mathematicians ``creating'' new mathematical disciplines.

It needs to be clearly highlighted that both emphasising the fallibility of mathematical knowledge and the ability to discern the reference in mathematics to the processes of falsification and induction, as opposed to deduction, as well as concentration on the description of the work of a ``creative'' mathematician, including his imagination and intuition, belong to the ``hard core'' of the later \textit{quasi}-empiricism.

The above statements lead to two vital---from the viewpoint of the present study---conclusions:

\begin{enumerate}
\item An alternative variant of deliberation on mathematics---outlined by Bôcher---as well as the later \textit{quasi}-empiricism, with which, as it has already been established, the variant coincides in many respects---focuses on the research into the context of discovery. By way of reference to Bôcher's terminology, one might say that the non-deductive aspect of mathematics, which he emphasised, is in essence the context of discovery.
\end{enumerate}
2. On account of the fact that Bôcher undertakes reflection on the non-deductive aspect of mathematics, he can be reckoned among the thinkers creating the pre-history of \textit{quasi}-empiricism.

\section{A ``historicised'' philosophy of mathematics}
It has been demonstrated above that Bôcher called for an alternative way of philosophical reflection on mathematics, as opposed to the ``discussion [...] of the so-called foundations,'' to use his own expression. It has been found that this way was supposed to allow for the aspects of mathematics that came to lie outside the field of research conducted in the ``spirit of foundations,'' and that the alternative philosophical proposal coincided with the trend of the philosophy of mathematics emphasised by Lakatos.

However, it is worth noting that so far in Bôcher's concept no imperative to include the history of mathematics in the alternative approach to the philosophy of mathematics has been pointed to, which was one of the component parts of \textit{quasi}-empiricism.

The explanation is as follows: As he outlined the concept of an alternative version of the philosophy of mathematics, Bôcher did not include a metaphysical demand that the results of research into the history of mathematics should be included. Nevertheless, he himself, in the text under examination, makes several references to the history of mathematics so as to consolidate the theses with regard to the proposed---alternative---philosophy of mathematics.

That is why one may claim that although Bôcher \textit{explicite} did not combine the alternative philosophy of mathematics with a metaphilosophical demand for inclusion of the history of mathematics in it, he as a matter of fact realised this demand.

An example of such realisation is a short sketch included in Bôcher's text under discussion, which shows the ways in which Cauchy's, Abel's and Weierstrass' works contributed to the consolidation of the foundations of mathematical analysis. Bôcher uses the historical reference to show how the mathematical theories which were already ``discovered,'' but were not fully ``substantiated,'' came to be consolidated. And then he ventures a more general reflection that certainly is of a philosophical character---it will be presented here with the aid of ``contextual'' terminology. Namely, Bôcher essentially concludes that the borderline between the context of discovery and the context of substantiation has not been in the history of mathematics some kind of \textit{constans}, but a changeable function of time. That is to say, in other words: the standards of that which is scientific and non-scientific, the requirements concerned with acceptance or non-acceptance of certain processes (of one proof or another) within the context of substantiation (in mathematics) are not absolute.\footnote{``We are in the habit of speaking of logical rigor and the consideration of axioms and postulates as the foundations on which the superb structure of modern mathematics rests; and it is often a matter of wonder how such a great edifice can rest securely on such a small foundation. Moreover, these foundations have not always seemed so secure as they do at present. During the first half of the nineteenth century certain mathematicians of a critical turn of mind---Cauchy, Abel, Weierstrass, to mention the greatest of them---perceived to their dismay that these foundations were not sound, and some of the best efforts of their lives were devoted to strengthening and improving them. And yet I doubt whether the great results of mathematics seemed less certain to any of them because of the weakness they perceived in the foundations on which these results are built up. The fact is that what we call mathematical rigor is merely one of the foundation stones of the science; an important and essential one surely, yet not the only thing upon which we can rely. A science which has developed along such broad lines as mathematics, with such numerous relations of its parts both to one another and to other sciences, could not long contain serious error without detection. This explains how, again and again, it has come about, that the most important mathematical developments have taken place by methods which cannot be wholly justified by our present canons of mathematical rigor, the logical ‘foundation' having been supplied only long after the superstructure had been raised''
%\label{ref:RNDucNQpwEv1l}(Bôcher, 1904, pp.133–134).
\parencite[][pp.133–134]{bocher_fundamental_1904}.%
}

At any rate, with regard to the present study, it is significant that even though Bôcher did not formulate a metaphilosophical imperative to make use of the findings of the history of mathematics in the philosophy of mathematics, in his deliberation on mathematics he made references to those findings. This serves to reinforce the thesis that Bôcher's alternative variant of the philosophy of mathematics is in essential respects coincident with the later \textit{quasi}-empiricism.

\section{Conclusion}
Taking into account the findings of the paper
%\label{ref:RNDJzGF41Hy7P}(Dadaczyński, 2015)
\parencite[][]{dadaczynski_tendencje_2015}
 and the results of the above-performed analyses, one can draw the following conclusions concerned with the ``content'' of Bôcher's metaphilosophy of mathematics:

\begin{enumerate}
\item Two aspects of mathematics ought to be distinguished: a deductive and a non-deductive one;
\item The deductive aspect of mathematics essentially coincides with the object of the normative studies concerned with the philosophy of mathematics, which emerged as late as the turn of the 19\textsuperscript{th} and 20\textsuperscript{th} centuries---by referring to the contextual terminology, it was pointed out that it was the context of substantiation;
\item The non-deductive aspect of mathematics essentially coincides with the object of study of the philosophy of mathematics determined by the later \textit{quasi}-empiricism---by referring to appropriate terminology, it was pointed out that it was the context of discovery;
\item The study of the deductive aspect is an inalienable element of deliberation on mathematics;\footnote{``I fear that many of you will think that what I have been saying is of an extremely one-sided character, for I have insisted merely on the rigidly deductive form of reasoning used and the purely abstract character of the objects considered in mathematics. These, to the great majority of mathematicians, are only the dry bones of the science. Or, to change the simile, it may perhaps be said that instead of inviting you to a feast I have merely shown you the empty dishes and explained how the feast would be served if only the dishes were filled. I fully agree with this opinion, and can only plead in excuse that my subject was the \textit{fundamental} conceptions and methods of mathematics, not the infinite variety of detail and application which give our science its real vitality''
%\label{ref:RNDiVKCDnjedh}(Bôcher, 1904, p.132).
\parencite[][p.132]{bocher_fundamental_1904}.%
}
\item Taking into account the deductive aspect (the context of substantiation) only is one-sided and inadequate for the philosophical description of the phenomenon of mathematics;
\item The philosophical deliberation on the phenomenon of mathematics should necessarily allow for the non-deductive aspect (the context of discovery);\footnote{``This explains how, again and again, it has come about, that the most important mathematical developments have taken place by methods which cannot be wholly justified by our present canons of mathematical rigor, the logical ``foundation'' having been supplied only long after the superstructure had been raised. A discussion and analysis of the non-deductive methods which the creative mathematician really uses would be both interesting and instructive''
%\label{ref:RND9yNU8hpl5N}(Bôcher, 1904, p.134).
\parencite[][p.134]{bocher_fundamental_1904}.%
}
\item Both the aspects are complementary to each other---an analysis of them provides a fuller picture of mathematics from the philosophical perspective.\footnote{``While no one of these methods can in any way compare with that of rigorous deductive reasoning as a method upon which to base mathematical results, it would be merely shutting one's eyes to the facts to deny them their place in the life of the mathematical world, not merely of the past but of today. There is now, and there always will be room in the world for good mathematicians of every grade of logical precision. It is almost equally important that the small band whose chief interest lies in accuracy and rigor should not make the mistake of despising the broader though less accurate work of the great mass of their colleagues; as that the latter should not attempt to shake themselves wholly free from the restraint the former would put upon them. The union of these two tendencies in the same individuals, as it was found, for instance, in Gauss and Cauchy, seems the only sure way of avoiding complete estrangement between mathematicians of these two types''
%\label{ref:RNDfxZs3AsNsD}(Bôcher, 1904, p.135).
\parencite[][p.135]{bocher_fundamental_1904}.%
}
\end{enumerate}
The above conclusions concerned with the ``content'' of Bôcher's metaphilosophy of mathematics lead to the conclusion that at the beginning of the 20\textsuperscript{th} century he outlined a concept of a complementary, two-aspect philosophy of mathematics.

Significantly, the study of the appropriate aspects (the deductive and the non-deductive one) suggested by Bôcher essentially coincides with those ways of deliberation on mathematics which---while keeping a suitable temporal distance from Lakatos' ``revolutionary'' publications, which not infrequently drew harsh criticism, or even negation of the normative value of non-descriptive approaches to mathematics---began to be treated (one should add: again) as complementary types of reflection, and even combined into one, complementary philosophy of mathematics.

The import of Bôcher's concept, which also results from its relevance, is by no means belittled by the fact that in the very same year when Bôcher's work was published, that is 1904, Poincaré's book
%\label{ref:RNDapLZGtW5ba}(Poincaré, 1904)
\parencite[][]{poincare_valeur_1904}
 was released in Paris; its first part contains a description of two aspects of mathematics (logic and intuition) characterised in a very similar manner, and to be more precise: aspects of mathematicians' working practices, with an addition of an emphatic, metaphilosophical imperative to include both in the deliberation on mathematics, so that its fuller image could be obtained.

It can be surmised that both the ``parallel'' concepts---of Poincaré's and Bôcher's---were a reaction to the one-sided accentuation of the ``modern'' way of pursuing the philosophy of mathematics as a ``discussion of the so-called foundations,'' which emerged at the turn of the 19\textsuperscript{th} and 20\textsuperscript{th} centuries in some leading milieus of mathematicians and philosopher-mathematicians---gathered primarily around Frege, Russell, Hilbert, as well as Italian geometricians along with Peano, and pre-intuitionists---while at the same time the milieus were exposed (at least methodologically) to indispensable aspects of mathematics as actually pursued.

\end{artengenv}

