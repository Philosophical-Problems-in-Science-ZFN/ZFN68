\begin{recengenv}{Michael Dominic Taylor}
	{The hidden theological and philosophical presuppositions motivating the modern cosmological discussion: Why both sides are wrong}
	{The hidden theological and philosophical presuppositions motivating the modern cosmological discussion: Why both sides are wrong}
	{David Alcalde, \textit{Cosmology Without God?: The Problematic Theology Inherent in Modern Cosmology}, Cascade, Eugene 2019, pp.226.}





David Alcalde contributes to the clarification of thought on the origin of the universe from the unique position of an author that holds doctorates in both astrophysics and theology. Critiques of modern science and its limitations are increasingly necessary and increasingly common, but this work is indispensible because it approaches its subject matter from both sides of the argument. Alcalde addresses the insufficiencies of scientism and positivism cogently while explaining the unavoidability of metaphysical and theological presuppositions held by atheistic cosmologists, but he also succeeds in dismantling the unconsciously and improperly assumed presuppositions of cosmologists claiming to argue from a position of faith in favor of the beginning of the universe. Thus, this book reads both as a philosophical critique of modern science, and as a manual for Christian scientists who may be guilty of failing to comprehend what they claim to promote.

In a clear and coherent style, the author unravels the myth of modern science',s supposed neutrality, while, at the same time, opening up the world of cosmological theory to readers from all backgrounds. Knowledge of highly technical language is not a requirement for comprehending the descriptions the author offers of the most important cosmological theories in discussion today: the Big Bang model, with its modifications of dark matter, inflation, and dark energy; the steady-state model; the Penrose-Hawking singularity theorem, which incorporates the theory of quantum tunneling; the no-boundary proposal of Hartle and Hawking; the so-called “creation out of nothing'', model from Vilenkin, within which, nothing is actually still something; the cyclic universe, promoted by Steinhardt and Turok; the multiverse hypothesis; and even the proposal that the universe was created by a designer (or designers) who are not gods but highly evolved aliens.

Theological extrinsicism, the position that holds that God is extrinsic and outside of nature and therefore has no relevance to scientific problems, is prevalent in both camps who compose two sides of the same coin. On the one hand, openly atheistic scientists -- Stephen Hawking, perhaps the most representative among them -- go through a great deal of effort to avoid what would appear to be a beginning to the universe, because to them, this temporal beginning is too similar to what they perceive to be the creation event of the Judeo-Christian tradition. From the “steady-state model'', to Hawking',s “no boundary proposal'', and the multiverse, Alcalde describe in detail, yet without heavy academic jargon, the many efforts employed by atheists to defend their theological beliefs. The problem, however, is that these scientists fail, philosophically and theologically, to understand the doctrine of \textit{creatio ex nihilo}. Thus, these scientists find themselves battling a straw man and sacrifice the scientific rigor of providing testable hypotheses in order to do so. The theological and philosophical presuppositions of atheistic cosmologists betray their scientific intentions.

However, what is perhaps the most valuable contribution of Alcalde',s exposition is his critique of those cosmologists who claim to argue in favor of a Christian position. In the case of many of the most vocal advocates, Christian cosmologists also fail dismally to defend the philosophical doctrine of \textit{creatio ex nihilo} and defend the theological extrinsicism that atheists reject. Principal among these thinkers are the philosophical theologian William Craig and the Jesuit Robert Spitzer who both defend the \textit{kalam} cosmological argument, which bears within it a mechanistic conception of the universe. Thus, the author reveals that mechanism is not only a problem for science but also one for theology.

Alcalde',s systematic description of the arguments employed on both sides of this theological extrinsicism reveal the inescapable need for philosophical literacy, both to preserve the scientific pursuit and to prevent theology from slipping into the temptation of claiming empirical evidence as its own justification. The philosophical concept of \textit{nihil} is perhaps the \textit{crux} of Alcalde',s argument. It serves both: to express the gratuity of all of creation and it removes any limitation from the act of creation itself. For both atheists and Christians, comprehending creation in this light is essential to a proper understanding of God, for, as Thomas Aquinas pointed out nearly eight centuries ago, “…an error about creation is reflected in a false opinion about God'', (\textit{Summa Contra Gentiles}, II, 3.1). As the author shows, the simple yet radical assertion of the real distinction between essence and existence is the key to disentangling scientific investigation from theological extrincisim. Being cannot be taken for granted, as sheer facticity that is “just there.'', But neither can scientific evidence be offered for the existence of God without predetermining an extrinsic conception of God that is un-Trinitarian and therefore, not Christian, because creation and the temporal origin of the universe are two separate phenomena that occur on two separate levels.

Scientists, regardless of their theological beliefs, cannot ignore the author',s cogent presentation of the doctrine of \textit{creatio ex nihilo}, which must bear on human reason in a recognition of the logical necessity of the metaphysical question par excellence: why is there something rather than nothing? The alternative is a scientific position either elevated to a rank its condition ought not bear (as evidence of an omnipotent creator, who is not the Christian God but a “god of the gaps'',) or pushed into science fiction (e.g. a multiverse or alien designers) by an “atheism of the gaps.'', In this sense, the “warm reception'', the theory of the Big Bang has received by theologians is misguided and misleading, as it could never be equated with the event of creation. In both cases, scientific rigor is sacrificed in favor of theological extrinsicism, in its positive or negative sense, which the author whole-heartedly rejects. In the end, what Alcalde is fighting for is a true science, unassail by theological Trojan horses, not because is feigns neutrality and indifference towards philosophy and theology, but because it recognizes its own proper place in the hierarchy of human knowledge.



\autorrec{Michael Dominic Taylor}

\end{recengenv}