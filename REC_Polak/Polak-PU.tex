\begin{recplenv}{Paweł Polak}
	{Przedmioty wirtualne -- składnik naszego świata}
	{Przedmioty wirtualne -- składnik naszego świata}
	{Bondecka-Krzykowska I., Brzeziński K.M., Bulińska-Stangrecka H.,~i in., \textit{Przedmioty wirtualne}, red. P. Stacewicz, B. Skowron, Oficyna Wydawnicza Politechniki Warszawskiej, Warszawa 2019, ss.~135.}






Pojęcie ,,wirtualny'', choć używane dziś potocznie w~wielu różnych znaczeniach, w~filozofii łączy najczęściej refleksję ontologiczną z~filozoficznym namysłem nad techniką. Choć samo pojęcie wirtualności sięga korzeniami średniowiecznych zaawansowanych dystynkcji metafizycznych, to dziś nabrało nowej wagi z~racji rozpowszechnienia technik informatycznych i~roli, jaką odgrywają w~naszym życiu. Przedmioty wirtualne są więc jednym z~najciekawszych artefaktów współczesnej techniki -- daleko odchodzą bowiem od wyobrażeń o~ewidentnie materialnym przedmiocie działań technicznych. Rozszerzają one znacznie obszary technicznej manipulacji człowieka na obiekty będące tworami realizowanymi obliczeniowo. Choć ,,substrat'' tych przedmiotów wydawać się może ulotny, to ich własności ontologiczne oraz rola, jaką odgrywają w~naszym świecie trudno nazwać efemerycznymi lub ulotnymi. Stąd konieczna jest interdyscyplinarna refleksja nad nimi, którą w~recenzowanej książce podjęło grono badaczy związanych lub zaprzyjaźnionych ze środowiskiem filozoficznym Politechniki Warszawskiej. To, że uczelnie techniczne zaczynają gromadzić środowiska filozofów jest wymownym znakiem naszych czasów -- filozofia nie jest już tylko obowiązkowym przedmiotem humanistycznym, któremu muszą poświęcać się studenci studiów technicznych. Filozofia staje się jednym z~ważnych narzędzi dzisiejszej techniki. Wyrafinowane urządzenia i~systemy, jakie potrafimy dziś tworzyć, przynoszą wiele nowych problemów, wobec których dotychczasowy model uprawiania nauk technicznych jest bezradny, stąd bierze swe źródła zwrot ku filozofii we współczesnej inżynierii.

Recenzowana praca zbiorowa składa się z~dziewięciu rozdziałów, z~których większość (pięć) poświęconych jest różnym aspektom ontologicznym związanym z~przedmiotami wirtualnymi. Zbiór otwiera opracowanie autorstwa K. Brzezińskiego i~J. Lubacza, ,,Skąd się biorą przedmioty wirtualne?''. Artykuł zawiera cenne uwagi na temat pojęcia przedmiotów wirtualnych. Postawienie tytułowego zagadnienia pochodzenia przedmiotów wirtualnych prowadzi autorów do szczegółowych pytań, dotychczas rzadko podejmowanych w~literaturze. Interesujące i~oryginalne jest np. wyróżnienie klasy artefaktów wirtualnych w~oparciu o~metody projektowania, a~nie zbiory cech, jak się to dotychczas czyniło. Oczywiście praca przedstawia z~konieczności pewien subiektywnie dokonany wycinek tematyki -- stąd może rodzić pewne dyskusje, niemniej jest inspirująca do refleksji.

Izabela Bondecka-Krzykowska w~kolejnym rozdziale postawiła natomiast pytanie o~cechy obiektów rzeczywistości wirtualnej. Artykuł stanowi interesujący i~dobrze napisany syntetyczny przegląd cech obiektów rzeczywistości wirtualnej, a~zaproponowana metoda porównawcza wydaje się dobrze uzasadniona.

Kolejny rozdział, autorstwa Pawła Stacewicza pt. ,,Wirtualność w~perspektywie obliczeniowej'', stanowi ważną i~w dużej mierze nowatorską próbę analizy zagadnienia wirtualności z~perspektywy procesów obliczeniowych fundujących rzeczywistość wirtualną. Rozważania prowadzą Autora do interesujących tez, jak np. teza o~obliczeniowej stopniowalności wirtualności. Wnosi to cenną perspektywę do dyskusji nad zagadnieniem wirtualności, której wyraźnie brakowało w~polskiej literaturze. Co prawda nieco sztuczne jest przedefiniowanie pojęcia ‘cyfrowy'. W~nauce oraz w~technice najczęściej występuje ono jako synonim pojęcia ‘dyskretny'. Tutaj pojęcie ‘cyfrowy' utożsamione zostało z~pojęciem obliczeń Turingowskich, co jest niezbyt trafne terminologicznie -- pewien podzbiór operacji cyfrowych nazywamy modelem obliczeń Turinga, a~nie odwrotnie jak sugeruje Autor. Innymi słowy cyfrowy charakter operacji jest założeniem modelu Turinga, a~nie skutkiem jego przyjęcia.

Szczególnie interesujący filozoficznie dla mnie jest rozdział autorstwa Bartłomieja Skowrona, ,,O urealnianiu się przedmiotów wirtualnych''. Warto tu wspomnieć, że tekst ten jest rozwinięciem tez, zaprezentowanych po raz pierwszy podczas II Konferencji ,,Filozofia w~informatyce'' w~Krakowie. Pomysł Autora jest zdawałoby się aż nazbyt oczywisty -- zastosowanie Ingardenowskich narzędzi analizy ontologii do opisu przedmiotów wirtualnych. Ontologia egzystencjalna Ingardena słusznie uchodzi bowiem za jedno z~najciekawszych narzędzi wypracowanych we współczesnej filozofii. Zagadnienie to stało się jednak bardzo interesujące z~powodu pewnego -- jakże wymownego -- niepowodzenia. Okazuje się bowiem, że świat przedmiotów wirtualnych jest bardziej skomplikowany niż zakłada to koncepcja Ingardena. Okazuje się, że przedmioty wirtualne nie spełniają według Skowrona warunku stałości sposobu istnienia: ,,opisane zjawisko usamoistniania, o~ile zostało dobrze ujęte, przełamuje to ontologiczne prawo: przedmioty wirtualne usamoistniając się pozostają tymi samymi przedmiotami, choć są o~wiele bliżej realności, niż były u~początków swego istnienia'' (s.~73).

Kwestię realności przedmiotów wirtualnych, a~dokładniej kwestię odmowy możliwości przypisania realności przedmiotom wirtualnym w~pewnym określonym sensie, podjął Michał Głowala w~rozdziale ,,Cztery wersje antyrealizmu co do przedmiotów wirtualnych''. Autor przyjął interesujące założenie metodologiczne -- porzucił łatwiejszą drogę zastosowania ontologii warstw/poziomów (\textit{levels}), w~zamian za to próbując zastosować klasyczne narzędzia pojęciowe arystotelizmu, obecne we współczesnych analitycznych odmianach tegoż nurtu. Choć adekwatność pojęć arystotelesowskich może budzić również pewne zastrzeżenia, to Autor próbował w~miarę możliwości ukazać ograniczenia takiego opisu. Nie usuwa to zastrzeżeń wobec przyjętego aparatu, ale dobrze wskazuje granice klasycznego opisu stechnicyzowanej rzeczywistości, która jest środowiskiem życia ludzi XXI wieku.

Nieco inną perspektywę patrzenia na zagadnienie przedmiotów wirtualnych przyjął Jakub Jernajczyk w~rozdziale ,,Zasada wzorca i~kopii -- o~podobieństwie metod stosowanych w~rzemiośle, filozofii i~programowaniu''. Autor porusza bardzo interesującą kwestię wpływu modelu działania na różnorodne obszary ludzkiej aktywności: nie tylko na działalność o~charakterze technicznym (informatyka), ale i~na działalność o~charakterze \textit{stricte} teoretycznym jak filozofia. Analizy tego typu pojawiały się już w~literaturze, dość wspomnieć książkę \textit{Człowiek Turinga} J.D. Boltera (1990). Niektóre aspekty poruszone w~tym tekście są zbliżone do tych prezentowanych przez Boltera, natomiast tutaj rozważania prowadzone są w~bardzo interesującym kontekście programowania obiektowego i~filozofii uwikłanej w~ten paradygmat programowania. To decyduje o~oryginalności i~wartości wspomnianego tekstu. Jest to też praca mocno inspirująca do dalszych przemyśleń filozoficznych.

Marcin Trybulec w~rozdziale ,,Przedmioty wirtualne jako niedoskonałe narzędzia poznawcze'' stawia natomiast pytania odnośnie możliwości wykorzystania wirtualnej rzeczywistości jako narzędzia poznawczego. Zagadnienie to jest bardzo ważne i~aktualne w~kontekście przyszłości edukacji. Kwestia zastępowania rzeczywistych układów poprzez wirtualne odpowiedniki w~procesie nauczania budzi wiele wątpliwości, zatem potrzebna jest rzeczowa analiza tego zagadnienia, w~tym analiza filozoficzna. Co prawda z~dyskusjami nad możliwością zastępowania w~dydaktyce rzeczywistych układów laboratoryjnych dobrymi symulacjami stykam się od co najmniej ćwierć wieku, to muszę przyznać, że metodyczna refleksja filozoficzna na ten temat nie jest nazbyt rozwinięta. Negatywne stanowisko Autora wydaje się dobrze uzasadnione, choć dla lepszego wyrażenia znaczenia przeprowadzonych przez niego rozważań celowe mogło by być wprowadzenie odróżnienia treningu od świadomych, krytycznych czynności poznawczych. Do tego pierwszego techniki wirtualnej rzeczywistości nadają się doskonale, do drugiego -- jak przekonuje Autor -- nie pasują. Być może tak jasne postawienie sprawy pomoże ostudzić nieco zapał żarliwych zwolenników wprowadzania technik wirtualnej rzeczywistości do edukacji, uświadamiając wszystkim, jaki jest faktyczny zakres stosowalności tych metod.

Ostatnie dwa rozdziały wskazują natomiast społeczne perspektywy zastosowania wirtualności. Jacek Janowski w~rozdziale ,,Wirtualizacja prawa cywilizacji informacyjnej: Technologiczna i~ideologiczna symulacja zjawisk prawnych'' podejmuje próbę analizy rozwoju prawa w~kontekście technologii informatycznych. Autor próbuje wykorzystać koncepcje Baudrillarda do wykazania, że prawo deformuje się pod wpływem rozpowszechniania się technik wirtualizujących rzeczywistość. Natomiast Helena Bulińska-Stangrecka w~ostatnim rozdziale ,,Organizacja wirtualna w~naukach o~zarządzaniu: wybrane konsekwencje wirtualizacji'' ukazuje wprost socjologiczną perspektywę wirtualizacji rzeczywistości. To perspektywa znacząco odmienna od perspektywy pozostałych artykułów, niemniej może zostać potraktowana jako uzupełnienie obrazu o~przyczynki z~innych dziedzin.

Lektura książki jest interesującą wyprawą przez rozważania na temat przedmiotów wirtualnych. Cieszy fakt, że polska filozofia wzbogaciła się o~kolejne wartościowe opracowanie z~zakresu filozofii informatyki. W~czasie lektury rodzą się jednak nazbyt często pytania, czy poprzez publikację rozważań tylko w~języku polskim, nie zostaną one bez większego oddźwięku na arenie międzynarodowej. Polska filozofia w~przeszłości wielokroć cierpiała z~powodu bariery językowej. Miejmy nadzieję, że choćby część opisywanych rozważań zostanie w~przyszłości rozwinięta i~opisana w~języku angielskim, tak aby mogły stać się inspiracją dla innych badaczy, niekoniecznie urodzonych nad Wisłą czy nad Odrą.




\autorrec{Paweł Polak}

\end{recplenv}